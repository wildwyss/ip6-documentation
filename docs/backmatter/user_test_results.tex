\chapter{User Test Results}
\label{chap:app_user_test_results}
To minimize linguistic misunderstandings (especially with text answers), we
decided to conduct the questionnaire of the user test in the native language of
the participants (German). This appendix, therefore, lists all questions and
answers in the original language.
\section{Questions about Sequences} % (fold)
\label{sec:Questions about Sequences}
\subsection*{Question 1}
\label{sub:ut_q1}
Mir war sofort klar, was die Parameter des Sequenz-Konstruktors bedeuten und wie ich sie definieren muss.
\subsection*{Question 2}
\label{sub:ut_q2}
Ich brauche Funktionen wie map, filter, take, reduce häufig wenn ich mit Listen, Arrays oder Ähnlichem arbeite.
% section Questions about Sequences (end)

\subsection*{Question 3}
\label{sub:ut_q3}
Die Codebeispiele in der JSDoc von take, uncons usw. waren hilfreich. 
\subsection*{Question 4}
\label{sub:ut_q4}
Ich kann mir vorstellen, in einem zukünftigen Projekt das Sequenzframework zu verwenden.
\subsection*{Question 5}
Diese Funktionen brauche ich sonst noch oft im Zusammenhang mit Listen. \\
This question allows us to discover missing operations in the Sequence Library.
\label{sub:ut_q5}
\subsection*{Question 6}
\label{sub:ut_q6}
Ich finde es sinnvoll, das Sequenzframework über einen Namen (im Test `\_`) zu importieren, um bessere IDE Unterstützung zu erhalten.
\subsection*{Question 7}
Was gefiel Ihnen im Zusammenhang mit Sequenzen, was weniger?
\label{sub:ut_q7}
% section Questions about Sequences (end)
\section{Questions about JINQ} % (fold)
\label{sec:Questions about JINQ}

\subsection*{Question 8}
\label{sub:ut_q8}
Es war einfach JINQ zu verwenden.
\subsection*{Question 9}
Ich habe bereits oft mit ähnlichen Abstraktionen wie JINQ gearbeitet. (zB.
LINQ).\\
This question allows us to determine whether participants use these concepts in
other languages often.
\label{sub:ut_q9}
\subsection*{Question 10}
\label{sub:ut_q10}
Ich sehe Sinn darin, mittels JINQ JSON-Strukturen zu durchsuchen. 
\subsection*{Question 11}
\label{sub:ut_q11}
Ich finde die Operationen von JINQ sind sinnvoll benannt.
\subsection*{Question 12}
\label{sub:ut_q12}
JINQ könnte in einem zukünftigen Projekt ein Gewinn sein.
\subsection*{Question 13}
\label{sub:ut_q13}
Was gefiel Ihnen im Zusammenhang mit JINQ, was weniger?
% section Questions about JINQ (end)

\section{General Questions} % (fold)
\label{sec:General Questions}

\subsection*{Question 14}
\label{sub:ut_q14}
Ich habe Erfahrung in der funktionalen Programmierung.
\subsection*{Question 15}
\label{sub:ut_q15}
Für diesen Test nutzte ich folgende IDE.
\subsection*{Question 16}
Meine IDE hat mich während dem Usertest sinnvoll unterstützt. (Code Completion,
Inline Docs, Code Navigation,  Fehler/Warnungen) \\
This question allows us to find out how well IDEs support JSDoc.
\label{sub:ut_q16}
\subsection*{Question 17}
\label{sub:ut_q17}
Das möchte ich sonst noch sagen:
% section General Questions (end)
