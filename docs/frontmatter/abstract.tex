\chapter*{Abstract}
\thispagestyle{empty}
JavaScript supports just a few array processing operations.
Additionally, these operations are inconsistent: some modify the existing
array, but others create a new one. The programming language Haskell provides a
much more extensive and concise collection of such operations for
lists. Additionally, its powerful type system can express
constructs (called type classes) that enable the definition of abstract
operations usable with many different types.\\ 
This project, therefore, investigated how to port various functionalities of
the Haskell standard library to JavaScript. Furthermore, the focus was to
figure out how to model type classes analogous to Haskell using "JSDoc", the
optional type system of JavaScript. The solutions and
findings of this project will become part of the existing zero-dependency Web
UI Toolkit "Kolibri". \\
Since Haskell is a functional programming language, it uses many concepts that
are unfamiliar with JavaScript. However, JavaScript supports some of them,
which have been used where possible. Using them allows the writing
of robust, well-testable, and reusable code. Unit and user tests ensured the
correctness and usability of the artefacts. \\
JavaScript defines iteration protocols to process data structures element by
element. This project exploited these protocols to define a new data structure
called "sequence" and operations to transform and process it. These artefacts
form the "Sequence library". Operations of the Sequence library are compatible
with any data structure that complies with the iteration protocols.
All functions evaluate their input lazily and do not mutate it.\\
Using JSDoc, the type class monad could be defined. However, limitations lie in
the missing name overloading in JavaScript, the reduced type safety, and the
lack of higher-kinded types in JSDoc. \\
Based on these findings, JINQ emerged, capable of uniformly querying any data
structures implementing the type class monad.\\
In his paper "Why Functional Programming Matters", John Hughes shows how
functional programming enables developers to write reusable and modular
code. He explains this with several example programs.
The results of this project allow the implementation of the programs showcased
in Hughes' paper. Additionally, JINQ demonstrates the power of monads in
JavaScript. Thus, this project reveals that JavaScript can effectively
implement and utilize many concepts known from functional programming.
