\section{Comparable Works} % (fold)
\label{sec:Comparable Works}

\subsection{Lodash} % (fold)
\label{sub:Lodash}
Lodash~\cite{lodash_2023} is a modern JavaScript utility library that brings many solutions
similar to this project. Nevertheless, there are some differences:
\begin{itemize}
  \item Functions in Lodash do not take their parameters curried.
  \item Functions in Lodash do not have the receiver at the end.
\end{itemize}
Section~\ref{subsub:Placing the Receiver at the End} describes how these two
ideas (and some more) are essential when writing reusable code. However, Lodash
also provides a functional programming first module that supports the
above.\footnote{https://github.com/lodash/lodash/wiki/FP-Guide}
Nevertheless, the documentation for this is much less extensive than that of
the regular Lodash module. \\ 
The Sequence library supports the monadic
operations described in Section~\ref{subsub:Which operations fit JavaScript?}.
Thus, the Sequence library allows generic functionalities based on the monadic
API, such as JINQ, to be used.
% subsection Lodash (end)

\subsection{rxjs} % (fold)
\label{sub:rxjs}
rxjs~\cite{rxjs_2023} offers very similar functions to the Sequence Library.
However, these functions work with an asynchronous data structure called
|Observable|. Nevertheless, rxjs served several times as a good template for
design decisions. The idea to combine multiple operations using |pipe| comes
from rxjs.
% subsection rxjs (end) section Comparable Works (end)

\subsection{Lambda Calculus for Practical JavaScript} % (fold)
\label{sub:Lambda Calculus for Practical JavaScript}
"Lambda Calculus for Practical JavaScript"~\cite{andermatt_lambda_2022}
provides an introduction and some applications of lambda calculus in
JavaScript. Things like Pair, the commonly used function id, or even the
implementation of the Immutable data structure "Stack" come from this project
work. The constructor "StackSequence" provides by the Sequence library offers
the possibility to make a stack iterable. 
% subsection Lambda Calculus for Practical JavaScript (end)
