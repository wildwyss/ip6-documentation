\section{Comparable Works} % (fold)
\label{sec:Comparable Works}
Some projects and approaches already have similar goals as this thesis. This
section explains them and shows how the standard library differs. All the
projects listed influenced the standard library in a particular way. However,
the functional standard library fundamentally differs from the projects listed
below since it was built explicitly for the Kolibri and supports the "Pair" and
"Stack" types. It does not introduce new dependencies to other projects or
dependency management systems.

\subsection{Lodash} % (fold)
\label{sub:Lodash}
Lodash~\cite{lodash_2023} is a modern JavaScript utility library that brings many solutions
similar to the functional standard library. Nevertheless, there are some differences:
\begin{itemize}
  \item Functions in Lodash do not take their parameters curried.
  \item Functions in Lodash expect the receiver as first parameter.
\end{itemize}
Section~\ref{subsub:Placing the Receiver at the End} describes how these two
ideas (and some more) are essential when writing reusable code. However, Lodash
also provides a functional programming first module that supports the
above.\footnote{https://github.com/lodash/lodash/wiki/FP-Guide}
Nevertheless, the documentation for this is much less extensive than that of
the regular Lodash module. Both library versions provide much less precise
typing than this standard library.\\
The Sequence library additionally supports monadic operations described in
section~\ref{sub:Monads in JavaScript}. Thus, the Sequence
library allows generic functionalities based on the monadic API, such as JINQ,
to be used.
% subsection Lodash (end)

\subsection{rxjs} % (fold)
\label{sub:rxjs}
rxjs~\cite{rxjs_2023} offers very similar functionality as the Sequence library.
However, these functions work with a reactive data structure called
|Observable|. Nevertheless, rxjs served several times as a good template for
design decisions. The idea to combine multiple operators using |pipe| comes
from rxjs.
% subsection rxjs (end) section Comparable Works (end)

\subsection{Lambda Calculus for Practical JavaScript} % (fold)
\label{sub:Lambda Calculus for Practical JavaScript}
"Lambda Calculus for Practical JavaScript"~\cite{andermatt_lambda_2022}
provides an introduction and some applications of lambda calculus in
JavaScript. Things like |Pair|, the commonly used function |id|, or even the
implementation of the Immutable data structure |Stack| come from this project
work. The constructor |StackSequence| provided by the Sequence library offers
the possibility to make a |Stack| iterable. 
% subsection Lambda Calculus for Practical JavaScript (end)
