\section{Description of the Application Domain}
\label{sec:Description of the Application Domain}
This section gives a rough overview of the Kolibri Toolkit and the environment
of this work. Thereby we see which role the Functional Standard Library plays.
Furthermore, the end of the chapter discusses to whom such a Toolkit could be of interest.

\subsection{Functional Programming in JavaScript}
\label{sub:Functional Programming in JavaScript}
JavaScript is omnipresent on the web client side. In many aspects, JavaScript
differs from other programming languages. Sometimes it has the reputation of
being a bit strange. Nevertheless, the language offers some excellent concepts
which enable functional programming. For example, in JavaScript, functions are
ordinary objects. \\
As Listing~\ref{lst:curried_fn_in_js} shows, defining functions receiving their
parameters in a curried style is also possible. This means, passing one
argument to a function taking two parameters returns a new function just taking
one parameter. \\ 
This paper explores these powerful concepts in depth. Among other things,
writing more robust code in JavaScript is possible with functional concepts.
And that’s where the Kolibri Toolkit comes in. You’ll learn more about the
Kolibri Toolkit in the next section.

\begin{lstlisting}[
  style=ES6,
  caption=Curried functions in JavaScript,
  label={lst:curried_fn_in_js}
]
const plus = a => b => a + b; // A function taking two params
const plusOne = plus(1);      // plus(1) returns a new function taking 1 param

console.log(plusOne(5));      // applying another param calculates the result
// => Logs '6'
\end{lstlisting}


\subsection{The Kolibri Web UI Toolkit}
\label{sub:The Kolibri Web UI Toolkit}
The web page of the Kolibri describes the toolkit as, "Kolibri wants to be a
sustainable, high quality Toolkit"~\cite{kolibri}, - and it is.
The Kolibri Web UI Toolkit is a collection of implementations. Experts test all
components, which must satisfy high functional and non-functional standards. An
essential part of the toolkit is the dependency-less implementation. Thus, it
represents a certain antithesis to the frameworks dominated by |npm| libraries.
This is important to fulfill the quality requirements in the long term.\\
The toolkit is under continuous development. Mainly from student work, each
focuses on one topic. Implementing a standard library based on functional
concepts, is the focus of this work. \\ 
The implementations provide examples in each case, making adaptations and use
possible without an intensive training period.

\subsection{The Root of the Project}
\label{sub:The Root of the Project}
This work continues the previous project from the fifth semester of Computer
Science studies. Thus, we have already dealt with the topic before this thesis,
but on a smaller scale. This predecessor project includes a Logging Framework
for the Kolibri toolkit and the first version of the functional standard
library. These results made it possible to develop the library to its current
state further. By researching the literature and trying different approaches, a
versatile and robust implementation of a functional standard library emerged.

\subsection{Role of the Standard Library}
\label{sub:Role of the Standard Library}
The Kolibri toolkit ships the standard library. The current state of the
library already covers many needs. Section~\ref{sec:Examples} shows that it is
possible to create stunning projects based on it. But such a library has
enormous expansion capabilities. Chapter~\ref{sec:Future Features} describes
the possibilities of extending the library.

\subsection{Who is the Toolkit for?}
\label{sub:Who is the Toolkit for?}
The toolkit is for anyone who wants to implement sustainable web applications 
and does not want to be exposed to the hell of dependencies and dangers
of vulnerabilities in various packages. It is an open-source project that
anyone can use. The basic idea is that you copy the toolkit into your project.
Then you can use the things you need. Of course, everyone is free to adapt
things for their purposes or to add new implementations.
