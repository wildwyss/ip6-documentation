\section{Description of the Application Domain}
\label{sec:Description of the Application Domain}
\subsection{Functional Programming in JavaScript}
\label{sub:Functional Programming in JavaScript}
JavaScript is omnipresent on the web client side. In many aspects, JavaScript
differs from other programming languages, and sometimes it has the reputation of
being a bit strange. Nevertheless, the language offers some excellent concepts
which enable functional programming. 
Therefore, the language provides the following features: 

\begin{itemize}
  \item{Arrow functions: Assignment of functions to variables} 
  \item{Higher-order functions: Functions that can take other functions as arguments or return functions as results}
  \item{First class functions: Functions are objects that can be treated as any other object}
  \item{Lazy evaluation: calculations are done when the result is needed}
  \item{Curried functions: passing one argument to a function taking two parameters returns a new function just taking one parameter}
\end{itemize}

Listing~\ref{lst:curried_fn_in_js} shows these concepts with some examples.
First, on line~\ref{line:start_plus}, the arrow function |plus| defines a function that expects two arguments in curried style.
Then, the function is partially evaluated by passing one argument to it. 
The result is a new function |plusOne|. This process is called partial application.
|plusOne| is now a function that expects one more argument
to evaluate the result. Passing the next argument calculates the addition and
stores the result in the variable |result| on line~\ref{line:end_plus}.
\\


\begin{lstlisting}[
  style=ES6,
  caption=Functional concepts in JavaScript,
  label={lst:curried_fn_in_js}
]
const plus    = a => b => a + b; // an arrow function taking two params *'\label{line:start_plus}'*
const plusOne = plus(1);         // plus(1) returns a new function 
const result  = plusOne(5);      // applying another param calculates the result *'\label{line:end_plus}'*

const commentResult = func => {  // a function as argument*'\label{line:start_commentResult}'*

  // defining a new function which takes one argument
  const returnFunction = value => {
    const comment = "The result is: ";
    const result  = func(value);
    return comment + result;
  };

  // returning a function from a function
  return returnFunction;
};*'\label{line:end_commentResult}'*

const commentResultFn = commentResult(plusOne); // yields a new function
const result          = commentResultFn(6);     // evaluating the result *'\label{line:comment_result_fn}'*   
console.log(result);
// => Logs 'The result is: 6'
\end{lstlisting}
Next, let us consider the |commentResult| function on
line~\ref{line:start_commentResult} to~\ref{line:end_commentResult}, 
which takes another function
as input and generates a new function. To achieve this, the original function,
referred to as |func| is enclosed within a wrapper function to produce a
commented result. The outcome of invoking |commentResult| with the |plusOne|
function is stored in |commentResultFn,| which itself becomes a function. The
actual computation and output of results occur when |commentResultFn| is
called on line~\ref{line:comment_result_fn}. This example showcases how 
functions can be treated as regular objects.

\subsection{The Kolibri Web UI Toolkit}
\label{sub:The Kolibri Web UI Toolkit}
The web page of the Kolibri describes the toolkit as, "Kolibri wants to be a
sustainable, high-quality toolkit"~\cite{kolibri}, - and it is.
The Kolibri Web UI Toolkit is a collection of JavaScript abstractions. Experts test all
components, which must satisfy high functional and non-functional standards. An
essential part of the toolkit is the dependency-less implementation. Thus, it
represents a certain antithesis to the frameworks dominated by |npm| libraries.
This is important to fulfil the quality requirements in the long term.\\
The toolkit is under continuous development, with each student project focusing
on a new topic. The main goal of this thesis is to extend the toolkit
with a standard library based on functional concepts. 

\subsection{The Basis of the Project}
\label{sub:The Basis of the Project}
This work continues the previous project from the fifth semester of Computer
Science studies. Therefore, we have already dealt with the topic before this thesis,
but on a smaller scale. This predecessor project's artefacts include a Logging Framework
for the Kolibri toolkit and the first version of the functional standard
library. These results made it possible to develop the library to its current
state. By researching the literature and trying different approaches, a
versatile and robust implementation of a functional standard library emerged.

\subsection{Who is the Toolkit for?}
\label{sub:Who is the Toolkit for?}
The toolkit is intended for people who want to develop robust web applications
without the burdensome dependencies and vulnerabilities associated with
multiple packages. As an open-source project, it is available for anyone to
utilize. The fundamental concept entails copying the toolkit into your project,
allowing you to utilize the necessary components. Naturally, users have the
freedom to modify elements to suit their specific needs or contribute new
implementations as desired.
