\section{Description of the Application Domain}
\label{Description of the Application Domain}
This chapter gives a rough overview of the Kolibri Toolkit and the environment
of this work. Thereby we see which role the Functional Standard Library plays.
Furthermore, we discuss for whom such a Toolkit could be of interest.

\subsection{Functional Programming in JavaScript}
\label{sub:Functional Programming in JavaScript}
You come into contact with JavaScript when you want to program something on the
web client side. In many aspects, JavaScript differs from other programming
languages. Sometimes it has the reputation of being a bit strange.
Nevertheless, the language offers some excellent language concepts of
functional programming. For example, in JavaScript, functions are ordinary
objects. Also, it is possible to define the parameters of functions in a
curried style. Such concepts are very powerful, and we will explore this in
more depth in this paper. Among other things, with functional concepts, writing
more robust code in JavaScript is possible. And that's where the Kolibri
Toolkit comes in. You'll learn more about the Kolibri Toolkit in the next chapter.

\subsection{The Kolibri Web Ui Toolkit}
\label{The Kolibri Web Ui Toolkit}
On the main page of Kolibri it says, "Kolibri wants to be a sustainable, high
quality Toolkit"~\cite{kolibri}, - and it is.
The Kolibri Web Ui Toolkit is a collection of implementations. Experts test all
components and must meet high functional and non-functional standards. An
essential part of the Toolkit the dependencyless implementation. Thus, it
represents a certain antithesis to the frameworks dominated by npm Libraries.
This is important to fullfill the quality requirements in the long run.

The Toolkit is under continuous development. Mainly from student work, each
focuses on one topic area. Functional Standard Library is the focus of this
work. The implementations provide examples in each case, making adaptation and
use possible without a long training period.

\subsection{The Root of the Project}
\label{The Root of the Project}
This work is a continuation of our previous project from the fifth semester of
computer science studies. Thus, we have already dealt with the topic before
this thesis, but on a smaller scale. We implemented a Logging Framework for the
Kolibri Toolkit and a first version of the Functional Standard Library. Based
on these results we were able to further develop the Library to its current
state. By researching the literature and trying out different approaches, we
have now managed to create a versatile and robust implementation of a
Functional standard Library.

\subsection{Role of the Standard Library}
\label{sub:Role of the Standard Library}
The Functional Standard Library is a part of the Toolkit. With
this realization, future implementations may base on it. However, the Toolkit
will also find stand-alone use. But, the Library can still extend.
Chapter~\ref{sec:Future Features} describes the possibilities of extending the Library.

\subsection{Who is the Toolkit for?}
\label{sub:Who is the Toolkit for?}
The Toolkit is for anyone who wants to implement sustainable applications on
the Web and does not want to be exposed to the hell of dependencies and dangers
of packet managers. It is an open-source project that
anyone can use. The basic idea is that you copy the Toolkit into your project.
Then you can use the things you need. Of course, everyone is free to adapt
things for his purposes or to add new implementations.

