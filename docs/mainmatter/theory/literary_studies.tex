\section{Literary and Studies}
\label{sec:Literary and Studies}
This section is about formative sources for this thesis. The following briefly
introduces the sources and explains their meaning for this work.

\subsection{Why Functional Programming Matters}
\label{sub:Functional Programming Matters}

Why Functional Programming Matters~\cite{hughes_why_1989} by John Hughes gives
an impressive insight into functional programming. Mainly higher-order
functions, and laziness is discussed intensively. This paper strongly
influenced the development of the standard library, as it describes many
concepts and best practices. In addition, the user test is based on an example
of this paper. \\ 
Furthermore, this paper has improved our understanding of function composition,
partial application, and laziness. Therefore, it has enhanced the quality and
robustness of the standard library. \\
The Sequence library, part of the functional standard library, implements ideas
functionalities described in this paper. It is a prominent part of this thesis,
allowing to write good, modularized, and reusable code.
Chapter~\ref{chap:Modularizing Programs} deals with these ideas in depth.


\subsection{Frege Goodness}
\label{sub:Frege Goodness}
Frege Goodness~\cite{frege_goodness} by Prof. Dierk König is a book about
Frege, a functional programming language for the JVM. This book contains
examples and explanations of various applications. We gained new insights and
inspiration for implementations and examples based on these. We treated
individual chapters more intensively, such as \textit{The Power of the Dot}.
The insights gained helped to make far-reaching decisions for the Sequence
library. Chapter~\ref{chap:The Power of the Dot} describes more details about
that.

\subsection{Quickcheck}
\label{sub:Quickcheck}
QuickCheck: A Lightweight Tool for Random Testing of Haskell
Programs~\cite{quickcheck_hughes} by Koen Claessen and John Hughes describes
how property-based testing works in Haskell. The adaption of specific
approaches described by this paper improved the code quality of the Sequence
library. Since this project did not focus on this topic, we only took over a few
ideas. However, further implementations in this direction are possible.
Chapter~\ref{sec:Testing based on Properties} discusses the adapted
concepts in more detail. 

\subsection{MDN - Iteration Protocols}
\label{sub:MDN - Iteration Protocols}
The Mozilla Firefox page describing the iteration
protocols~\cite{mdn_protocols} is the basis for developing the Sequence
library. The page contains all relevant facts about Iterable and Iterators.
Section~\ref{sec:Iterables in General} explains the implementation
of protocols.

\subsection{LINQ}
\label{sub:LINQ}
LINQ (Language integrated queries)~\cite{billwagner_language-integrated_2023}
is a widely used concept for querying data. It enables to process different
data sources in a fluent and declarative way. This declarative nature makes it
straightforward to write queries and understand them later. JINQ, which emerged
in this project, is based on the same ideas. LINQ heavily inspired the naming
and the scope of JINQ, enabling users familiar with LINQ to use JINQ
immediately. Chapter~\ref{sec:Query different Data Structures} discusses JINQ
in detail.
