\section{Literary and Studies}
\label{sec:Literary and Studies}
This chapter is about formative sources for this thesis. In the following, the
sources are briefly introduced and explained as to their use for this work.

\subsection{Why Functional Programming Matters}
\label{sub:Functional Programming Matters}
Why Functional Programming Matters~\cite{hughes_why_1989} by John Hughes gives an 
impressive insight into the world of functional
programming. Mainly higher-order functions and laziness are discussed
intensively. In development, we could reproduce specific applications from this
paper in analog form using the Sequence Library. This gave us confidence that
we were on the right track because the concepts discussed in this paper have
been proven for a long time. In addition, we implemented a user test based on
examples from this paper to obtain feedback on the Sequence Library.
\newline
Furthermore, this paper has improved our understanding of function composition
and partial application. Therefore, this paper has enhanced the quality and
robustness of the Sequence Library. Chapter~\ref{sec:Modularizing Programs} 
deals with this topic in depth.

\subsection{Frege Goodness}
\label{sub:Frege Goodness}
Frege Goodness~\cite{frege_goodness} by Prof. Dierk König is a book about Frege, a functional
programming language for the JVM. This book contains examples and
explanations of various applications. Based on these, we gained new insights and inspiration for
implementation examples. We treated individual chapters more intensively, such
as \textit{The Power of the Dot}. The insights gained helped us to make far-reaching
decisions for the Sequence Library. You find more details about that in
Chapter~\ref{sec:The Power of the Dot}.

\subsection{Quickcheck}
\label{sub:Quickcheck}
QuickCheck: A Lightweight Tool for Random Testing
of Haskell Programs~\cite{quickcheck_hughes} by Koen Claessen and John Hughes describes how
property-based testing is applied in Haskell. We were able to adapt specific
exciting approaches from this paper. This has additionally enabled us to
improve our testing base. Even though we could only use minor implementations
based on this paper. However, further implementations in this direction are
possible. Chapter~\ref{sub:A Kind of Property based Testing} discusses the
adapted concepts in more detail. 

\subsection{MDN - Iteration Protocols}
\label{sub:MDN - Iteration Protocols}
The Mozilla Firefox page of iteration protocols~\cite{mdn_protocols} is the basis for developing the
Sequence Library. The page contains all relevant facts about Iterable and
Iterators. Chapter~\ref{sec:Sequence and Iterable in General} explains the
implementation of protocols.

\subsection{LINQ}
\label{sub:LINQ}
LINQ (Language integrated queries)~\cite{billwagner_language-integrated_2023} 
is a widely used concept for querying data. It enables to process of any data
in a fluent and declarative way. Using this to solve problems is convenient and
easy to read and understand afterward.
We have built our implementation JINQ analogously. Therefore,
we have kept the functionalities and naming of existing implementations as much 
as possible to help users become familiar with 
Chapter~\ref{sec:Query different Data Structures} discusses JINQ in detail.
