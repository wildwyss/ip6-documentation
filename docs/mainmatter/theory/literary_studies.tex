\section{Literary and Studies}
\label{sec:Literary and Studies}
This chapter is about formative sources for this thesis. In the following, the
sources are briefly introduced and explained as to their use for this work.

\subsection{Why Functional Programming Matters}
\label{sub:Functional Programming Matters}
Why Functional Programming Matters~\cite{hughes_why_1989} by John Hughes gives an 
impressive insight into the world of functional
programming. Mainly "Higher Order Functions" and laziness are discussed
intensively. Many of the concepts from this paper have given us confidence that
we are on the right track. For example, we rebuilt the application examples
from the article with our Sequence Library.
Furthermore, this paper has improved our understanding of function composition
and partial application. We believe this paper has improved the quality and
robustness of our library.

\subsection{Frege Goodness}
\label{sub:Frege Goodness}
Frege Goodness~\cite{frege_goodness} by Prof. Dierk König is a book about Frege, a functional
programming language for the JVM. This book contains clear examples and
explanations of various applications. Based on these, we gained new insights and inspiration for
implementation examples. We treated individual chapters more intensively, such
as The Power of the Dot. The insights gained helped us to make far-reaching
decisions for the Sequence Library.

\subsection{Quickcheck}
\label{sub:Quickcheck}
QuickCheck: A Lightweight Tool for Random Testing
of Haskell Programs~\cite{quickcheck_hughes} by Koen Claessen and Jon Hughes describes how
property-based testing is applied in Haskell. We were able to adapt specific
exciting approaches from this paper. This has additionally enabled us to
improve our testing base. Even though we could only use minor implementations
based on this paper, it showed that one could develop further in this
direction.

\subsection{MDN - Iteration Protocols}
\label{sub:MDN - Iteration Protocols}
The Mozilla Firefox page of iteration protocols~\cite{mdn_protocols} is the basis for developing the
Sequece Library. The page contains all relevant facts about Iterables and
Iterators.

\subsection{LINQ}
\label{sub:LINQ}
LINQ (Lnaguage integrated queries)~\cite{LINQ} is a widely used concept for querying data.
We have built our implementation JINQ analogously. The source of Microsoft has
a formative part, which functionalities we provide. Users should use familiar
concepts in this case. Therefore, it makes sense to stick to existing
functionalities and terminologies.
