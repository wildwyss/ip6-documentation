\section{Future Features}
\label{sec:Future Features}
During the project, some ideas emerged that were not feasible for the time
being for various reasons. This section describes these possible extensions for
the standard library.

\subsection{Logging}
\label{sub:Logging}
Logging is a topic that has come up several times. On the one hand, it appears
during the implementation of the Sequence library, but also through feedback
from a test proband of the user test. In the predecessor project, we implemented
a logging framework~\cite{wild_ip5_2023}. This would also be useful to include
in the Sequence library. Especially in case of errors, analyzing debug or
tracing messages from the involved functions would be helpful. Currently, the
logging framework is only included in a part of the test framework. But this
could be extended in a further step.

\subsection{Operators and Operations}
\label{sub:Operators and Operations}
Following are some possible extensions for the Sequence library:

\subsubsection{uncons for empty Sequences}
\label{subsub:uncons}
|uncons| is an operation that returns the head and the rest of an iterable in a
pair. The Sequence library already includes this operation.
But it does not work on empty sequences. Meaning |uncons| applied to an
empty iterable fails. Thus, one would have a version, which for example, wraps the
result in a maybe. |uncons| on an empty list would then return |Nothing|.
The API documentation page~\cite{hoogle_uncons} describes the function in
detail.

\subsubsection{tail}
\label{subsub:tail}
|tail|~\cite{hoogle_tail} is a useful function that removes the first element
of an iterable and returns the rest of it. By implementing |tail|, one has to
pay attention to empty or single-valued iterables, which do not have a tail. An
option could be to return a maybe. |Nothing|, if the list has one or zero
elements, and |Just| including the result otherwise.

\subsubsection{unfoldr}
\label{subsub:unfoldr}
|unfoldr| is a powerful concept for creating lists in Haskell. It would
supplement the options to create new sequences besides the constructor |Sequence|. 
In Haskell, it works using a |Maybe|. |unfoldr| creates a list that returns a
pair wrapped in |Just| with the value of the current iteration and the next
element. If the stop condition is fulfilled, |unfoldr| returns |Nothing|. For
further information, look at the documentation on ~\cite{hoogle_unfoldr}.

\subsubsection{iterate}
\label{subsub:iterate}
|iterate| takes two arguments. The first argument is a function, and the second
is a start value. It generates an infinite iterable of repeated applications of
the function to the calculated value. You will find more information about
|iterate| on hoogle~\cite{hoogle_iterate}.

\subsection{JINQ Functions}
\label{sub:JINQ Functions}
It is possible to get errors when working with |JINQ| and |JSONMonad| by grabbing
not present properties. That means if a function in |select| accesses a property
which is not defined, it will throw an error. This kind of null-case handling
takes a lot of work. To remedy this situation, having error-safe functions like
|safeSelect| to access probably not existing values would be excellent.

\subsection{Applications}
\label{sub:Applications}
\subsubsection{Fix Point Sequence}
\label{subsub:Fixpoint Sequence}
In the mathematical context, having a tool for approximations is helpful. A
constructor for such a case could be the fix point sequence. It allows us to
find an approximation with a given number of iterations or by providing a value
of minimal deviation. The function |within| described in 
section~\ref{sub:Numerical Differentiation} already serves as a good starting
point.

\subsubsection{Use JINQ to process HTML}
\label{subsub:Use JINQ to process HTML}
Scalpel~\cite{scalpel} is a Haskell library to scrap web content. Building a
similar application based on the present implementations could be possible.
With |JINQ| and |JsonMonad|, the standard library already contains a scraping
tool for JSON Objects. A future implementation could use a similar approach to
develop such an application.
