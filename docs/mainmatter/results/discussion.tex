\section{Conclusion}
\label{sec:conclusion}
This Chapter discusses the results and findings of the thesis. The beginning reflects the
standard library we created in this thesis and its achievements. The following
Sections deal with more topic specific aspects. Therefore, we make a
recap of particular decisions during development. Finally, non-functional
aspects are in the foreground, where we see what other values were important to
us while implementing the project.

\subsection{Findings and Achievements}
\label{sub:Findings and Achievements}
The objective of this thesis was to develop a functional standard library for
the Kolibri Web Ui Toolkit in JavaScript. To achieve this, we delved into the
iterable protocols of JavaScript, leveraging our understanding to construct the
concept of Sequences. Sequences are characterized by their lazy evaluation and
immutability, offering a powerful tool for handling any iterable object. By
employing the decorator approach, we built various functionalities capable of
processing regular JavaScript arrays and other iterables. This was made
possible by passing the receiver to the function, which also allows us to take
advantage of eta reduction.
\newline
An excellent example of this is the pipe function, enabling the execution of
multiple operations in a sequential manner, similar to what can be achieved
with Java Streams. With the Sequence Library's compatibility with all iterable
objects, it became logical to make other existing collections iterable as well.
As a result, we extended the Kolibri Toolkit's existing collections such as
Pair, Stack, and Tuple to support iterability. Additionally, we implemented
specialized Sequences that facilitate the generation of application-specific
data, such as the |AngleSequence|, which produces angles at regular intervals,
or a Sequence of prime numbers.
\newline
To ensure clarity and maintain consistency, all implementations are typed using
jsDoc. While overall the process went smoothly, we encountered limitations when
combining multiple types, as outlined in the thesis. Despite these challenges,
we made significant progress in developing a functional standard library for
the Kolibri Web Ui Toolkit, offering enhanced functionality and versatility to
JavaScript developers.

\subsection{A Closer Look to Particular Findings}
\label{sub:A Closer Look to Particular Findings}

\subsubsection{Similarity to Haskell}
\label{subsub:Similarity to Haskell}
As Haskell is a well-established and widely used functional language, we
frequently draw inspiration from its problem-solving approaches when making
decisions. In each case, this has proven to be a valuable choice. However, it
is essential to note that certain remarkable language concepts from Haskell
could not be directly adopted in JavaScript. Nevertheless, specific
implementations are pretty similar, as the following Listings~\ref{lst:comparing_with_javascript} 
and~\ref{lst:comparing_with_haskell} demonstrates.

\begin{lstlisting}[
  style=Haskell, 
  caption=Haskell vs. Sequence library - Haskell implementation, 
  label={lst:comparing_with_javascript}
  ]
-- Creating a list from 0 to 4
let list = unfoldr (\x -> if x < 5 then Just(x, x + 1) else Nothing) 0

-- mapping the list
map (\x -> x * 2) list 
\end{lstlisting}

\begin{lstlisting}[
  style=ES6, 
  caption=Haskell vs. Sequence library - JavaScript implementation,
  label={lst:comparing_with_haskell}
  ]
// Creating a list from 0 to 4
const list = Sequence(0, x => x < 5, x => x + 1);

// mapping the list
map(x => x * 2)(list)
\end{lstlisting}


\subsubsection{Robust Programming in JavaScript}
\label{subsub:Robust Programming in JavaScript}
Strange situations arise in JavaScript more often than in other programming
languages. One contributing factor is the nature of its type system. However,
within this thesis, we have demonstrated that it is still possible to program
reasonably robustly by leveraging functional programming concepts. Crucially,
this requires taking advantage of the functional aspects inherent to the
language, such as higher-order functions and partial application. Nonetheless,
we encountered certain limitations along the way. In particular, not all
desired behaviors can be achieved when dealing with the type system.
An illustrative example is the combination of a |MonadType| with an |Iterable|
type. When a function returns a  |MonadType|, the valuable information that could
have been preserved as Iterable is lost. Resolving such a situation requires
type-casting.

\subsubsection{Monadic Structures}
\label{subsub:Monadic Structures}
With |JINQ|~\ref{sec:Query different Data Structures}, we have realized an
concept based on monadic types. Thus, we could show that it is also
possible and valuable to implement such concepts in JavasScript.
This opens numerous possibilities for extensions that act on monadic types in
JavaScript.

\subsubsection{Testing}
\label{subsub:Testing}
A key ingredient for achieving a robust and sustainable library lies in having
a reliable test framework. In the case of the Sequence library, stability was
attained through the utilization of the TestingTable and configured Test
Objects~\ref{sec:Testing}, which provided a solid foundation for incremental growth.
By gradually incorporating new functionalities and corresponding test cases,
the core of the library became increasingly robust. Moreover, novel test
concepts were introduced during the development process to enhance the testing
capabilities further. A notable example is the introduction of invariant tests~\ref{subsub:Invariant Testing},
which systematically assess the behaviors of functionalities through
diverse testing approaches, ensuring comprehensive validation.

\subsection{Non-Functional Findings}
\label{sub:Non-Functional Aspects}

\subsubsection{Library Organization}
\label{subsub:Library Organisation}
During the development of the Sequence library, we consistently paid close
attention to non-functional aspects. As part of this effort, we adjusted the
organization of the project's development setup. This led us to the current
state, where each functionality is located in its own separate file. The
adoption of such a project structure has had a significant impact on the
overall clarity and navigability of the codebase. Moreover, it contributes to a
sense of order, which, in turn, enhances the overall code quality.
Additionally, it helps to prevent cycling dependencies when importing
particular functionalities.

\subsubsection{Code Quality}
\label{subsub:Code Quality}
By maintaining high code quality standards, we created a clear and consistent
code base. We paid attention to the following points:

\begin{itemize}
  \item{No code duplications}
  \item{Good naming}
  \item{Standardized formatting}
  \item{Only necessary exports of functions}
  \item{Appropriate comments}
  \item{Project organization and structure}
\end{itemize}

Strict adherence to these principles makes development easier for us and helps
library users find their way around more quickly. Additionally, it facilitates
future developers to learn how to add new functionalities.

\subsection{User Testing}
\label{sub:User Testing}
The user testing conducted in Section~\ref{sub:User Testing} was crucial for further improvement
of the Sequence library. It provided us with valuable feedback from programmers
who had no prior knowledge of the library. This insight allowed us to address
specific issues promptly and immediately improve based on the test results.
Additionally, we added other valuable suggestions to the list of potential
enhancements, as outlined in Chapter~\ref{sec:Future Features}.
