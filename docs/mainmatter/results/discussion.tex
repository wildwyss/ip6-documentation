\section{Discussion}
\label{sec:Discussion}
This chapter discusses the achieved results of the Thesis. Following, we have a look at the
Functional Standard Library and the most important aspects of it. Subsequent sections
discuss the findings we have made in this Thesis. The end describes the
findings from the user test. 

\subsection{Functional Standard Library}
\label{sub:Functional Standard Library}
In this work, the goal was to build a Functional Standard Library for the
Kolibri Web Ui Toolkit~\cite{kolibri}. We achieved
this goal. The Library offers a lot of the necessary functionalities to implement
various applications. As we have seen in the Examples Chapter~\ref{sec:Examples}, one can implement
even complex algorithms such as the alpha-beta. We implemented also the
non-functional aspects of the Functional Standard Library. The library is fully
documented with jsDoc, explained with examples, and typed by jsDoc~\cite{jsdoc_use_2023}. 
This represents the maximum that can
be done in JavaScript to minimize errors in the application.

\subsection{Similarity to Haskell}
\label{sub:Similarity to Haskell}
Because Haskell is a widely used and established functional language, we have
aligned implementations of the Sequence Library with Haskell. For
example the functions are stand alone and the receiver is passed as argument. 
Likewise, we build the Sequence Library lazy, as is also the case in Haskell. 
These concepts are precious and beneficial in each implementation using the Sequence Library.

\subsubsection{Comparison - Haskell vs. Sequence Library}
\label{Comparion - Haskell vs. Sequence Library}
For visualization, listings~\ref{lst:comparing_with_javascript} and~\ref{lst:comparing_with_haskell} 
shows example similarities of Haskell and the Sequence Library:

\begin{lstlisting}[
  style=Haskell, 
  caption=Haskell code for comparison, 
  label={lst:comparing_with_javascript}
  ]
-- Creating a list from 0 to 4
let list = unfoldr (\x -> if x < 5 then Just(x, x + 1) else Nothing) 0

-- mapping the list
map (\x -> x * 2) list 
\end{lstlisting}

\begin{lstlisting}[
  style=ES6, 
  caption=JavaScript code for comparison,
  label={lst:comparing_with_haskell}
  ]
// Creating a list from 0 to 4
const list = Sequence(0, x => x < 5, x => x + 1);

// mapping the list
map(x => x * 2)(list)
\end{lstlisting}

\subsection{Findings}
\label{sub:Findings}
This section highlights the main findings of the Thesis.

\subsubsection{Robust Programming in JavaScript}
\label{subsub:Robust Programming in JavaScript}
Strange situations arise in JavaScript more often than in other programming
languages. One of the reasons for this is the type system. However, in this
thesis, we have shown that a reasonably robust way of programming is possible
using functional programming concepts. The prerequisite for this is the
functional aspects implemented in the language, such as higher-order functions
and partial application. However, it must also be clearly said that we have
reached the limits. Particularly in connection with the type-system,
not all desired behaviors can be achieved.
An example is the combination of a |MonadType|, which is also an |Iterable|
type. If a function returns a |MonadType|, one loses the information that this
could have been |Iterable|. Such a situation requires type-casting.

\subsubsection{Monadic Structures}
\label{subsub:Monadic Structures}
With |JINQ|~\ref{sec:Query different Data Structures}, we have realized an
implementation based on monadic types. Thus, we could show that it is also
possible and valuable to implement such concepts in JavasScript.
This opens numerous possibilities for extensions that act on monadic types in
JavaScript.

\subsubsection{Testing}
\label{subsub:Testing}
A recipe for the success of a robust and sustainable implemented library is a
solid test framework. The Sequence Library has become stable through the
|TestingTable| and configured Test Objects~\ref{sec:Testing}. By incrementally adding
functionality and test cases, the library's core became more and more robust.
In addition, new test concepts have been added during development. One example
is the invariant tests~\ref{subsub:Invariant Testing}. These ensure even more
behaviors of functionalities by testing in various ways.

\subsubsection{Library Organization}
\label{subsub:Library Organisation}
While developing a larger application, we also scrutinized non-functional
aspects repeatedly. We have adjusted the organization of how the project is set up
in development. So we came to the current state, where each
functionality is in a separate file. Such a project structure has an excellent
impact on the overview and navigability of the project. In addition, it helps
to create a good order, which also helps to improve the code quality.

\subsubsection{Code Quality}
\label{subsub:Code Quality}
By maintaining a high code quality standard, we created a clear and consistent
code base. We paid attention to the following points:

\begin{itemize}
  \item{No code duplications}
  \item{Good naming}
  \item{Standardized formatting}
  \item{Only necessary exports of functions}
  \item{Appropriate comments}
  \item{Project organization and structure}
\end{itemize}

Strict adherence to these principles makes development easier for us and helps
library users find their way around more quickly. Additionally, it facilitates
future developers to learn how to add new functionalities.

\subsection{User Testing}
\label{sub:User Testing}
The user testing~\ref{sub:User Testing} helped us to improve the Sequence Library further. It was also
valuable to gain insight into what other programmers without prior knowledge of
the library think. We were able to implement certain things immediately after
the test using the results. Others we added to the list of possible enhancements described in
Chapter~\ref{sec:Future Features}.
