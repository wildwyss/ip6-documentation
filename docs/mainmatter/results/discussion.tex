\section{Conclusion}
\label{sec:conclusion}
\subsection{Findings and Achievements}
\label{sub:Findings and Achievements}
This thesis aimed to develop a functional standard library for the Kolibri Web
Ui Toolkit in JavaScript. To achieve this, we delved into the iterable
protocols of JavaScript, leveraging our understanding to construct a new data
structure: the sequence. The characteristics of sequences are lazy evaluation
and immutability, which offer powerful ways of handling iterable objects. Based
on these findings, the Sequence library emerged. It can process, transform, and
create sequences by decorating existing iterables. This library does not only
work with sequences but also with every iterable object, such as JavaScript
arrays - passing the receiver to the functions enabled this.\\
The power of eta reduction manifests itself in the pipe function, enabling the
execution of multiple operations sequentially, dramatically improving code
readability. With the Sequence library’s compatibility with all iterable
objects, it became logical to make other existing data structures iterable as
well. As a result, existing collections of the Kolibri, such as pair, stack,
and tuple, are now iterable and, therefore, compatible with the Sequence
library!\\
Typing the Sequence library using JSDoc ensures clarity and maintains
consistency.

The work also shows that monads in JavaScript can be very helpful and enable
new ways to reason about problems. For example, JINQ enables the processing of
various data structures that provide a monadic API in a very declarative way.
The JSON monad makes JINQ compatible with arrays of JavaScript objects, solving
daily problems that web developers face!

In addition, many examples show the strengths of the standard library. Its
tools enable easy transfers of Haskell algorithms based on lists
to JavaScript!\\
Despite some challenges, a powerful functional standard library for the Kolibri
Web Ui Toolkit emerged, offering enhanced functionality and versatility to
JavaScript developers!\\
\subsection{A Closer Look to Particular Findings}
\label{sub:A Closer Look to Particular Findings}

\subsubsection{Similarity to Haskell}
\label{subsub:Similarity to Haskell}
As Haskell is a well-established and widely used functional language, it serves
as a great role model for solutions to problems when making decisions. \\
However, it is essential to note that some remarkable language concepts of
Haskell do not exist in JavaScript, making it harder to translate code directly
from one language to the other. \\
Nevertheless, specific implementations are
pretty similar, as the following listings~\ref{lst:comparing_with_javascript}
and~\ref{lst:comparing_with_haskell} demonstrate:

\begin{lstlisting}[
  style=Haskell, 
  caption=Haskell vs. Sequence library - Haskell implementation, 
  label={lst:comparing_with_javascript}
  ]
-- Creating a list from 0 to 4
list = unfoldr (\x -> if x < 5 then Just(x, x + 1) else Nothing) 0

-- mapping the list
map (\x -> x * 2) list 
\end{lstlisting}

\begin{lstlisting}[
  style=ES6, 
  caption=Haskell vs. Sequence library - JavaScript implementation,
  label={lst:comparing_with_haskell}
  ]
// Creating a list from 0 to 4
const list = Sequence(0, x => x < 5, x => x + 1);

// mapping the list
map(x => x * 2)(list)
\end{lstlisting}
\subsubsection{Robust Programming in JavaScript}
\label{subsub:Robust Programming in JavaScript}
Strange situations arise in JavaScript more often than in other programming
languages. One contributing factor is the nature of its type system. However,
this thesis demonstrates that functional programming concepts and distinct
usage of JSDoc allow the development of reasonably robust programs.\\
Crucially, this requires taking advantage of the functional aspects inherent to
the language, such as higher-order functions and partial application.
Nevertheless, certain limitations sometimes pose significant problems on the
solution path. In particular, not all desired behaviours are achievable when
dealing with the type system: JSDoc is very limited when combining multiple
types or writing general functions that need types to abstract over another
type. An illustrative example of this is the combination of a monad with an
iterable: when a function returns a monad, the valuable information
that this monad could also be iterable is lost. Resolving such a situation
requires type-casting.\\ 
The user test shows that modern IDEs deliver excellent support for JSDoc and
help the developers get along the way when working with JavaScript.
\subsubsection{Monadic Structures}
\label{subsub:Monadic Structures}
|JINQ| explained in section ~\ref{sec:Query different Data Structures}, is a
concept based on monadic types. This shows that implementing such concepts in
JavaScript is also possible and valuable, which opens numerous possibilities
for extensions that act on monadic types in JavaScript.

\subsubsection{Testing}
\label{subsub:Testing}
A reliable test framework is crucial for achieving a robust and sustainable
library. In the case of the Sequence library, the testing table attained the
stability of the Sequence library. This lies a solid foundation for incremental
growth. The library's core became increasingly robust by gradually
incorporating new functionalities and corresponding test cases. Moreover, novel
test concepts were introduced during the development process to enhance the
testing capabilities further. A notable example is the introduction of
invariant tests (section~\ref{sub:Invariant Testing}), which systematically
assess the behaviours of functionalities through diverse testing approaches,
ensuring comprehensive validation.

\subsection{Non-Functional Findings}
\label{sub:Non-Functional Aspects}
\subsubsection{Library Organization}
\label{subsub:Library Organisation}
During the development of the Sequence library, we consistently paid close
attention to non-functional aspects. As part of this effort, we adjusted the
organization of the project's development setup. This led to the current state,
where each operation of the Sequence library is in a separate file. Adopting
such a project structure has significantly impacted the codebase's overall
clarity and navigability. Moreover, it contributes to a sense of order, which,
in turn, enhances the overall code quality. Additionally, it helps to prevent
cycling dependencies when importing particular functionalities.
\subsubsection{Code Quality}
\label{subsub:Code Quality}
Maintaining high code quality standards led to a clear and consistent
code base. Several points were particularly important during development:
\begin{itemize}
  \item{No code duplications}
  \item{Good naming}
  \item{Standardized formatting}
  \item{Only necessary exports of functions}
  \item{Appropriate comments and JSDoc, including examples}
  \item{Project organization and structure}
\end{itemize}

Strict adherence to these principles made development easier for us and helped
library users find their way around more quickly. Additionally, it facilitates
future developers to learn how to add new functionalities.

\subsection{User Testing}
\label{sub:User Testing}
The user testing conducted in section~\ref{sub:User Testing} was crucial for
assessing the usefulness of the Sequence library. It provided valuable feedback
from programmers without prior knowledge of the library. These insights allowed
for addressing specific issues and offered helpful improvements. Additionally,
other useful suggestions contributed to the list of potential enhancements, as
outlined in section ~\ref{sec:Future Features}.
