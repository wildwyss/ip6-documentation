\section{Query different Data Structures}
\label{sec:Query different Data Structures}
This section explains what Language Integrated Queries (LINQ) are and how they
can be implemented in JavaScript to query various data structures easily. It
also goes into detail about the implementation of LINQ in JavaScript. Finally,
the section shows how you can easily browse lists of JavaScript objects using
LINQ.
\subsection{Introduction to LINQ} % (fold)
\label{sub:Introduction to LINQ}
Some programming languages offer uniform ways to query different data
structures in an SQL-like syntax.

C\# calls this concept LINQ (Language Integrated Query). LINQ allows to
decarlatively query compatible data sources. Listing~\ref{lst:linq_in_csharp}
shows how to use LINQ to query a simple array of numeric values.

\begin{lstlisting}[
  style=sharpc,
  caption=LINQ in C\# \cite{billwagner_language-integrated_2023},
  label={lst:linq_in_csharp}
]
// Specify the data source.
int[] scores = { 97, 92, 81, 60 }; *'\label{line:linq_in_csharp1}'*

// Define the query expression.
IEnumerable<int> scoreQuery =
    from score in scores
    where score > 80
    select score;

// Execute the query.
foreach (int i in scoreQuery)
{
    Console.Write(i + " ");
}

// Output: 97 92 81
\end{lstlisting}

You can replace the data structure defined on line~\ref{line:linq_in_csharp1}
with any other one compatible with this API. This abstraction makes it very
easy to define reusable queries.


% subsection Introduction to LINQ (end)
\subsection{Why does this not exist in JavaScript?} % (fold)
\label{sub:Why does this not exist in JavaScript?}
However, JavaScript does not define a uniform API for data structures except
for the JS iteration protocols and thus cannot offer language-integrated queries
without further effort. \\
Section~\ref{sub:Making the Sequence Monadic} introduced the monadic sequence,
which provides additional operations to work with a sequence. All of these
operations are available through the sequence prototype. Since these
operations work solely on the committed properties of the JS iteration
protocols, the function |toMonadicIterable|, explained in
Section~\ref{sec:Iterables Everywhere} can quickly turn them into a monadic
sequence.\\
With that, a more extensive API is now available to every data structure
conforming to the JS iteration protocols. This monadic API makes it possible to
implement abstractions similar to LINQ in JavaScript.
% subsection Why does this not exist in JavaScript? (end)

\subsection{Introducing JINQ} % (fold)
\label{sub:Introducing JINQ}
JINQ (JavaScript integrated query) is the implementation of LINQ for
JavaScript. It can handle any data that conforms to the |MonadType| explained
in Section~\ref{subsub:The MonadType}. So JINQ can handle monadic iterables and
every monadic type, such as the type |Maybe| introduced in
Section~\ref{subsub:Doing the same in JavaScript}.\\
\textit{Note:} All operations supported by JINQ are explained in detail in 
Section~\ref{sec:API_JINQ}. This sections descibres how JINQ works
internally.

Listing~\ref{lst:keepeven_recap} shows again the function |keepEven| already
introduced in Section~\ref{subsub:The MonadType}, which works on a |Maybe| as
well as on a |Sequence|:
\begin{lstlisting}[
  style=ES6,
  caption=Recapitulate keepEven,
  label={lst:keepeven_recap}
]
const keepEven = monad => monad
  .and(x => {
    if (x % 2 === 0) {
      return monad.pure(x);
    } else {
      return monad.empty();
    }
}); 
\end{lstlisting}

JINQ makes it possible to simplify the implementation of this function.
Listing~\ref{lst:keepeven_jinq} therefore introduces a new function
|keepEvenJINQ|, which does precisely the same as |keepEven|:

\begin{lstlisting}[
  style=ES6,
  caption=keepEvenJINQ does the same as keepEven,
  label={lst:keepeven_jinq}
]
const keepEvenJINQ = monad =>
  from(monad)
    .where(x => x % 2 === 0)
    .result();
\end{lstlisting}

|keepEvenJINQ| is not only shorter but also more readable! It becomes clear
within moments what this function does, as it reads almost like prose!\\
This is the power of these abstractions - types must only follow a
minimal API to be compatible with JINQ. Additionally, they are very readable.

\textit{Note:} As explained before, JINQ works only on the monadic API. So you
can just as well use the monadic functions to achieve the same. However, the
example |keepEven| shows that it is easier to work with JINQ and save lines of
code.

\subsubsection{How does JINQ work?} % (fold)
\label{sec:How does JINQ work?}
JINQ uses a pattern analogue to the Builder pattern
\cite[Chapter~3.2]{gang_of_four_depa} to create a new structure which can be
used to transform the initially passed monad. \\
Have a look at the function on line~\ref{line:jinq_impl1} of
Listing~\ref{lst:jinq_impl}: If |from| is called, a new builder is created.
|from| expects a monad, which serves as the starting point of the builder. The
next operation executed on the builder operates on this monad. |result| then
returns the monad created based on the builder.

\textit{Note:} None of the functions change the parameter |monad| - therefore,
JINQ is immutable! This allows reusing an intermediate state of the JINQ
builder!

\begin{lstlisting}[
  style=ES6,
  caption=How is where implemented?,
  label={lst:jinq_impl}
]
// jinq.js
export { from }
const jinq = monad => ({ *'\label{line:jinq_impl1}'*
  pairWith: pairWith(monad),
  where:    where   (monad),
  select:   select  (monad),
  map:      select  (monad),
  inside:   inside  (monad),
  result:   () =>    monad
});

const from = jinq;*'\label{line:jinq_impl2}'*

// ...
\end{lstlisting}

Listing~\ref{lst:jinq_where_impl} uses the already familiar function |where| to
showcase how the builder operations work:

\begin{lstlisting}[
  style=ES6,
  caption=How is where implemented?,
  label={lst:jinq_where_impl}
]
// jinq.js
// ...

const where = monad => predicate => {
  const processed = 
    monad.and(a => predicate(a) ? monad.pure(a) : monad.empty()); *'\label{line:jinq_where_impl1}'*
  return jinq(processed);*'\label{line:jinq_where_impl2}'*
};

// ...
\end{lstlisting}

You may have already guessed it - the implementation of |where| is almost
precisely the implementation of |keepEven|! It is more general because the
predicate |x % 2 === 0| is outsourced to a function called |predicate|. See
line~\ref{line:jinq_where_impl1} in Listing~\ref{lst:jinq_where_impl}: |and|
keeps a value matching the predicate using |monad.pure| or discards it
otherwise using |monad.empty|.\\
Line~\ref{line:jinq_where_impl2} then wraps the resulting monad in a new
builder instance and returns it.

Another notable function of JINQ is |pairWith| - its implementation is shown in
Listing~\ref{lst:jinq_pairwith_impl}. Use it to combine a data source with
another one (or even with itself):

\begin{lstlisting}[
  style=ES6,
  caption=How is pairWith implemented?,
  label={lst:jinq_pairwith_impl}
]
// jinq.js
// ...

cconst pairWith = monad1 => monad2 => {
  const processed = monad1.and(x =>
    monad2.fmap(y => Pair(x)(y))
  );

  return jinq(processed)
};
// ...
\end{lstlisting}

|pairWith| takes a second monad. It now forms a new monad with a |Pair| in it!
But what happens here? Difficult to say because we can not know it! It just
calls |and| on |monad1| and combines it with |monad2|. \\
So when combining two sequences, every element of the first sequence gets
paired up with every element of the second sequence. One could argue that this
will take too much memory and could be more efficient. But let us step back and
think about Section 2.3, which discusses the laziness of sequences. |monad1|
(a sequence in the current example) is evaluated lazily! So never all
combinations will be materialized in memory at once!
% subsubsection How does JINQ work? (end)

\subsubsection{Creating Sequences using JINQ} % (fold)
\label{sec:Creating Sequences using JINQ}
A list comprehension in Haskell is an expression form that allows generating
lists in a declarative way. See \cite[Chapter 5]{hutton_pih_2016} for an
in depth introduction to list comprehensions. \\
Listing~\ref{lst:list_comp_hs} creates a list using a list comprehension in
Haskell:

\begin{lstlisting}[
  style=Haskell,
  caption=List comprehension in Haskell,
  label={lst:list_comp_hs}
]
pairs = [(i,j) |        -- create a list of pairs where
          i <- [1..10], -- i can have the values 1 to 10
          j <- [1..4],  -- j can have the values 1 to 4
          i - j == 1]   -- only keep pairs with i - j == 1

print pairs
-- Prints '[(2,1),(3,2),(4,3),(5,4)]'
\end{lstlisting}

Using JINQ, it is possible to create sequences similarly:
\begin{lstlisting}[
  style=ES6,
  caption=Creating Sequences using JINQ,
  label={lst:list_comp_js}
]
const pairs =
  from(Range(1,10))                 // create a seq with values 1 to 10
    .pairWith(Range(1, 4))          // union with a seq containing 1 to 4
    .where(([i, j]) => i - j === 1) // only keep pairs with i - j == 1
    .result();

console.log(pairs.fmap(show).toString());
// => Logs '[[2,1],[3,2],[4,3],[5,4]]'
\end{lstlisting}

Of course, list comprehensions in Haskell provide more syntacitcal sugar than
JINQ. Nevertheless, JINQ provides an easy way to create sequences based on
rules, coming quite close to a list comprehension.
% subsubsection Creating Sequences using JINQ (end)

\subsubsection{Using the JSONMonad to process lists of JavaScript objects} % (fold)
\label{subsub:Using the JSONMonad to process lists of JavaScript objects}
The |JSONMonad| provides arrays of JavaScript objects with a monadic API. The
|JSONMonad| makes them, therefore, compatible with JINQ. This is especially
useful when JavaScript Objects are created from JSON because they often have
missing or nullable fields. The monadic operations of |JSONMonad| deal with
that and ensure that querying nullable attributes do not cause problems.

The Listing~\ref{lst:jsonmonad_data} dives right into an example defining two
records that are related to each other. The first one contains data about an
ancient battle, while the second one contains data about the heroes of the
battle. The |heroId| connects these two records. The battle had a |winner| and
a |loser|. Imagine you want to find all names of the heroes of the winner team.

\begin{lstlisting}[
  style=ES6,
  caption=Two data sources,
  label={lst:jsonmonad_data}
]
const battleData = JSON.parse(`
  {
    "battleName": "The battle of Curly",
    "numberOfDeaths": 420000,
    "winner": { "teamName": "JSON", "outStandingHeroes": [1] },
    "loser": { "teamName": "XML", "outStandingHeroes": [] }
  }
`);

const heroes = JSON.parse(`
  [
    { "heroId": 1, "kills": 47076, "name": "Atonadias" },
    { "heroId": 2, "kills": 5691,  "name": "Tanobiri" },
    { "heroId": 3, "kills": 3707,  "name": "Tonadri" }
  ]
`);
\end{lstlisting}

Listing~\ref{lst:jsonmonad_example} combines these two data sources using JINQ.
Line~\ref{line:jsonmonad_example1} wraps the |battleData| with the |JSONMonad|
to become queriable. |select| then accesses the property |winner| and from
there the property |outstandingHeroes|. Line~\ref{line:jsonmonad_example2}
pairs the second data source before Line~\ref{line:jsonmonad_example3}, then
destructures the |Pair| and only keeps the heroes (from the second data source)
that belong to the winning team (from the first data source).

\begin{lstlisting}[
  style=ES6,
  caption=Combining data sources using JINQ and JSONMonad,
  label={lst:jsonmonad_example}
]
const outstandingHeroNames =
  from(JsonMonad(battleData)) *'\label{line:jsonmonad_example1}'*
    .select   (x => x["winner"])
    .select   (x => x["outStandingHeroes"])
    .pairWith (JsonMonad(heroes)) *'\label{line:jsonmonad_example2}'*
    .where    (([heroId, hero]) => heroId === hero["heroId"]) *'\label{line:jsonmonad_example3}'*
    .select   (([_, hero]) => hero["name"])
    .result   ();

console.log(...outstandingHeroNames);
// => Los 'Atonadias'
\end{lstlisting}

% subsubsection Using the JSONMonad to process lists of JavaScript objects (end)
\subsubsection{Conclusion} % (fold)
\label{subsub:Conclusion}
Monadic APIs allow the building of very general abstractions that can handle a
wide variety of data types, even if they seem to have nothing in common at
first glance - such as a sequence or a Maybe! JINQ showcases an excellent
instance of such a general abstraction - its operations only compose generic
monadic functions. Nevertheless, JINQs operations have very expressive names
describing their purpose, making it easy to use JINQ with different data types!
That is truly beautiful and quite rare: general concepts with specific
names!\\ 
JINQ shows its versatility through the possibility of creating new lists based
on declarative rules and using the |JSONMonad|.
% subsubsection Conclusion (end)
% subsection Introducing JINQ (end)
% section Why does this not exist in JavaScript? (end)
