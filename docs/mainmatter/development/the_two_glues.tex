\section{Modularizing programs}
This section explains how to break down programs and functions into small,
reusable pieces using lazy evaluation and higher-order functions. The start
shows how Haskell implements these concepts. The subsequent discussion covers
the application and limitations of these concepts in JavaScript before showing
how the Sequence Library overcomes these limitations, providing valuable tools
for writing reliable and reusable code.

\subsection{Gluing Programs together} % (fold)
\label{sub:Gluing Programs together}

John Hughes says functional programming languages provide two extra ways to
better partition programs and easily hang them together later. He refers to
this as "glue". These "glues" are on the one hand higher order functions and on
the other hand lazy evaluation. \cite{hughes_why_1989}
% subsection Gluing Programs together (end)


\subsection{Using these Glues in Haskell} % (fold)
\label{sub:Using these Glues in Haskel}

\subsubsection{Higher Order Functions in Haskell} % (fold)
\label{sec:Higher order functions Haskell}
Higher order functions are functions that receive another function as arguments
or produce a function as a result. In Haskell, where the whole program is
actually a single function, this concept is of course particularly strong.
The easiest way to understand higher order functions is to look at an
example:

\begin{lstlisting}[style=Haskell, caption=Higher order functions in
Haskell \label{lst:hof_haskell}]
double    = \x -> 2 * x
quadruple = \x -> 4 * x
doubles   = map double [1,2,3,4]
quads     = map quadruple [1,2,3,4]

print $ doubles ++ quads
-- Prints '[2,4,6,8,4,8,12,16]'
\end{lstlisting}

Listing~\ref{lst:hof_haskell} shows how higher order functions work in Haskell.
Code can easily be reused this way, since |map| can be parametrized with a
function.
% subsubsection Higher order functions Haskell (end)

\subsubsection{Lazy evaluation} % (fold)
\label{subsub:Evaluation}
Lazy evaluation means that only as many program statements are evaluated as are
actually required. This makes it possible to work with lists that have an
infinite amount of values. This can best be shown in an example as well.

\begin{lstlisting}[
  style=Haskell,
  caption=lazy evaluation in Haskell,
  label={lst:lazy_eval_haskell}
]
as = [1..]*'\label{line:lazy_eval_haskell_1}'*
doubles = map double as*'\label{line:lazy_eval_haskell_2}'*

print $ take 4 doubles 
-- Prints '[2,4,6,8]'
\end{lstlisting}

Line~\ref{line:lazy_eval_haskell_1} in Listing~\ref{lst:lazy_eval_haskell}
defines a list with infinite values is created. Then the function |double|
defined in Listing~\ref{lst:hof_haskell}, is applied on this list. Since the
list has an infinite amount of elements this would take forever. But since
Haskell evaluates its statements lazy, Nothing happens on
line~\ref{line:lazy_eval_haskell_2}. The |double| function is only applied when
the elements are actually used. This is only the case when they are printed
with |print|. With |take| the list is reduced to four elements. The function
|double| is then only applied to these four elements. \\ More theoretically,
when the function |f| is applied to |g x|, |g| produces only as much output as
|f| actually needs. Therefore, working with a big amount data becomes much more
convenient and faster as well. If the elements in a list are generated with a
rule (as with list comprehensions), lazy evaluation brings great advantages:
The whole list will never materialize in the memory, but only the current
value. So you don't have to worry about what happens when large amounts of data
are processed. \\ Using these two concepts, it is much easier to build large
programs consisting of small parts.
% subsubsection Lazy evaluation (end)
% subsection Using the glues in Haskell (end)

\subsection{Using these Glues in JavaScript} % (fold)
\label{sub:Using these Glues in JavaScript}

\subsubsection{Higher Order Functions in JavaScript} % (fold)
\label{subsub:Higher Order Functions in JavaScript}
Higher order functions in JavaScript work similarly as they to in Haskell. \\
So the meaning of the Listing~\ref{lst:hof_haskell} can be translated to 
JavaScript like:

\begin{lstlisting}[
  style=ES6,
  caption=Higher order functions in JavaScript,
  label={lst:hof_in_js}
]
const double    = x => 2 * x;
const quadruple = x => 4 * x;
const doubles   = [1,2,3,4].map(double);
const quads     = [1,2,3,4].map(quadruple);

console.log(doubles.concat(quads));
// => Logs '2, 4, 6, 8, 4, 8, 12, 16'
\end{lstlisting}
% subsubsection Higher Order Functions in JavaScript (end)

\subsubsection{Eager Evaluation in JavaScript} % (fold)
\label{subsub:Eager Evaluation in JavaScript}

% subsubsection Lazy Evaluation in JavaScript (end)
When it comes to lazy evaluation in JavaScript, it becomes less intuitive:
JavaScript does not know this concept when working with its in-built arrays.
Mapping over an array producing a side effect can shed some light on this:

\begin{lstlisting}[
  style=ES6,
  caption=JavaScript evaluates eagerly,
  label={lst:js_eagerly_eval}
]
const as     = [1,2,3,4,5];
const bs     = [];
const mapped = as.map(x => {
  bs.push(x);
  return 2*x;
});

console.log(mapped.slice(0,4));
// => Logs '2, 4, 6, 8'
console.log(bs);
// => Logs '2, 4, 6, 8, 10'
\end{lstlisting}

Through Listing~\ref{lst:js_eagerly_eval} it becomes clear, that although only
the first four elements of the array are actually used, each element of the
array is still traversed using the |map| function. This can be seen from the
fact that the array |bs| contains all elements of the array |as| and not just
the first four.

\subsubsection{Lazy Evaluating Iterables} % (fold)
\label{Lazy Evaluate Iterables}

% subsubsection Lazy Evaluate Iterables (end)
The eager evaluation of JavaScript is a significant limitation when working with it.
Fortunately, JavaScript provides a way to emulate this laziness using the
Iterable protocols, which the section~\ref{sec:Sequence and Iterable in General}
describes. \\ 

With the use of the sequence library, it is possible to rewrite the JS Code of
Listing~\ref{lst:js_eagerly_eval} lazily:


\begin{lstlisting}[
  style=ES6,
  caption=Lazy evaluation in JavaScript,
  label={lst:lazy_eval_js}
]
const as     = [1,2,3,4,5];
const bs     = [];
const mapped = map(x => {
  bs.push(x);
  return 2*x;
})(as);

console.log(...take(4)(mapped));
// => Logs '2, 4, 6, 8'
console.log(bs);
// => Logs '2, 4, 6, 8'
\end{lstlisting}

The first output is the same as in Listing~\ref{lst:js_eagerly_eval} - the
first four elements of the array |as| have been multiplied by 2. However, the
second output differs: the array |bs| now contains only four elements. So the
function passed to |map| was only applied to four elements of the initial
array! This proves that arrays are processed lazily when transformed using the
Sequence Library!\\
This works because the iterable protocols process element by element. The
iterator delivers the next element only if it is requested. If no further
element is needed, you just don't ask the iterator for the next element! This
decouples the termination condition (asking the iterator for another element)
from the loop body (how the iterator generates/accesses the next element).\\

Furthermore, it is even possible to port the initial Haskell laziness
example defined in Listing~\ref{lst:lazy_eval_haskell} to JavaScript:
\begin{lstlisting}[
  style=ES6,
  caption=Process infinite sequences in JavaScript,
  label={lst:inf_seq_js}
]
const seq = Sequence(1, _ => true, x => x + 1);*'\label{line:inf_seq_js_1}'*
const doubles = map(double)(seq);

console.log(...take(4)(doubles));
// => Logs '2, 4, 6, 8'
\end{lstlisting}

If the code displayed in Listing~\ref{lst:inf_seq_js} were evaluated eagerly,
this would go on forever since line~\ref{line:inf_seq_js_1} defines a sequence
with infinite elements. \\ 

\subsection{Lazy evaluation and side effects} % (fold)
\label{sub:Lazy evaluation and side effects}
Decoupling the loop body and termination condition helps to consider these two
concepts separately. This split reasoning makes sense since they often do not
relate to each other. Why change the whole list of elements if the program only
uses the first five? In JavaScript, however, this thinking carries a danger
that cannot occur in pure functions like those used in Haskell. \\ 
The following Listing~\ref{lst:side_effects_lazy_js} states the problem:

\begin{lstlisting}[
  style=ES6,
  caption=Side effects and lazy evaluation,
  label={lst:side_effects_lazy_js}
]
let i = 0;
const mapped = map(x => {
  i++;
  return 2 * x;
})([1,2,3,4]);

console.log(i);
// => Logs '0' i has not been incremented
console.log(...take(2)(mapped);
// => Logs '2, 4' 
console.log(i);
// => Logs 2 i has only been incremented two times
\end{lstlisting}

The developer writing the loop body cannot infer from the length of the list
how often the function runs. Performing side effects in the loop body can lead
to unexpected behavior, therefore. 

It is tough to make assumptions about |i|, because:
\begin{itemize}
  \item |i| is only calculated when the program is evaluated
  \item It is not possible to predict what value |i| will have when the program
    finishes.
\end{itemize}
This example thus shows that lazy evaluation and side effects almost exclude
each other. Therefore, omitting side effects when working with lazy evaluation
makes sense. In Haskell, where the type system can exclude side effects per
function, this is not a problem. Since side effects often make a program harder
to understand, getting rid of it feels even more natural.
% subsection Lazy evaluation and side effects (end)

\subsubsection{Placing the Receiver at the End} % (fold)
\label{subsub:Placing the Receiver at the End}
Compared to object-oriented programming languages, where the receiver of an
operation is always placed before the point (i.e., at the beginning), functions
in Haskell usually expect it as the last argument. If one sticks to this, it is
possible to build a new function from a pre-configured function in Haskell
(line~\ref{line:precon_fs_hs_1} of Listing~\ref{lst:precon_fs_hs}) and then
reuse these this function:

\begin{lstlisting}[
  style=Haskell,
  caption=Pre-configured functions in Haskell,
  label={lst:precon_fs_hs}
]
double    = \x -> 2 * x
doubleAll = map double --*'\label{line:precon_fs_hs_1}'* preconfigure map with the function double
as        = [1,2,3,4]
bs        = [5,6,7,8]

print $ doubleAll as ++ doubleAll bs
-- Prints '[2,4,6,8,10,12,14,16]'
\end{lstlisting}

This paradigm allows it also to link existing functions together using the |.|
operator:

\begin{lstlisting}[
  style=Haskell,
  caption=Function composition in Haskell,
  label={lst:fn_comp_hs}
]
-- "sum" sums up all elements in the list
doubleAndSum = sum . doubleAll

print $ doubleAndSum [1,2,3,4]
-- Prints '20'
\end{lstlisting}

The Listing~\ref{lst:fn_comp_hs} links the function |doubleAll| defined in
Listing~\ref{lst:precon_fs_hs} and the function |sum| to create a new function
applicable to any list with numerical values. This concept makes the code very
reusable.\\
Placing the receiver at the end is another powerful modularization
tool. Therefore, the sequence library supports this kind of modularization.
Let us refactor the initial example from Listing~\ref{lst:hof_in_js}, which 
shows how higher-order functions work in JavaScript:
\begin{lstlisting}[
  style=ES6,
  caption=Receiver before the point,
  label={lst:rec_bef_point}
]
// initial code
const double  = x => 2 * x;
const doubles = [1,2,3,4].map(double);

// refactored using plain JS
const doubleAll = receiver => receiver.map(double);
console.log(doubleAll([1, 2, 3, 4]));
// => Logs '2, 4, 6, 8'
\end{lstlisting}

The Listing~\ref{lst:rec_bef_point} shows that this could be fixed using a
function which gets the receiver as an argument. However, this is not very
convenient. The sequence library provides a more elegant way to solve this:

\begin{lstlisting}[
  style=ES6,
  caption=Place the receiver at the end in JavaScript,
  label={lst:rcv_at_end_js}
]
const doubleAll = map(double);
\end{lstlisting}

The code in Listing~\ref{lst:rcv_at_end_js} is not only less to type but also
more accurate when reading because the term |doubleAll = map(double)| says that
the pre-configured function |map(double)| gets a new name (|doubleAll|). No worrying about
parameter naming is needed here.
% subsubsection Placing the Receiver at the End (end)

