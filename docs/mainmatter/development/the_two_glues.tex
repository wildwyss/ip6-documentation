\section{Gluing programs together}
John Hughes says functional programming languages provide two extra ways to
better partition programs and easily hang them together later. He refers to
this as "glue". These "glues" are on the one hand higher order functions and on
the other hand lazy evaluation. \cite{hughes_why_1989}

\subsection{Using the glues in Haskell} % (fold)
\label{sub:Using the glues in Haskel}
\subsubsection{Higher order functions Haskell} % (fold)
\label{sec:Higher order functions Haskell}
Higher order functions are functions that receive another function as arguments
or produce a function as a result. In Haskell, where the whole program is
actually a single function, this concept is of course particularly strong. \\
The easiest way to understand higher order functions is to look at an
example:\\

% subsubsection Higher order functions Haskell (end)
% subsection Using the glues in Haskell (end)
