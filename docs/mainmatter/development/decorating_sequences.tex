\section{Decorating Sequences}
\label{sec:Decorating Sequences}
This section explores how to modify sequences by implementing the decorator 
pattern and effectively managing sequence state.

\subsection{Decorator Pattern}
\label{sub:Decorator Pattern}
Let us look at the decorator pattern~\cite[p.~226]{gang_of_four_depa} to understand the content of the 
following sections. In object-oriented programming, the decorator pattern is a 
widely used concept: An object decorates another, as the name implies. As a 
result, an outer object holds an inner object, while both implement the same 
interface. The outer object forwards requests to the inner one. This enables
the outer to modify (decorate) the behaviour of the inner.
Figure \ref{fig:seq_decorator} shows how a decorator forwards the receiving calls and 
transfers the answer back to the client.

\begin{figure}[H]
  \centering
  \begin{sequencediagram}                                                      
    \newthread{client}{:Client}                                                        
    \newinst[3]{decorator}{:Decorator}                                                     
    \newinst[3]{decorated}{:Decorated}                                                     

    \begin{call}{client}{operation()}{decorator}{return value}                                  

      \begin{sdblock}{}{optionally do somthing befor calling
        inner.operation()}
        \begin{call}{decorator}{}{decorator}{}                                  
        \end{call}                                                                    
      \end{sdblock}

      \begin{call}{decorator}{operation()}{decorated}{}                                  
      \end{call}                                                                    

      \begin{sdblock}{}{optionally do somthing after calling
        inner.operation()}
        \begin{call}{decorator}{}{decorator}{}                                  
        \end{call}                                                                    
      \end{sdblock}

    \end{call}                                                                    
  \end{sequencediagram}    
  \caption{Decorator sequence diagram}
  \label{fig:seq_decorator}
\end{figure}

The decorator pattern is often used in object-oriented programming languages 
because of its ability to add functionality to an object at runtime. It is also 
possible to add extended functionality to a single class object. This pattern
gives a great way to modify the behaviour of iterables! The Sequence library
uses this approach to create operations to transform sequences.

\subsection{Processing Sequences}
\label{sub:Processing Sequences}

\subsubsection{Processing Iterables with Functions}
\label{subsub:Processing Iterables with Functions}
In the following, |map| serves as a representative for any function of the
Sequence library. The Sequence library calls such functions operators, according
to the module they are in.
Listing~\ref{lst:impl_map} shows how |map| processes a sequence. The
implementation uses the decorator approach just mentioned. |map| decorates a
sequence on line~\ref{line:obj_mapped}.
\newline
The function signature of |map| on line~\ref{line:args} shows that a client
can invoke |map| with two arguments:

\begin{enumerate}
  \item{A mapping-function to transform one single element.}
  \item{An object of type |Iterable<T>|}
\end{enumerate}

An iterable must adhere to the JS iteration protocols outlined in
section~\ref{subsub:The Iterable Protocol}. Therefore, |map| can process
sequences, arrays, or any other iterable. 

\begin{lstlisting}[
  style=ES6, 
  caption=Implementation of map,
  label={lst:impl_map}
  ]
const map = mapper => iterable => { *'\label{line:args}'*
*'\label{line:state_iterable}'*
const mapIterator = () => {
   const inner = iteratorOf(iterable);*'\label{line:state_iterator}'*
   let mappedValue;
 
   const next = () => {
     const { done, value } = inner.next();*'\label{line:inner_next}'*
     if (!done) mappedValue = mapper(value);
 
     return { done, value: mappedValue }
   };
 
   return { next };
  };

  *'// explained in \ref{sub:Adding Monadic Functions to Iterables}'*
  return createMonadicSequence(mapIterator); };

const sequence = Sequence(0, x => x < 10, x => x + 1);*'\label{line:seq_definition}'*
const mapped   = map(x => x * 2)(sequence);*'\label{line:obj_mapped}'*
\end{lstlisting}

Because |map| decorates iterables, it also returns an object adhering to JS
iteration protocols. We define this object on line~\ref{line:obj_mapped} as
|mapped|. Consequently, |map| also has a function called |next|. Since the JS
iteration protocols specify only this single function, it is the only one that
must be exposed. The newly created object |mapped| forwards function
calls on |next| to the inner function |next|, on line~\ref{line:inner_next}.
The |mapper| function then processes the result and returns it. So |map|
\textit{decorates} the function |next| of the inner iterable.

\subsection{Benefits of the Decorator Approach}
\label{sub:Benefits of the Decorator Approach}
\subsubsection{Stand-alone Functions}
\label{subsub:Standalone Functions}
In an object-oriented approach, the sequence object would provide a function
|map|. Therefore, such operators are available by using the dot notation,
similar to Java implementing the Stream API \cite{java_stream}.
However, with the approach of providing independent functions, there 
are three significant advantages:

\begin{enumerate}
  \item {Adherence to the open-close
      principle~\cite[p.~3]{eilebrecht_patterns_2019}. Changes to an operator
      do not affect the implementation of constructors which create sequences. Also,
      extensions to the Sequence library do not affect existing code.
    }
  \item{Adherence to the single responsibility approach. The constructor
    |Sequence| has only the task of creating a sequence. Mapping the elements
  of a sequence is not its responsibility. }
  \item{Easy scalability is guaranteed. It is very straightforward to add new
    functionality from the outside. }
\end{enumerate}

Additionally, the |operators| implement their arguments in a curried style and
accept the receiving iterable at the last position. This allows using
eta-reduction and other advantages, as section~\ref{sec:Placing the Receiver at the End} outlines.

\subsubsection{Stateful Decorating}
\label{subsub:Stateful Decorating}
A state is present as soon as operators decorate iterables or implement 
additional functionalities. In the case of |map| this state is the function
|mapper| and the iterator of the inner iterable.

There are two valid possibilities for including the state:
\begin{enumerate}
  \item Placing the state into the closure scope of the iterable.
  \item Implementing the state into the closure scope of the iterator.
\end{enumerate}
In both variants, ensuring the underlying object remains unchanged is crucial.

\subsubsection{Scenario 1: Placing the State inside the Iterable}
\label{subsub:Scenario 1}
Listing~\ref{lst:scen_1} shows an example of a sequence generating numbers from
zero to five. On line ~\ref{line:scen_1_state}, value |i| works as a counter 
and represents the state. 
A call to |SampleIterable| creates a state which is valid for the entire
iterable's lifetime.

\begin{lstlisting}[
  style=ES6, 
  caption=Scenario 1 - State in closure scope of iterable,
  label={lst:scen_1}
  ]
const SampleIterable = () => {

  let i = 0; *'\label{line:scen_1_state}'*
  const next = () => {
    return { done: i > 5, value: i++ };
  };

  return { [Symbol.iterator]: () => ({ next }) }
};
\end{lstlisting}

Listing~\ref{lst:scen_1_demo} shows a possible program flow. The program first
creates an object of |SampleIterable|, maps it and processes an element on the
original object. So both for loops work on the same underlying iterable.
Line~\ref{line:scen1_output} shows an expected result.

\begin{lstlisting}[
  style=ES6, 
  caption=Scenario 1 - Example usage,
  label={lst:scen_1_demo}
  ]
const seq = SampleIterable(); // [0, 1, 2, 3, 4]
const mapped = map(id)(seq); *'\label{line:scen1_map}'*

for (const elem of mapped) {
  console.log(elem);
  break; // Just consuming one element
}
// => Logs: '0'

for (const elem of seq) {
  console.log(elem);
}
// => Should log: '0, 1, 2, 3, 4' *'\label{line:scen1_output}'*
\end{lstlisting}
However, since both loops operate on the same iterable (|seq|), the second output
will not give the expected result but "1, 2, 3, 4".\\
To achieve the expected result, |map| must copy the object before processing.
That implies that each iterable must be copyable. An example implementation
of a copyable iterable shows listing~\ref{lst:iterable_with_copy}.

\begin{lstlisting}[
  style=ES6, 
  caption=Iterable with copy,
  label={lst:iterable_with_copy}
  ]
const SampleIterable = () => {

  let i = 0;
  const next = () => {
    return { done: i > 5, value: i++ };
  };

  *'\colorbox{code-highlight}{const copy = () => SampleIterable();}'*

  return {
    [Symbol.iterator]: () => ({ next }),
    copy: copy *'\label{line:iterable_copy}'*
  };
};
\end{lstlisting}
Since |copy| must be callable from outside, returning it alongside 
|[Symbol.iterator]| on line~\ref{line:iterable_copy} is required. |map| is now
able to copy the underlying  iterable before consuming its elements.
However, |map| itself must also implement |copy|, because its interface must be
the same as the decorated object (and also to ensure that iterables created
using |map| can be decorated as well). The following listing~\ref{lst:map_with_copy} 
shows an implementation of |map| supporting copy:

\begin{lstlisting}[
  style=ES6, 
  caption=Implementation of map with copy,
  label={lst:map_with_copy}
  ]
const map = mapper => iterator => {
 const inner = iterator.copy();*'\label{line:copy_of_inner}'*
 let mappedValue;

 const next = () => {
   const { done, value } = inner[Symbol.iterator]().next();*'\label{line:consuming_inner}'*
   if (!done) mappedValue = mapper(value);
   return { done, value: mappedValue }
 };

 return {
   [Symbol.iterator]: () => ({ next }),
   copy: () => map(mapper)(inner)*'\label{line:return_copy_map}'*
 };
};
\end{lstlisting}
On line~\ref{line:copy_of_inner}, |map| references a copy of the original
iterable. Further processing on line~\ref{line:consuming_inner} works on this 
copy to protect the original from being mutated.

In many respects, |copy| is an elaborate thing. All operators and constructors 
dealing with iterables must implement it. Thus, implementation errors are 
almost certain. On the other hand, it is an extra effort from a performance point 
of view. It means more function calls and also larger call stacks to manage.
The advantage of this implementation is that partially processed iterators can
be further utilized while maintaining their current state without
reinitializing the state with a new iteration.
Operators can only process objects that implement |copy|.
Other iterable objects are not processable any more, which is a significant
drawback.

\subsubsection{Scenario 2: Placing the State inside the Iterator}
\label{subsub:Scenario 2}
Listing~\ref{lst:scen_2} represents scenario 2. Here, the state is on
line~\ref{line:scen_2_state} inside the iterator. Running the code from the
previous example~\ref{lst:scen_1_demo} using this implementation produces the
expected output straight away. The distinction lies in the fact that the object
does not need to be copyable. Each call to |[Symbol.iterator]| creates a new
state. Apart from the iterator, this iterable is immutable. Another advantage
is that all Sequence library operators can handle any object conforming to the
JS iteration protocols!

\begin{lstlisting}[
  style=ES6, 
  caption=Scenario 2 - State in closure scope of iterator,
  label={lst:scen_2}
  ]
const SampleIterable = () => {

  return {
    [Symbol.iterator]: () => {
      let i = 0; *'\label{line:scen_2_state}'*
      return {
        next: () => ({ done: i > 5, value: i++ })
      };
    };
  };

};
\end{lstlisting}


\subsubsection{Comparing the two Scenarios}
\label{subsub:Comparing the Two Scenarios}
As mentioned in section~\ref{sub:Types of Iterables}, there is a reason for 
the deep nesting of the iterator into the iterable object. 
Each iterator can have its own state and the iterable itself is immutable! \\
In programming, immutability refers to creating objects that cannot be changed
once they are created. There are several advantages to using immutability in
code: \\
Firstly, it helps in creating more reliable and bug-free programs. Since
immutable objects cannot be modified, there is no risk of accidental changes or
unintended side effects. This simplifies the code and makes it easier to reason
about. \\
Secondly, immutability promotes functional programming practices. Immutable
data structures are crucial in functional programming, allowing for more
predictable and declarative code that is easier to test.

In scenario 2, the immutable approach has a specific impact on an iterable.
When an iterable is immutable, each time we access its iterator, a new iterator
instance is returned.
This means it is impossible to process an iterable partially or in multiple stages.
To achieve this behaviour, we need to implement scenario 1.
\\
Because of the significant benefits of immutability and, consequently, the
advantages of scenario 2, the Sequence library uses this concept. This way,
developers who work with sequences can be confident that they always operate
with a valid and unused instance that does what they expect.

\subsection{Adding Monadic Functions to Iterables}
\label{sub:Adding Monadic Functions to Iterables}
This last section of this chapter explains the function
|createMonadicSequence|, which the previous sections left out. Many of the
operations of the Sequence library use this function for creating new
sequences.

\subsubsection{A Convenience Function}
\label{subsub:A Convenience Function}
Consider listing~\ref{lst:impl_map} once again showing the implementation of
|map|. This implementation uses a function |mapIterator|, which returns an
object containing a function |next|. Thus, |mapIterator| forms an iterator.
|createMonadicSequence| now receives such an iterator as parameter.
Listing~\ref{lst:createMonadicSequence} demonstrates the implementation of
|createMonadicSequence|:

\begin{lstlisting}[
  style=ES6, 
  caption=Implementation of createMonadicSequence,
  label={lst:createMonadicSequence}
  ]
/**
 * Builds an {@link SequenceType} from a given {@link Iterator}.
 * @template _T_
 * @param { () => Iterator<_T_> } iterator
 * @returns { SequenceType<_T_> }
 */
 const createMonadicSequence = iterator => { *'\label{line:start_createMonadicSequence}'*
    const result = {[Symbol.iterator]: iterator};
    return setPrototype(result);
};*'\label{line:end_createMonadicSequence}'*
\end{lstlisting}

Line~\ref{line:start_createMonadicSequence}
to~\ref{line:end_createMonadicSequence} implement the functionality of
|createMonadicSequence|:

\begin{itemize}
  \item{Assign the passed iterator to the |[Symbol.iterator]| property}
  \item{Add the sequence prototype to the resulting object}
\end{itemize}


\subsubsection{Sequence Prototype}
\label{subsub:Sequence Prototype}
\textcquote{mdn_prototype_2023}{Prototypes are the mechanism by which
JavaScript objects inherit features from one another.}\\ 
Multiple objects can use the same implementation of functions simultaneously if
they have the same prototype.\\
\textit{Note:} An object can have just \textit{one} prototype. Hierarchies
emerge through chaining prototypes, meaning setting the prototype of a
prototype.

A prototype is just a function. A static function |setPrototypeOf| adds this
prototype to an existing object. Thus, any properties set on the prototype
become available to that object. These properties can access the current object
directly with the keyword |this|. Listing~\ref{lst:js_set_proto} shows how this
works for any object:

\begin{lstlisting}[
  style=ES6,
  caption=Setting the prototype to an object,
  label={lst:js_set_proto}
]
// The prototype is just a function
const Prototype = () => null; 

// Add a new function to this prototype
Prototype.getName = function () { return this.name; };

// Creating an object with a name
const tw = { name : "Tobias Wyss" };


// Setting the prototype
Object.setPrototypeOf(tw, Prototype);

console.log(tw.getName());
// => Logs 'Tobias Wyss' 
\end{lstlisting}


The function |setPrototype| on line~\ref{line:setPrototype} of
listing~\ref{lst:set_prototype} adds the sequence prototype to an iterable
using the function |setPrototypeOf|. This causes some functions to be
accessible on each sequence using the "|.|" (dot) notation. These are, among
others, monadic functions. You will read more about monads and their purpose in

\begin{lstlisting}[
  style=ES6,
  caption=Implementation of setPrototype,
  label={lst:set_prototype}
]
/**
 *
 * @template _T_
 * @param { Iterable<_T_> } iterable
 * @returns { SequenceType<_T_> }
 */
 const setPrototype = iterable => { *'\label{line:setPrototype}'*
  Object.setPrototypeOf(iterable, SequencePrototype);
  return /**@type SequenceType*/ iterable;
};
\end{lstlisting}

\subsubsection{JSDoc for the SequenceType} % (fold)
\label{subsub:The SequenceType}
JSDoc is the optional type system of JavaScript~\cite{jsdoc_use_2023}. As the
name says, it is pure documentation, therefore, not enforceable. Since newer
IDEs support it very well, it is still very reasonable.\\
The |SequenceType| is the JSDoc representation of a sequence.
Listing~\ref{lst:sequence_type} shows the definition of the type. Alongside, the
monadic functions |toString|, |pipe|, and equals |"=="| are also
accessible via the prototype. Having these functions on each |SequenceType| is
very convenient, as they improve readability and usability.

\begin{lstlisting}[
  style=ES6, 
  caption=SequenceType definition,
  label={lst:sequence_type}
  ]
/**
 * @template _T_
 * @typedef {
 * {
 *   and:  <_U_>(bindFn: (_T_)    => SequenceType<_U_>) => SequenceType<_U_>,
 *   pure: <_U_>(_U_)             => SequenceType<_U_>,
 *   fmap: <_U_>(f: (_T_) => _U_) => SequenceType<_U_>,
 *   empty:     ()                => SequenceType<_T_>,
 *   pipe: function(...transformers: SequenceOperation<*,*>): SequenceType<*>|*,
 *   toString:  ()                => String,
 *   "==":      (that:SequenceType<_T_>) => Boolean
 *  } & Iterable<_T_>
 * } SequenceType
 */
\end{lstlisting}

This approach ensures that sequences become monadic but still iterable.
Monadic APIs can thus also process sequences. An example of such an API is
JINQ, which section~\ref{sec:API_JINQ} discusses in detail.

\subsubsection{SequenceType vs. Iterable} % (fold)
\label{subsub:SequenceType vs. Iterable}
The |SequenceType| describes iterable objects having their
prototype set to |SequencePrototype|. An object of type iterable is just
iterable.
Every operator of the Sequence library receives an iterable as a parameter and
returns an object of type |SequenceType|.

\subsubsection{Turning an Iterable into a SequenceType}
\label{subsub:Turn Iterables into SequenceType}
Sometimes the |SequenceType| must be set to an arbitrary iterable. For example,
if you want to use the monadic functions on a JavaScript array. To make this
possible, the Sequence library provides a |toMonadicIterable| function.
Listing~\ref{lst:casting_iterables} shows the
implementation of it. This function takes an arbitrary iterable and returns a
|SequenceType| mapped by |id|\footnote{The id (identity) function always
returns the value passed to it.}. 
Therefore, the returned object is then of the desired type |SequenceType|.

\begin{lstlisting}[
  style=ES6, 
  caption=Casting an arbitrary iterable to SequenceType,
  label={lst:casting_iterables}
  ]
/**
 * Casts an arbitrary {@link Iterable} into the {@link SequenceType}.
 * @template _T_
 * @param { Iterable<_T_> } iterable
 * @return { SequenceType<_T_> }
 */
const toMonadicIterable = iterable => map(id)(iterable);
\end{lstlisting}
