\section{Monads in JavaScript} % (fold)
\label{sec:Monads in JavaScript}
The first part of this section describes what contexts in Haskell are and how
Haskell wraps values in such contexts. "Maybe" serves as an example of this.
Implementing this example using JavaScript shows that such contexts are also
useful in other programming languages.\\
The second part then explains what monads are and shows how to implement these
in JavaScript. It also highlights the missing features of JavaScript to adopt
this concept exactly from Haskell and describes acceptable workarounds. Thus, it
is also possible to transform values in a context in JavaScript. The
last part shows what the monadic operations of the Sequence look like.

\subsection{Wrapping Values in a Context} % (fold)
\label{sub:Wrapping values in a context}

In Haskell, you often work with values wrapped in a particular context. This
context can be, for example, a list. Nevertheless, this context can also be
another data structure, for instance, |Maybe|. \\
For the context |Maybe| two implementations in Haskell co-exist:

% section Monads in JavaScript (end)
\begin{lstlisting}[
  style=Haskell,
  caption=The data type Maybe in Haskell,
  label={lst:maybe_hs}
]
-- defining the datatype Maybe
data Maybe a = Nothing | Just a
\end{lstlisting}

Listing~\ref{lst:maybe_hs} defines this datatype. Why is this useful?
Imagine you want to create a function |head|, which returns the first value of
a given list. |head| is pretty simple to implement. But wait, what to do when
the list is empty? In object-oriented languages, one might return |null|. The
problem with this solution is that the user must remember that the list can be
empty, and thus the result of |head| can be |null|. This is very error-prone.
\\

That is where the new datatype |Maybe| comes in. |Maybe| allows describing
either the absence of a value or the value itself. The following
Listing~\ref{lst:safe_head} defines a new function |safeHead|
(line~\ref{line:safe_head1}), which does
precisely this - it returns |Just a| when the list is not empty
(line~\ref{line:safe_head2}) and |Nothing| otherwise (line~\ref{line:safe_head3}):

\begin{lstlisting}[
  style=Haskell,
  caption=Safely get the first element of a list,
  label={lst:safe_head}
]
-- a function which produces a Maybe:
safeHead :: [a] -> Maybe a *'\label{line:safe_head1}'*
safeHead (a:_) = Just a *'\label{line:safe_head2}'*
safeHead []    = Nothing *'\label{line:safe_head3}'*

printHead list = print $ case (safeHead list) of 
  Just val -> show val
  Nothing  -> "List was empty!"

printHead [1,2,3,4]
-- Prints '"1"'
printHead []
-- Prints '"List was empty!"'
\end{lstlisting}

When returning just |null|, the user has to explicitly deal with the case where
there is no first element. This leads to improved safety.

\subsubsection{Doing the Same in JavaScript} % (fold)
\label{subsub:Doing the same in JavaScript}
Listing~\ref{lst:maybe_js} defines the same type in JavaScript:
\begin{lstlisting}[
  style=ES6,
  caption=The data type Maybe in JavaScript,
  label={lst:maybe_js}
]
const Just    = x => _f => g => g(x);*'\label{line:maybe_js1}'*
const Nothing = f => _g => f(undefined);*'\label{line:maybe_js2}'*
\end{lstlisting}

So |Just| and |Nothing| are only functions! How could it be different in
functional programming? \\ 
Line~\ref{line:maybe_js1} defines the |Just| case:
|Just| takes a value |x| and two functions while not using the first function
at all. |Just| applies the second function to the initial value |x|. \\ 
|Nothing| looks very similar to |Just|, with the difference that it does not
receive a value. Furthermore, |Nothing| calls the second function at the end.\\ 
Now how to use this? Noticeably, |Just(value)| and |Nothing| have the same
structure: both receive two functions as arguments. Meaning they are
structurally the same. Listing~\ref{lst:js_safeHead} takes advantage of this
property to port the previous function |safeHead| to JavaScript:

\begin{lstlisting}[
  style=ES6,
  caption=safeHead implmented in JavaScript,
  label={lst:js_safeHead}
]
const safeHead = list => {
  const head = list[0];
  return head ? Just(head) : Nothing;
};

const printHead = list => {
  const maybeHead = safeHead(list);
  maybeHead *'\label{line:js_safeHead1}'*
    (_ => console.log("List was empty!")) // Nothing case
    (head => conso.log(head));            // Just case
};

printHead([1,2,3,4]);
// => Logs '1'
printHead([]);
// => Logs 'List was empty!'
\end{lstlisting}

Using the structural similarity of |Just(value)| and |Nothing|,
lines~\ref{line:js_safeHead1} and following evaluate the result of |safeHead|.
% subsubsection Doing the same in JavaScript (end)
% subsection Wrapping values in a context (end)

\subsection{Working with Values in a Context} % (fold)
\label{sub:Working with values in a context}
\subsubsection{Introducing fmap} % (fold)
\label{subsub:Introducing fmap}
The example with |Maybe| shows how values work in a context. Over time,
however, it can become tedious to keep making this distinction between whether
it has value. Therefore, Haskell offers a concept of working with values in a
context. The simplest way to work with the value using this concept is using
the function |fmap|. |fmap| knows how to execute a function for the value(s) in
a context. Listing~\ref{lst:fmap_maybe_hs} shows the usage of |fmap| for
|Maybe|:

\begin{lstlisting}[
  style=Haskell,
  caption=fmap applied to Maybe,
  label={lst:fmap_maybe_hs}
]
double = \x -> 2 * x
print $ fmap double (Just 5)
-- Prints 'Just 10' 
print $ fmap double Nothing
-- Prints 'Nothing'
\end{lstlisting}

Similar to |Maybe| also lists describe such a context.
Listing~\ref{lst:hs_fmap_list} shows how |fmap| works on lists:

\begin{lstlisting}[
  style=Haskell,
  caption=fmap applied to a list,
  label={lst:hs_fmap_list}
]
print $ fmap double [1,2,3,4,5]
-- Prints '[2,4,6,8,10]'
\end{lstlisting}

Applying |fmap| to a list has the same result as applying |map| to a list.
Therefore, |fmap| is just a way to "map values".

% subsubsection Introducing fmap (end)
\subsubsection{Making a Context Monadic} % (fold)
\label{sec:Making a Context Monadic}
Haskell defines many other functions analogous to |fmap| to interact with
values in a context. So-called type classes define these functions. In his book
"Programming in Haskell" Hutton describes type-classes like
\textcquote[p.~31]{hutton_pih_2016}{[...], a class is a collection of types
that support certain overloaded operations called methods. }\\ 
Some of these functions are more powerful than |fmap|. The following section
briefly overviews which operations a context must provide to be considered
monadic:

\begin{itemize}
  \item \textbf{Functor}: A context |f| is a |Functor| when it provides a function
|fmap|. |fmap| receives another function as an argument and applies it to the
value(s) in the context.\\
|fmap| signature: |fmap :: Functor f => (a -> b) -> f a -> f b|\\

\item \textbf{Applicative}: A context |f| is an |Applicative| when it is a |Functor|
and additionally provides two functions:
\begin{itemize}
  \item |<*>| (pronounced "app") receives another function as an
    argument wrapped in the same context and applies the unwrapped function to
    the value(s) in the context. \\
    |<*>| signature |(<*>) :: Applicative f => f (a -> b) -> f a -> f b|
  \item |pure| receives a single argument and wraps it in the context. When
    describing |pure|, people often use the term lifting instead of wrapping.
    |pure| lifts a value into a context.\\
    |pure| signature: |pure :: Applicative f => a -> f a| \\
\end{itemize}
\item \textbf{Monad}: A context |f| is a |Monad| when it is an |Applicative| and
additionally provides a function |>>=| (pronounced "bind"). |>>=| takes another
function as an argument which, when applied to the value(s), again creates a
value in the same context. So that the value(s) are not nested in the context,
|>>=| resolves the inner context again. |>>=| is like |fmap| but after mapping,
the value(s) get flattened. Therefore, this operation is often also referred to
as |flatMap|.\\
|>>=| signature: |(>>=) :: Monad m => m a -> (a -> m b) -> m b|\\
\end{itemize}

This list should briefly overview this hierarchy of type classes. For a little
deeper introduction to this topic, see, for example, ~\cite{monads_adit_2013}.
To really dive into it and see the rules each function follows, please see
\cite[Chapter~12]{hutton_pih_2016}.

% subsubsection Making a Context Monadic (end)
\subsubsection{Why using these Abstractions ?} % (fold)
\label{sec:Why using these Abstractions?}
These abstractions allow building general functions that can handle various
types. Listing~\ref{lst:hs_why_use_constraints} shows a function |doubleAll|
applicable to |List|s and also |Maybe|s:

\begin{lstlisting}[
  style=Haskell,
  caption=Double the values in a context,
  label={lst:hs_why_use_constraints}
]
doubleAll :: (Functor f) => f Int -> f Int
doubleAll = fmap double

print $ doubleAll [1,2,3,4]
-- Prints '[2,4,6,8]'
print $ doubleAll $ Just 1
-- Prints 'Just 2'
\end{lstlisting}

Even though a |List| and a |Maybe| have nothing in common, the same function
can be applied to both types! \\ 
\textit{Note:} Another example that makes use of these abstractions is JINQ,
which is described in the Section~\ref{sec:Query different Data Structures}

% subsubsection Why using these constraints? (end)
% subsection Working with values in a context (end)

\subsection{Monads in JavaScript} % (fold)
\label{sub:Monads in JavaScript}
Since monads are so good at handling values in a context, this concept is also
interesting for JavaScript. However, JavaScript has a weaker type system than
Haskell. Therefore, two important concepts can not be transferred to
JavaScript: 
\begin{enumerate}
  \item In Haskell, it is possible to force the type hierarchy hinted in
    Section~\ref{sec:Making a Context Monadic}.
  \item Haskell can use its type system to determine a specific implementation
    for a function. The Listing~\ref{lst:hs_fn_body} shows the binding of the
    function body to the name using the example of |fmap|:
    \begin{lstlisting}[
      style=Haskell,
      caption=Haskell determines the correct function body,
      label={lst:hs_fn_body}
    ]
-- The implementation for List is used
print $ fmap double [1,2,3,4]
-- The implementation for Maybe is used
print $ fmap double (Just 1)
    \end{lstlisting}
\end{enumerate}


Since it is not possible to implement these two concepts in JavaScript, a
different approach to these two problems is needed:
\begin{enumerate}
  \item Although it is possible to model a type hierarchy in JavaScript,
    enforcing it is impossible. Therefore, we created only one type, the
    |MonadType|. |MonadType| can be used as an interface, which defines all
    operations a monadic type must support.  
  \item Instead of inferring global functions automatically, the prototype of a
    monadic object gives access to them. \cite{mdn_prototype_2023}
\end{enumerate}
Section~\ref{sub:Using JSDoc to type monads} describes these workarounds in
detail.

\subsubsection{Which Operations fit JavaScript?} % (fold)
\label{subsub:Which operations fit JavaScript?}
Since JavaScript works quite differently than Haskell, not all operations are
equally suitable to adopt. \\
The |MonadType| specifies the following operations:
\begin{itemize}
  \item |famp|: Changing values inside a context is a typical pattern,
    also in JavaScript. Porting |fmap| to JavaScript brings many benefits.
  \item |pure|: At first sight |pure| is a function that is not needed.
    However, it quickly became apparent that lifting an element into context is
    often practical via an abstracted function. 
    \\ This is often the case when using |>>=|.
  \item |>>=|: The |bind| operator allows access to the result of the last
    computation. |bind| is the only way to determine a new result depending on
    the previous result. \\ Since in JavaScript function names must not contain
    the special characters |>| and |=|, |>>=| cannot be used. |bind| is also
    already an existing function on every object. Using another term is
    required, therefore. We used the name |and| because it nicely expresses
    that the following result depends on the previous one.
  \item |empty|: The |empty| function creates a context without a value. For
    example, with |Maybe| it is |Nothing| with |List| it is |[]|. A monad does
    not need to provide a function |empty|, so Section~\ref{sec:Making a
    Context Monadic} does not include it. However, |empty| is a function
    available on many monads and is very handy during the programming of
    generic functions. Therefore, the |MonadType| specifies |empty| as well.

\end{itemize}
% subsubsection Which operations fit JavaScript? (end)
\subsubsection{What about the app Operator (<*>)?} % (fold)
\label{sec:Which operations do not fit JavaScript?}
The operator |<*>| is great when applying a function wrapped in a context to
multiple values wrapped in the same context. This use-case is relatively rare
in JavaScript because you do not operate as strictly in contexts as in Haskell.
In addition, the app operator also only works on functions that receive their
arguments curried. Most functions in JavaScript do not. Using the JavaScript
|Maybe| as an example, Listing~\ref{lst:app_js_maybe} shows how |<*>| would
work:

\begin{lstlisting}[
  style=Es6,
  caption=The app operator in JavaScript used on Maybe,
  label={lst:app_js_maybe}
]
const xs        = [1, 2, 3];    // array with x values
const ys        = [4, 5, 6];    // array with y values
const maybeX    = safeHead(xs); // getting the head of xs, defined by *'Listing~\ref{lst:js_safeHead}'*
const maybeY    = safeHead(ys); // getting the head of ys

const plus      = x => y => x + y; *'\label{line:app_js_maybe1}'*
const maybePlus = Just(plus); *'\label{line:app_js_maybe2}'*

const sum       = maybePlus.app(maybeX).app(maybeY); *'\label{line:app_js_maybe3}'*
sum 
  (_ => console.log("Something went wrong") // Nothing case
  (s => console.log(s));                    // Just case 
// => Logs '5'
\end{lstlisting}
Suppose you have two lists from which you want to add the first two elements.
Of course, these lists can be empty. |safeHead| fetches the first elements
safely! The function |plus| on line~\ref{line:app_js_maybe1} allows you to sum
up two numbers. However, these two addends are now in a context. So |plus|
cannot be applied directly to the values. That is when |app| comes into play!
|app| expects a parameter in the same context as the original wrapped function.
Line~\ref{line:app_js_maybe2}, therefore, wraps |plus| in |Just|. So it is
now also in the Maybe context. Line~\ref{line:app_js_maybe3} repeatedly calls
|app| on the wrapped function to pass parameter after parameter. In the end,
the variable |sum| contains the result of the addition! If one of the addends
was |Nothing| because one list was empty, |app| deals with that, and the whole
result evaluation to |Nothing|! \\ 
As said above, this only works because |plus| receives its arguments in a
curried style. Applying |plus| to |x| returns a function taking another
argument. Calling this function with another number returns the sum!

Even if this pattern appears less, it is still useful sometimes. Luckily |and|
capable of everything |app| can do. So consider Listing~\ref{lst:and_js_maybe}
which does the same thing using |and|, even though it is a bit more to type:

\begin{lstlisting}[
  style=ES6,
  caption=The app operator replaced by and,
  label={lst:and_js_maybe}
]
// maybeX, maybeY and maybePlus defined in Listing    *'\ref{lst:app_js_maybe}'*
const maybeSumX = maybePlus.and(p => maybeX.fmap(p));   // plus(x) *'\label{line:and_js_maybe1}'*
const maybeSum  = maybeSumX.and(pX => maybeY.fmap(pX)); // plus(x)(y)*'\label{line:and_js_maybe2}'*

maybeSum 
  (_ => console.log("Something went wrong") // Nothing case
  (s => console.log(s));                    // Just case
// => Logs '5'
\end{lstlisting}

Line~\ref{line:and_js_maybe1} shows how |and| gives access to the last element,
which is |p|, the function |plus|! |fmap| applies this function to |maybeX|.
This creates a new funciton which is again in the |Maybe| context.\\
Line~\ref{line:and_js_maybe2} does exactly the same - but for the second
argument. \\ 
This shows that the behavior of |app| can be emulated using |and|.
% subsubsection Which operations do not fit JavaScript? (end)

% subsection Monads in JavaScript (end)
\subsection{Using JSDoc to type Monads} % (fold)
\label{sub:Using JSDoc to type monads}
JSDoc is the optional type system of JavaScript. ~\cite{jsdoc_use_2023}  As the
name says, it is pure documentation, therefore, not enforceable.

\subsubsection{The MonadType} % (fold)
\label{subsub:The MonadType}
The |Monadtype| combines all the operations described in
Section~\ref{subsub:Which operations fit JavaScript?} into a single JSDoc type.
It serves as a structural interface that functions can use to describe their
arguments or return values.

\begin{lstlisting}[
  style=ES6,
  caption=The MonadType,
  label={lst:monad_type_js}
]
/**
 * Defines a Monad.
 * @template  _T_
 * @typedef  MonadType
 * @property { <_U_> (bindFn: (_T_) => MonadType<_U_>) => MonadType<_U_> } and
 * @property { <_U_> (f:      (_T_) => _U_)            => MonadType<_U_> } fmap
 * @property {       (_T_)                             => MonadType<_T_> } pure
 * @property {       ()                                => MonadType<_T_> } empty
 */
\end{lstlisting}

Listing~\ref{lst:monad_type_js} shows this type with the four available
operations |and|, |fmap|, |pure| and |empty|. |And| as well as |fmap| \textit{do}
something with the current value(s) wrapped in this context. |Pure| and |empty|
work more like static operations. They are accessible using the object but have
no direct connection to its properties. 

Listing~\ref{lst:monad_type_usage_js} defines a function |keepEven| which
discards odd values in the context. As parameter, this function accepts every context
adhering to |MonadType|. Therefore, |keepEven| can access all of
its functions. This way JavaScript knows which implementation to call:

\begin{lstlisting}[
  style=ES6,
  caption=Usage of the MonadType,
  label={lst:monad_type_usage_js}
]

/**
 * @param { MonadType<Number> } monad
 * @returns MonadType<Number>
 */
const keepEven = monad => 
  monad
   .and(x => {
     if (x % 2 === 0) return monad.pure(x);
     else return monad.empty();
   }); 
\end{lstlisting}

Listing~\ref{lst:keep_even_maybe} shows how |keepEven| can be called with a
|Maybe|. If the |Maybe| contains an odd number, |keepEven| discards it. The
best aspect is that |Nothing| does not need any special treatment - the
|and| implementation of |Maybe| knows what to do when a |Nothing| appears. So
it is not the job of |keepEven| to handle such special cases!

\begin{lstlisting}[
  style=ES6,
  caption=keepEven called with a Maybe,
  label={lst:keep_even_maybe}
]
/** Prints the value of a Maybe if it exists */
const evalMaybe = maybe =>
  maybe
  (_ => console.log("There was no value!"))
  (x => console.log(x));

const maybe1 = Just(1);
const maybe2 = Just(2);
const maybe3 = Nothing;

evalMaybe(keepEven(maybe1));
// => Logs 'There was no value!'
evalMaybe(keepEven(maybe2));
// => Logs '2'
evalMaybe(keepEven(maybe3));
// => Logs 'There was no value!'
\end{lstlisting}

The same works for a Sequence - |keepEven| discards every odd number of this
sequence:

\begin{lstlisting}[
  style=ES6,
  caption=KeepEven called with a Sequence,
  label={lst:keep_even_iterable}
]
const seq   = Sequence(0, i => i < 5, i => i + 1);

console.log(...keepEven(seq));
// => Logs '0, 2, 4'
\end{lstlisting}
The example shows well how one can program very declaratively using monads. You
do not have to worry about the actual structure of the context, but only about
the logic being correct!
|keepEven| can also be tested easily: you create a new type that conforms to
|MonadType|, which has little logic.

% subsubsection The MonadType (end)
% subsection Representing monads as a type using JSDOC (end)
\subsection{Implementing Monadic Operations in JavaScript} % (fold)
\label{sub:Implementing Monadic Operations in JavaScript}
This section covers how to make a context conform |MonadType| discussed in
Section~\ref{subsub:The MonadType}.\\ 
This covers following steps:
\begin{enumerate}
  \item Creating a prototype for the context (see Section~\ref{subsub:Sequence
    Prototype} which explains prototypes)
  \item Defining the four operations specified by |MonadType| on this prototype
  \item Specify the JSDoc of this context
\end{enumerate}

\textit{Note:} Defining the monadic operations on a prototype is mostly a good
idea, since there are often multiple implementations of a type, like it is in
Maybe with |Just| and |Nothing|. This way it becomes possible to share the
implemenations among them.
\subsubsection{Defining the Monadic Operations} % (fold)
\label{sec:Defining the Monadic Operations}
Listing~\ref{lst:monadic_maybe_js} uses the example of |Maybe| to show how to
implement the monadic operations |fmap|, |pure|, |empty|, and |and|.

\begin{lstlisting}[
  style=ES6,
  caption=Making Maybe monadic,
  label={lst:monadic_maybe_js}
]
const MaybePrototype = () => undefined;

MaybePrototype.pure = val => Just(val); *'\label{line:monadic_maybe_js1}'*

MaybePrototype.empty = () => Nothing; *'\label{line:monadic_maybe_js2}'*

MaybePrototype.and = function (bindFn) { *'\label{line:monadic_maybe_js3}'*
  let returnVal;
  this
    (_ => returnVal = Nothing)
    (x => returnVal = bindFn(x));
  return returnVal;
};

MaybePrototype.fmap = function (mapper) { *'\label{line:monadic_maybe_js4}'*
  return this.and(x => Just(mapper(x)));
};
\end{lstlisting}
This code does the following:
\begin{itemize}
  \item Line~\ref{line:monadic_maybe_js1} defines |pure| - The given value is turned
    into a |Maybe|
  \item Line~\ref{line:monadic_maybe_js2} defines |empty| - No value is in this
    context
  \item Line~\ref{line:monadic_maybe_js3} and following define |and| - If a
    value is present, |and| applies the function |bindFn| to this value and
    returns the result. If there is no value, |and| returns Nothing directly.
  \item Line~\ref{line:monadic_maybe_js4} and following define |fmap| - |and|
    is used to transform the current value. Since |bindFn| needs to return a
    |Maybe| again, the result of |mapper| gets wrapped with |Just|.
\end{itemize}

Since both |Just(value)| and |Nothing| are functions, they must provide the respective
operations as properties. In JavaScript, where functions are objects, you can
define their properties and modify or override existing ones. \\
Listing~\ref{lst:proto_maybe_js} now sets the prototype to the two functions
|Just(value)| and |Nothing|.

\begin{lstlisting}[
  style=ES6,
  caption=Setting the prototype of Maybe,
  label={lst:proto_maybe_js}
]
const Nothing = f => _g => f(undefined);*'\label{line:proto_maybe_js1}'*
Object.setPrototypeOf(Nothing, MaybePrototype);

const Just = x => { *'\label{line:proto_maybe_js2}'*
  const just = _f => g => g(x);
  Object.setPrototypeOf(just, MaybePrototype);
  return just;
};*'\label{line:proto_maybe_js3}'*
\end{lstlisting}

The definition of |Nothing| on line~\ref{line:proto_maybe_js1} stays exactly
the same as before. But consider line~\ref{line:proto_maybe_js2} to
\ref{line:proto_maybe_js3}. These changed, because not |Just| should become a
|MonadType| but |Just(value)|!

To make another object monadic, apply the same pattern to it!


% subsubsection Defining the Monadic Operations (end)

\subsubsection{Specifying the JSDoc} % (fold)
\label{sec:Specify the JSDoc}
JSDoc is structurally typed. We exploit this to extend existing types with
the monadic type. Such complemented types are then compatible with functions
that expect |MonadType|s as parameters. \\ 
Listing~\ref{lst:jsdoc_add_maybe} adds the |MonadType| JSDoc to Maybe:

\begin{lstlisting}[
  style=ES6,
  caption=Adding JSDoc to Maybe,
  label={lst:jsdoc_add_maybe}
]
/** *'\label{line:jsdoc_add_maybe1}'*
 * @template _T_
 * @typedef { 
 *              ((f:<omitted>)  => (g:<omitted>) => _T_) *'\label{line:jsdoc_add_maybe2}'*
              & *'\colorbox{code-highlight}{MaybeMonadType<\_T\_>} \label{line:jsdoc_add_maybe3}'*
            } MaybeType
 */

/** 
 * @typedef MaybeMonadType
 * @template _T_
 * @property { <_V_> ((_T_) => MaybeType<_V_>) => MaybeType<_V_> } and
 * @property { <_V_> ((_T_) => _V_)            => MaybeType<_V_> } fmap
 * @property { <_V_> (_V_)                     => MaybeType<_V_> } pure
 * @property {       ()                        => MaybeType<_T_> } empty
 */
\end{lstlisting}

What happens here? \\ 
Line~\ref{line:jsdoc_add_maybe1} and following specify the updated |MaybeType|.
|MaybeType| is a function, which receives two other functions as parameters
(line~\ref{line:jsdoc_add_maybe2}). And it is also everything that is specified by
|MaybeMonadType| (line~\ref{line:jsdoc_add_maybe3})! \\
This |&| sign allows intersecting two JSDoc type definitions into a new one.\\

Using such an intersection type brings the advantage of splitting up type
declarations, which makes it easier to extend types and also to reuse specific
parts of a type definition.

% subsubsection Specify the JSDoc (end)
\subsubsection{Limitations of JSDoc} % (fold)
\label{sec:Limitations of JSDoc}
You may have recognized that |MaybeMonadType| and |MonadType| look almost
identical. So why not drop the specification for |MaybeMonadType| and directly
use |MonadType|? The reason is simple: |MonadType| is less specific than
|MaybeMonadType| - every operation on |MonadType| returns a |MonadType|
again.\\
Consider line~\ref{line:monad_type_maybe_directly1} in the following
Listing~\ref{lst:monad_type_maybe_directly}, where the |MaybeType| is defined
using the more generic |MonadType| (defined in Listing~\ref{lst:monad_type_js})
instead of the |MaybeMonadType| (defined in Listing~\ref{lst:jsdoc_add_maybe}):

\begin{lstlisting}[
  style=ES6,
  caption=Why not use MonadType with Maybe?,
  label={lst:monad_type_maybe_directly}
]
/** *'\label{line:jsdoc_maybe1}'*
 * @template _T_
 * @typedef { 
 *              ((f:<omitted>)  => (g:<omitted>) => _T_)
              & *'\colorbox{code-highlight}{MonadType<\_T\_>}  \label{line:monad_type_maybe_directly1}'*
            } MaybeType
 */
const maybe        = Just(5); // MaybeType
const mappedMaybe  = maybe.fmap(x => 2*x); // MonadType *'\label{line:monad_type_maybe_directly2}'*

// produces a type error
evalMaybe(mappedMaybe); // *'\label{line:monad_type_maybe_directly3} Defined in Listing~\ref{lst:keep_even_maybe}'*

const mappedMaybe2 = /** @type { MaybeType } */ maybe.fmap(x => 2*x); *'\label{line:monad_type_maybe_directly4}'*
evalMaybe(mappedMaybe); // *'Defined in Listing~\ref{lst:keep_even_maybe}'*
\end{lstlisting}

Since |fmap| is defined by |MonadType|
line~\ref{line:monad_type_maybe_directly2} loses the information that |maybe|
was of |MaybeType|. Therefore, line~\ref{line:monad_type_maybe_directly3} does
not accept |mappedMaybe| anymore since |evalMaybe| requires a |MaybeType| and
not a |MonadType|. The workaround would be to add a typecast every time any
monadic function has been called, like on
line~\ref{line:monad_type_maybe_directly4}.

For this to work without type casting, JSDoc would have to support higher
kinded types \cite{baeldung_higher-kinded_2020}, which would allow type definitions as defined by
Listing~\ref{lst:not_working_type}:

\begin{lstlisting}[
  style=ES6,
  caption=Not working JSDoc types,
  label={lst:not_working_type}
]
  /**
   * Defines a Monad.
   * @typedef   MonadType
   * @template  *'\colorbox{code-highlight}{\{ MonadType \} \_M\_} \label{line:def_template_constraint}'*
   * @template  _T_
   * ...
   * <other functions omitted>
   * ...
   * @property  { <_U_>
   *                 ((_T_) => <_U_>)
   *              => *'\colorbox{code-highlight}{\_M\_<\_U\_>} \label{line:use_template_constraint}'*
   *            } fmap
   */
\end{lstlisting}

Line~\ref{line:def_template_constraint} defines a generic type variable that
has to conform to |MonadType|. So far, everything works with JSDoc.\\ But
line~\ref{line:use_template_constraint} does not work anymore because it is not
possible to abstract over another type using JSDoc.\\
% subsubsection Limitations of JSDoc (end)
% subsection Implementing Monadic Operaions in JavaScript (end)

\subsection{Making the Sequence Monadic} % (fold)
\label{sub:Making the Sequence Monadic}
Section~\ref{sub:Using JSDoc to type monads} defines the function |keepEven|
which accepts any |MonadType| as an argument. Since the |Sequence| is monadic,
Listing~\ref{lst:keep_even_iterable} applies |keepEven| to a |Sequence| without
defining how to implement the monadic operations on it. The following section
shows the implementation of the monadic operaitons on the |Sequence|.\\
The signatures of the operations are analogous to those of |Maybe|. However,
they do not work with |MaybeMonadType| but with |SequenceMonadType|. Again, the
prototype defines these functions. |Object.setPrototypeOf| then adds it to the
sequence.

Listing~\ref{lst:monadic_ops_sequence} shows the definition of these
operations - it is straight forward:
\begin{lstlisting}[
  style=ES6,
  caption=The monadic operations on the Sequence,
  label={lst:monadic_ops_sequence}
]
/**
 * @template _T_
 * @returns SequenceType<_T_>
 */
SequencePrototype.empty = () => Sequence(undefined, _ => false, _ => undefined); *'\label{line:monadic_empty_sequence}'*

/**
 * @template _T_
 * @param { _T_ } val - lifts a given value into the context
 * @returns SequenceType<_T_>
 */
SequencePrototype.pure = val => PureSequence(val); *'\label{line:monadic_pure_sequence}'*

/**
 * @template _T_, _U_
 * @param { (_T_) => _U_ } mapper - maps the value in the context
 * @returns SequenceType<_U_>
 */
SequencePrototype.fmap = function (mapper) {*'\label{line:monadic_fmap_sequence}'*
  return map(mapper)(this); 
};

/**
 * @template _T_, _U_
 * @param { (_T_) => SequenceType<_U_> } bindFn
 * @returns { SequenceType<_U_> }
 */
SequencePrototype.and = function (bindFn) {*'\label{line:monadic_and_sequence}'*
  return bind(bindFn)(this); 
};

\end{lstlisting}

\begin{itemize}
  \item |empty| (line~\ref{line:monadic_empty_sequence}): An empty |Sequence|
    contains no values and returns, therefore immediately |done: true| when
    calling |next| on its iterator.
  \item |pure| (line~\ref{line:monadic_pure_sequence}): A |Sequence| with one
    single value iterates only once, returning this value. |PureSequence| does
    precisely this.
  \item |fmap| (line~\ref{line:monadic_fmap_sequence}): Mapping a |Sequence|
    applies the given function to each element. Which is what |map| does.
  \item |and| (line~\ref{line:monadic_and_sequence}): Bind on a |Sequence|
    applies the given function to each element and then flattens it. This is
    what |bind| does.
\end{itemize}


% subsection Making the Sequence Monadic (end)
