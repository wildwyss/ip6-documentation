\section{Monads in JavaScript} % (fold)
\label{sec:Monads in JavaScript}

\subsection{Wrapping values in a context} % (fold)
\label{sub:Wrapping values in a context}

In Haskell, you often work with values wrapped in a particular context. This
context can be, for example, a list. Nevertheless, this context can also be
another data structure, for example, |Maybe|. \\
For the context |Maybe| two implementations in Haskell co-exist:

% section Monads in JavaScript (end)
\begin{lstlisting}[
  style=Haskell,
  caption=The data type Maybe in Haskell,
  label={lst:maybe_hs}
]
-- defining the datatype Maybe
data Maybe a = Nothing | Just a
\end{lstlisting}

The Listing~\ref{lst:maybe_hs} defines this datatype. Why is this useful?
Imagine you want to create a function |head|, which returns the first value of
a given list. |head| is pretty simple to implement. But wait, what to do when
the list is empty? In object-oriented languages, one might return |null|. The
problem with this solution is that the user must remember that the list can be
empty, and thus the result of |head| can be |null|. This is very error-prone.
\\

That is where the new datatype |Maybe| comes in. |Maybe| allows us to describe
either the absence of a value or the value itself. The following
Listing~\ref{lst:safe_head} defines a new function |safeHead|
(line~\ref{line:safe_head1}), which does
precisely this - it returns |Just a| when the list is not empty
(line~\ref{line:safe_head2}) and |Nothing| otherwise (line~\ref{line:safe_head3}):

\begin{lstlisting}[
  style=Haskell,
  caption=Safely get the first element of a list,
  label={lst:safe_head}
]
-- a function which produces a Maybe:
safeHead :: [a] -> Maybe a *'\label{line:safe_head1}'*
safeHead (a:_) = Just a *'\label{line:safe_head2}'*
safeHead []    = Nothing *'\label{line:safe_head3}'*

printHead list = print $ case (safeHead list) of 
  Just val -> show val
  Nothing  -> "List was empty!"

printHead [1,2,3,4]
-- Prints '"1"'
printHead []
-- Prints '"List was empty!"'
\end{lstlisting}

The difference with returning just |null| is that the user now has to
explicitly deal with the case where there is no first element. This leads to
improved safety.
\subsubsection{Doing the same in JavaScript} % (fold)
\label{subsub:Doing the same in JavaScript}
Listing~\ref{lst:maybe_js} defines the same type in JavaScript:
\begin{lstlisting}[
  style=ES6,
  caption=The data type Maybe in JavaScript,
  label={lst:maybe_js}
]
const Just    = x => _f => g => g(x);*'\label{line:maybe_js1}'*
const Nothing = f => _g => f(undefined);*'\label{line:maybe_js2}'*
\end{lstlisting}

So |Just| and |Nothing| are just functions! How could it be different in
functional programming? \\ 
Line~\ref{line:maybe_js1} defines the |Just| case:
|Just| takes a value |x| and two functions while not using the first function
at all. |Just| applies the second function to the initial value |x|. \\ 
|Nothing| looks very similar to |Just|, with the difference that is does not
receive a value, and the first and not the second function will be called.\\ 

Now how can this be used? Noticeably, |Just(value)| and |Nothing| have the same
structure: both receive two functions as arguments. That means they are
structurally the same. Listing~\ref{lst:js_safeHead} takes advantage of this
property to port the previous function |safeHead| to JavaScript:

\begin{lstlisting}[
  style=ES6,
  caption=safeHead implmented in JavaScript,
  label={lst:js_safeHead}
]
const safeHead = list => {
  const head = list[0];
  return head ? Just(head) : Nothing;
};

const printHead = list => {
  const maybeHead = safeHead(list);
  maybeHead *'\label{line:js_safeHead1}'*
    (_ => console.log("List was empty!")) // Nothing case
    (head => conso.log(head));            // Just case
};

printHead([1,2,3,4]);
// => Logs '1'
printHead([]);
// => Logs 'List was empty!'
\end{lstlisting}

Using the structural similarity of |Just(value)| and |Nothing|,
lines~\ref{line:js_safeHead1} and following evaluate the result of |safeHead|.
% subsubsection Doing the same in JavaScript (end)
% subsection Wrapping values in a context (end)

\subsection{Working with values in a context} % (fold)
\label{sub:Working with values in a context}
\subsubsection{Introducing fmap} % (fold)
\label{subsub:Introducing fmap}
The example with |Maybe| shows how values work in a context. Over time,
however, it can become tedious to keep making this distinction between whether
it has value. Therefore, Haskell offers a concept of working with values in a
context. The simplest way to work with the value using this concept is using
the function |fmap|. |fmap| knows how to execute a function for the value(s) in
a context. Listing~\ref{lst:fmap_maybe_hs} shows the usage of |fmap| for
|Maybe|:

\begin{lstlisting}[
  style=Haskell,
  caption=fmap applied to Maybe,
  label={lst:fmap_maybe_hs}
]
double = (\x -> 2*x)
print $ fmap double (Just 5)
-- Prints 'Just 10' 
print $ fmap double Nothing
--> Prints 'Nothing'
\end{lstlisting}

Similar to |Maybe| also lists describe such a context.
Listing~\ref{lst:hs_fmap_list} shows how |fmap| works on lists:

\begin{lstlisting}[
  style=Haskell,
  caption=fmap applied to a list,
  label={lst:hs_fmap_list}
]
print $ fmap double [1,2,3,4,5]
--> Prints '[2,4,6,8,10]'
\end{lstlisting}

Applying |fmap| to a list has the same result as applying |map| to a list.
Therefore, |fmap| is just a way to "map values".

% subsubsection Introducing fmap (end)
\subsubsection{Making a Context Monadic} % (fold)
\label{sec:Making a Context Monadic}
Haskell defines many other functions analogous to |fmap| to interact with
values in a context. Some of them are more powerful than |fmap|. The following
section gives a brief overview of which operations exist on monads. For a
detailed introduction to this topic, see, for example, ~\cite{monads_adit_2013}. \\

If the context supports the following operations, it is considered monadic:

\begin{itemize}
  \item \textbf{Functor}: A context |f| is a |Functor| when it provides a function
|fmap|. |fmap| receives another function as an argument and applies it to the
value(s) in the context.\\
\item \textbf{Applicative}: A context |f| is an |Applicative| when it is a |Functor|
and additionally provides two functions:
\begin{itemize}
  \item |<*>| (pronounced "app") receives another function as an
    argument wrapped in the same context and applies the unwrapped function to
    the value(s) in the context.
  \item |pure| receives a single argument and wraps it in the context. When
    describing |pure|, people often use the term lifting instead of wrapping.
    |pure| lifts a value into a context.\\
 \\
\end{itemize}
\item \textbf{Monad}: A context |f| is a |Monad| when it is an |Applicative| and
additionally provides a function |>>=| (pronounced "bind"). |>>=| takes another
function as an argument which, when applied to the value(s), again creates a
value in the same context. So that the value(s) are not nested in the context,
|>>=| resolves the inner context again. |>>=| is like |fmap| but after mapping,
the value(s) get flattened. Therefore, this operation is often also referred to
as |flatMap|.
\end{itemize}

Some of these functions have special rules they must follow. Chapter "TODO link
chapter" discusses these rules using their corresponding JavaScript
implementation.
% subsubsection Making a Context Monadic (end)
\subsubsection{Why using these constraints?} % (fold)
\label{sec:Why using these constraints?}
These constraints allow building general functions that can handle various
types. Listing~\ref{lst:hs_why_use_constraints} shows a function |doubleAll|
applicable to |List|s and also |Maybe|s:

\begin{lstlisting}[
  style=Haskell,
  caption=Double the values in a context,
  label={lst:hs_why_use_constraints}
]
doubleAll :: (Functor f) => f Int -> f Int
doubleAll = fmap double

print $ doubleAll [1,2,3,4]
-- Prints '[2,4,6,8]'
print $ doubleAll $ Just 1
-- Prints 'Just 2'
\end{lstlisting}

Even though a |List| and a |Maybe| have nothing in common, the same function
can be applied to both types! \\ 
\textit{Note:} Another example that makes use of this is JINQ, which is described in the
section "TODO link section here".

% subsubsection Why using these constraints? (end)
% subsection Working with values in a context (end)

\subsection{Monads in JavaScript} % (fold)
\label{sub:Monads in JavaScript}
Since monads are so good at handling values in a context, this concept is also
interesting for JavaScript. However, JavaScript has a weaker type system than
Haskell. Therefore, two important concepts can not be transferred to
JavaScript: 
\begin{enumerate}
  \item In Haskell, it is possible, to force the type hierarchy hinted in
    Section~\ref{sec:Making a Context Monadic}.
  \item Haskell can use its type system to determine a specific implementation
    for a function. The Listing~\ref{lst:hs_fn_body} shows the binding of the
    function body to the name using the example of |fmap|:
    \begin{lstlisting}[
      style=Haskell,
      caption=Haskell determines the correct function body,
      label={lst:hs_fn_body}
    ]
-- The implementation for List is used
print $ fmap double [1,2,3,4]
-- The implementation for Maybe is used
print $ fmap double (Just 1)
    \end{lstlisting}
\end{enumerate}


Since it is not possible to implement these two concepts in JavaScript, a
different approach to these two problems is needed:
\begin{enumerate}
  \item Although it is possible to model a type hierarchy in JavaScript,
    enforcing it is impossible. Therefore, we created only one type, the
    |MonadType|. |MonadType| can be used as an interface, which defines all
    operations a monadic type must support.
  \item Instead of type inference and global functions, we added all operations
    to the corresponding object's prototype. \cite{mdn_prototype_2023}
\end{enumerate}

\subsubsection{Which operations fit JavaScript?} % (fold)
\label{Which operations fit JavaScript?}
Since JavaScript works quite differently than Haskell, not all operations are
equally suitable to adopt. \\
The |MonadType| specifies the following operations:
\begin{itemize}
  \item |famp|: Changing values inside a context is a typical pattern,
    also in JavaScript. Porting |fmap| to JavaScript brings many benefits,
    therefore.
  \item |pure|: At first sight |pure| is a function that is not needed.
    However, it quickly became apparent during use that it is often practical
    to lift an element into context via an abstracted function. \\ This is
    often the case when using |>>|.
  \item |>>=|: The |bind| operator allows access to the result of the last
    computation. |Bind| is the only way to determine a new result depending on
    the previous result. \\ Since in JavaScript function names must not contain
    the special characters |>| and |=|, |>>=| cannot be used. |Bind| is also
    already an existing function on every object. Using another term is
    required, therefore. We used the name |and| because it nicely expresses
    that the following result depends on the previous one.
  \item |empty|: The |empty| function creates a context without a value. For
    example, with |Maybe| it is |Nothing| with |List| it is |[]|. A monad does
    not need to provide a function |empty|, so Section~\ref{sec:Making a
    Context Monadic} does not include it. However, |empty| is a function
    available on many monads and is very handy during the programming of
    generic functions. Therefore, the |MonadType| specifies |empty| as well.

\end{itemize}
% subsubsection Which operations fit JavaScript? (end)
\subsubsection{What about the app operator?} % (fold)
\label{sec:Which operations do not fit JavaScript?}
The operator |<*>| is great when applying a function wrapped in a context to
values wrapped in a context. This use-case is relatively rare in JavaScript
because you do not operate as strictly in contexts as in Haskell. In addition,
the app operator also only works on functions that accept their arguments
curried. Most functions in JavaScript do not. Using the JavaScript |Maybe| as
an example, Listing~\ref{lst:app_js_maybe} shows how |<*>| would work:

\begin{lstlisting}[
  style=Es6,
  caption=The app operator in JavaScript used on Maybe,
  label={lst:app_js_maybe}
]
const x = 1, y = 0;
const plus = x => y => x + y; *'\label{line:app_js_maybe1}'*
const jPlus = Just(plus); 
const sum = jPlus.app(Just(x)).app(Just(y)); *'\label{line:app_js_maybe2}'*

sum 
  (_ => console.log("Something went wrong") // Nothing case
  (s => console.log(s));                    // Just case 
// => Logs '1'
\end{lstlisting}

Line~\ref{line:app_js_maybe1} specifies the curried function |plus|. This
function is lifted into the |Maybe| context. Then |<*>| applies the function
|plus| to summand by summand on line~\ref{line:app_js_maybe2}.\\
As this example shows, the usefulness of |<*>| is relatively limited in
JavaScript because functions rarely occur in context. Additionally, |>>=| can
solve everything solvable with |<*>|. \\ 
Listing~\ref{lst:and_js_maybe} shows how |>>| solves the same problem:

\begin{lstlisting}[
  style=ES6,
  caption=The app operator replaced by and,
  label={lst:and_js_maybe}
]
const x = 1, y = 0;
const plus = x => y => x + y; 
const jPlus = Just(plus); 
const sum = jPlus.and(f => f(x)).and(g => g(y)); *'\label{line:and_js_maybe1}'*

sum 
  (_ => console.log("Something went wrong") // Nothing case
  (s => console.log(s));                    // Just case
// => Logs '1'
\end{lstlisting}
Line~\ref{line:and_js_maybe1} shows how |and| gives access to the last element.
Which is the function |plus| the first time and then the function |plus(x)|.

% subsubsection Which operations do not fit JavaScript? (end)
% subsection Monads in JavaScript (end)
