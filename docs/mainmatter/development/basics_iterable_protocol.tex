\section{Sequence and Iterable in General}
\label{sec:Sequence and Iterable in General}
In this section, we explain the concept of iterable objects and show how to 
create them. Sequence is one such implementation. It illustrates how the 
benefits of iterators are utilized in a specific implementation using 
illustrations and explanations.

\subsection{What is an Iterator?}
\label{sub:What is an Iterator?}
Iterators are a common notion in computer science. An iterator is essentially 
a mechanism to access elements of any data structure. An iterator uses a 
function call to return the next element. The iterator either acts on a data 
structure in memory or computes the elements when they are queried. Iterators 
are an essential part of this work. As a result, the fundamentals are essential 
to understand and will be covered in greater depth in the next sections.

% \subsection{Iteration Protocols in JavaScript}
\subsection{Exploring the Meaning Of Protocols}
\label{sub:Exploring the Meaning Of Protocols}
An iterator in JavaScript is an object that is implemented in compliance with
the JavaScript iterator and iterable protocols \cite{mdn_iteration_2023}
(henceforth called JS iteration protocols). An iterator is characterized by the
fact that it can only be iterated once over a set of elements. After that the
iterator is used up and further access works on a new instance of it.


\subsubsection{The Iterable Protocol}
\label{subsub:The Iterable Protocol}
An object is an iterable according to the JS iterable protocol if it has a
property with the name |[Symbol.iterator]|. Thus, such an object is of type
|Iterable<T>| which defines its iteration behavior. All objects in JavaScript
that have this property can be processed using destructuring and |for...of|
loop. The sample code for defining a property named |[Symbol.iterator]| is shown
in Listing \ref{lst:iterable_protocols}. The value of this property is a
function which has to follow the iterator protocol.

\begin{lstlisting}[
  style=ES6, 
  caption=Iterable protocol,
  label={lst:iterable_protocols}
  ]
  return {
    [Symbol.iterator]: () => {
      return { next: next };
    }
  }
\end{lstlisting}

\subsubsection{The Iterator Protocol}
\label{subsub:The Iterator Protocol}
This protocol defines the structure of the iterator object. This object is
obtained when calling the |[Symbol.iterator]| property. The object must implement a
function next, which defines how and which values are returned when iterating.
Each iteration calls the function |next|. Next returns an object, which must
have two properties according to the iterator protocol: |value| and |done|.
Therefore, the property |value| contains the current value of the iteration,
while |done| represents the information on whether the end of the iteration has
been reached. The following code \ref{lst:iterator_protocol} shows a simple example implementation. 
Each call on |next| returns an object with the |value| 1 and the iteration is
never finished. 

\begin{lstlisting}[
  style=ES6, caption=Iterator protocol,
  label={lst:iterator_protocol}
  ]
  const next = () => {
    return { done: false, value: 1 };
  };
\end{lstlisting}

\subsubsection{Creating Iterables}
\label{subsub:Creating Iterables}
By combining the two protocols, the result could look like listing
\ref{lst:protocols}. in listing \ref{lst:protocols} the constructor 
|SampleIterable| on line 1 wraps the two previously
defined implementations of the protocols. Objects from this constructor are now
iterables. It has the type |Iterable<Number>| because the return value from the
iterator is one.

\begin{lstlisting}[
  style=ES6, caption=Iterable and iterator protocol,
  label={lst:protocols}
  ]
 const SampleIterable = () => {

  const next = () => {
    return { done: false, value: 1 };
  };

  return {
    [Symbol.iterator]: () => {
      return { next: next };
    }
  }
};
\end{lstlisting}

Since |SampleIterator| follows the JS iterator protocols it is now possible to
construct iterable objects.
Built-in language features like |for..of| and destructuring (|...|) can now process these objects. 
But beware, in this case, this would lead to an infinite loop because the 
property done never becomes true. We will see examples with non-endless iterators 
later in this. But for now, the focus here is on the protocols, therefore this 
example will be sufficient at this point.
\newline
Such protocols make it possible to iterate your customized objects and
collections. This opens new possibilities. Various programming tasks can be 
solved in a different, probably simpler way.
\newline
There are already some language built-in iterable in JavaScript. Array and
HTML Collections are probably the most prominent of these.

\subsection{Types of Iterables}
\label{sub:Types of Iterables}
Previously we saw that there is a distinction between iterables~\ref{subsub:Creating Iterables} 
and iterator~\ref{subsub:The Iterator Protocol}. These abstractions also have their own types. 
Listing \ref{lst:iterable_types}  shows an excerpt of the relevant types.

\begin{lstlisting}[
  style=ES6, caption=Types of iterables,
  label={lst:iterable_types}
  ]
// lib.es2015.iterable.d.ts

interface Iterable<T> {
    [Symbol.iterator](): Iterator<T>;
}

interface Iterator<T, TReturn = any, TNext = undefined> {
    next(...args: [] | [TNext]): IteratorResult<T, TReturn>;
    return?(value?: TReturn): IteratorResult<T, TReturn>;
    throw?(e?: any): IteratorResult<T, TReturn>;
}

type IteratorResult<T, TReturn = any> = IteratorYieldResult<T> 
                                      | IteratorReturnResult<TReturn>;

interface IteratorReturnResult<TReturn> {
    done: true;
    value: TReturn;
}
\end{lstlisting}

An iterable is of type |Iterable<T>|, whereas the object returned by the property
|[Symbol.Iterator]| is of type |Itertator<...>|. As shown in listing
\ref{lst:iterable_types}, an iterator is required to have a property next. 
This is the function that returns values when iterating. These values must be 
of type |IteratorResult<...>|. Defined on line 8 of listing 
\ref{lst:iterable_types}. |IteratorResult<...>| itself is defined to return an 
object of type |ItereratorReturnResult<...>|. This object contains the actual 
values we want to work with. Examining these types illustrates how JavaScript's 
iterators have been built and reveal that some objects are nested within each
other. There is a reason for this. We will get to the roots of these later.

\subsection{Building custom Series of Data}
\label{sub:Building custom Series of Data}
The last chapter has shown how to make an object iterable. This chapter 
demonstrates how powerful concepts can be created using the protocols 
discussed so far. Previously, an endless iterator that generates only ones 
served as an illustration. Now, the goal is to create a constructor that can 
generate arbitrary series of data.

\subsection{Sequence - The Naming of connected Data}
\label{sub:Sequence - The Naming of connected Data}
As is often the case in computer science, one of the hardest things is finding 
the right names. Various arguments have contributed to the fact that the 
constructor for creating any series is called "Sequence". First, it should be 
clear that sequences are not conventional lists known from other programming 
languages. However, it must be clear from the name that it is about connected 
data. The nice thing about the objects generated by Sequence is that one has 
the feeling of handling arbitrarily large amounts of data. In reality, however, 
only a minimum of memory is needed, because the subsequent elements are not yet 
in memory.

\subsection{Everything a Sequence needs}
\label{sub:Everything a Sequence needs}
To define a sequence of data, three essential elements must be specified. 
Firstly, a fixed starting value for the sequence is required. Secondly, a 
function must be defined to calculate the next element based on its predecessor. 
Lastly, a function is needed to determine if the sequence has reached its end.
These three elements basically define the sequence and are passed as arguments, 
see listing \ref{lst:sequence} line 2.
To keep the focus on the core elements of the Sequence, some functionality in 
listing \ref{lst:sequence} is discussed later. The Next function, explained in 
the Iterator Protocol section, is located on lines 12 -17. It contains the 
logic to return the next object of an iteration. First, it uses |untilFunction| 
to check if the sequence has finished. If this is not the case, the next value 
is calculated and returned with the |incrementFunction|.

\begin{lstlisting}[
  style=ES6, caption=Parts of Sequence,
  label={lst:sequence}
  ]
// Sequence.js
const Sequence = (start, untilFunction, incrementFunction) => {

  const iterator = () => {
    let value = start;
    /**
     * @template _T_
     * Returns the next iteration of this iterable object.
     * @returns { IteratorResult<_T_, _T_> }
     */
    const next = () => {
      const current = value;
      const done = !untilFunction(current);
      if (!done) value = incrementFunction(value);
      return { done, value: current };
    };

    return { next };
  };

  return createMonadicSequence(iterator);
};
\end{lstlisting}


\subsubsection{Using a Sequence}
\label{subsub:Using a Sequence}
Listing \ref{lst:even-sequence} shows the definition of a sequence over even numbers smaller than 
10. Then the |for..of| loop iterates over each element of the sequence until
|done| is true. Meanwhile, |console.log| writes the elements out to the terminal.

\begin{lstlisting}[
  style=ES6, 
  caption=Sequence over the even numbers,
  label={lst:even-sequence}
  ]
const startValue        = 0;
const untilFunction     = x => x < 10;
const incrementFunction = x => x + 2;

const seq = Sequence(startValue, untilFunction, incrementFunction);

for (const elem of seq) {
  console.log(elem);
}

// => Logs '0, 2, 4, 6, 8'

\end{lstlisting}

Now, the ability to generate arbitrary sequences of data has been discussed. 
However, just creation is often inadequate as there is a need to manipulate the 
data or introduce additional levels of abstraction. In the upcoming chapter, 
we will delve into the topic of processing such sequences.
