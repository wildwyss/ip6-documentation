\chapter{Modularizing Programs} % (fold)
\label{chap:Modularizing Programs}
This chapter explains how to break down programs and functions into small,
reusable pieces using lazy evaluation and higher-order functions. The start
shows how Haskell implements these concepts. The subsequent discussion covers
the application and limitations of these concepts in JavaScript before showing
how the Sequence library overcomes these limitations, providing valuable tools
for writing reliable and reusable code.

% chapter Modularizing Programs (end)
\section{Gluing Programs together} % (fold)
\label{sec:Gluing Programs together}
John Hughes argues that functional programming languages provide two extra ways
to better partition programs and easily combine them later. He refers to them
as "glues"~\cite[p.3]{hughes_why_1989}. These are, on the one hand,
higher-order functions and, on the other hand, lazy evaluation.
% subsection Gluing Programs together (end)

\subsection{Using these Glues in Haskell} % (fold)
\label{sub:Using these Glues in Haskel}

\subsubsection{Higher-order Functions in Haskell} % (fold)
\label{subsub:Higher-order functions Haskell}
Higher-order functions receive another function as arguments or produce a
function as a result. This concept is fundamental in Haskell, where the whole
program is a single function. The easiest way to understand higher-order
functions is to look at an example:

\begin{lstlisting}[style=Haskell, 
                  caption=Higher-order functions in Haskell, 
                  label={lst:hof_haskell}
]
double    = \x -> 2 * x
quadruple = \x -> 4 * x
doubles   = map double [1,2,3,4]
quads     = map quadruple [1,2,3,4]

print $ doubles ++ quads
-- Prints '[2,4,6,8,4,8,12,16]'
\end{lstlisting}

Listing~\ref{lst:hof_haskell} shows how higher-order functions work in Haskell.
This code is very reusable since |map| takes a function as an argument to
transform the values of a list.
% subsubsection Higher order functions Haskell (end)

\subsubsection{Lazy Evaluation} % (fold)
\label{subsub:Evaluation}
Lazy evaluation describes a concept where a program only evaluates as many
statements as required. This concept makes it possible to work with lists that
have an infinite amount of values. Again, an example explains the concept best:

\begin{lstlisting}[
  style=Haskell,
  caption=lazy evaluation in Haskell,
  label={lst:lazy_eval_haskell}
]
as = [1..]*'\label{line:lazy_eval_haskell_1}'*
doubles = map double as*'\label{line:lazy_eval_haskell_2}'*

print $ take 4 doubles *'\label{line:lazy_eval_haskell_3}'*
-- Prints '[2,4,6,8]'
\end{lstlisting}
Line~\ref{line:lazy_eval_haskell_1} of listing~\ref{lst:lazy_eval_haskell}
defines a list with infinite values. Then the function |double|, defined in the
previous listing ~\ref{lst:hof_haskell}, is applied to this list. As the list
has infinite elements, eagerly evaluating it would take forever. But since
Haskell evaluates its statements lazily, nothing happens on
line~\ref{line:lazy_eval_haskell_2}. Haskell applies |double| not before line
~\ref{line:lazy_eval_haskell_3}, where |print| evaluates a part of the list (the
first four elements). Even then, Haskell exclusively uses |double| on those
four elements.\\
More theoretically, when applying the function |f| to |g x|, |g| produces only
as much output as |f| actually needs. Therefore, working with large data sets
becomes much more convenient and faster. Suppose the elements in a list are
generated with a rule (as with list comprehensions or generators). Then lazy
evaluation brings excellent advantages: The whole list will never materialize
in the memory, only the current value. So you don't have to worry about what
happens when large amounts of data are processed. \\ Using these two concepts,
building large programs consisting of small parts is much easier.

% subsubsection Lazy evaluation (end)
% subsection Using the glues in Haskell (end)

\subsection{Using these Glues in JavaScript} % (fold)
\label{sub:Using these Glues in JavaScript}

\subsubsection{Higher-Order Functions in JavaScript} % (fold)
\label{subsub:Higher Order Functions in JavaScript}
Higher-order functions in JavaScript work similarly as they do in Haskell.
Transferring listing~\ref{lst:hof_haskell} to JavaScript results in the
following:

\begin{lstlisting}[
  style=ES6,
  caption=Higher-order functions in JavaScript,
  label={lst:hof_in_js}
]
const double    = x => 2 * x;
const quadruple = x => 4 * x;
const doubles   = [1,2,3,4].map(double);
const quads     = [1,2,3,4].map(quadruple);

console.log(doubles.concat(quads));
// => Logs '2, 4, 6, 8, 4, 8, 12, 16'
\end{lstlisting}
% subsubsection Higher Order Functions in JavaScript (end)

\subsubsection{Eager Evaluation in JavaScript} % (fold)
\label{subsub:Eager Evaluation in JavaScript}

% subsubsection Lazy Evaluation in JavaScript (end)
It becomes less intuitive when it comes to lazy evaluation in JavaScript:
JavaScript does not know this concept when working with its in-built arrays.
Mapping over an array producing a side effect can shed some light on this:

\begin{lstlisting}[
  style=ES6,
  caption=JavaScript evaluates eagerly,
  label={lst:js_eagerly_eval}
]
const as     = [1,2,3,4,5];
const bs     = [];
const mapped = as.map(x => {
  bs.push(x);
  return 2*x;
});

console.log(mapped.slice(0,4));
// => Logs '2, 4, 6, 8'
console.log(bs);
// => Logs '2, 4, 6, 8, 10'
\end{lstlisting}

Listing ~\ref{lst:js_eagerly_eval} clarifies that even though the program only
uses the first four elements of the array, it traverses each element using the
passed function. This gets clear because the array |bs| contains all array
elements of |as| and not just its first four.

\subsubsection{Lazy Evaluating Iterables} % (fold)
\label{subsub:Lazy Evaluate Iterables}

% subsubsection Lazy Evaluate Iterables (end)
The eager evaluation of JavaScript is a significant limitation. Fortunately,
JavaScript can emulate this laziness using the JS iteration protocols, which
section ~\ref{sec:Iterables in General} describes. \\ 

With the use of the Sequence library, it is possible to rewrite the JavaScript
code of listing~\ref{lst:js_eagerly_eval} lazily:

\begin{lstlisting}[
  style=ES6,
  caption=Lazy evaluation in JavaScript,
  label={lst:lazy_eval_js}
]
const as     = [1,2,3,4,5];
const bs     = [];
const mapped = map(x => {
  bs.push(x);
  return 2*x;
})(as);

console.log(...take(4)(mapped));
// => Logs '2, 4, 6, 8'
console.log(bs);
// => Logs '1, 2, 3, 4'
\end{lstlisting}
The first output is the same as in listing ~\ref{lst:js_eagerly_eval} - the
first four elements of the array |as| have been multiplied by 2. However, the
second output differs: the array |bs| now contains only four elements! So |map|
applied the passed function only to four initial array elements! This fact
proves that transforming arrays using the Sequence library happens lazily!\\
This works because the JS iteration protocols process element by element. The
iterator delivers the next element only if it is requested. If no further
elements are needed, it just stops iterating! \\
Iterators, therefore, decouple the termination condition (asking for another
element) from the loop body (generating/accessing the next element).\\

Furthermore, it is even possible to port the initial Haskell laziness
example defined in listing~\ref{lst:lazy_eval_haskell} to JavaScript:
\begin{lstlisting}[
  style=ES6,
  caption=Process infinite sequences in JavaScript,
  label={lst:inf_seq_js}
]
const seq     = Sequence(1, _ => true, x => x + 1);*'\label{line:inf_seq_js_1}'*
const doubles = map(double)(seq);

console.log(...take(4)(doubles));
// => Logs '2, 4, 6, 8'
\end{lstlisting}

If the code displayed in Listing~\ref{lst:inf_seq_js} were evaluated eagerly,
this would go on forever since line~\ref{line:inf_seq_js_1} defines a sequence
with infinite elements. \\ 

\subsection{Lazy Evaluation and Side Effects} % (fold)
\label{sub:Lazy Evaluation and Side Effects}
Decoupling the loop body and termination condition helps to consider these two
concepts separately. This split reasoning makes sense since they often do not
relate to each other. Why change the whole list of elements if the program only
uses the first five? In JavaScript, however, this thinking carries a danger
that cannot occur in pure functions like those used in Haskell. \\ 
The following listing~\ref{lst:side_effects_lazy_js} states the problem:

\begin{lstlisting}[
  style=ES6,
  caption=Side effects and lazy evaluation,
  label={lst:side_effects_lazy_js}
]
let i = 0;
const mapped = map(x => {
  i++;
  return 2 * x;
})([1,2,3,4]);

console.log(i);
// => Logs '0' i has not been incremented
console.log(...take(2)(mapped));
// => Logs '2, 4' 
console.log(i);
// => Logs '2' i has only been incremented two times
\end{lstlisting}

It is impossible to conclude from the length of the list how often a function
runs. Performing side effects in the loop body can lead to unexpected behaviour. 

It is tough to make assumptions about |i|, because:
\begin{itemize}
  \item |i| is only calculated when the program is evaluated
  \item It is impossible to predict what value |i| will have when the program
    finishes.
\end{itemize}
This example shows that lazy evaluation and side effects almost exclude each
other. Therefore, omitting side effects when working with lazy evaluation makes
sense. In Haskell, where the type system can exclude side effects per function
definition, this is not a problem. Since side effects often make a program
harder to understand, getting rid of it feels even more natural.
% subsection Lazy evaluation and side effects (end)

\section{Placing the Receiver at the End} % (fold)
\label{sec:Placing the Receiver at the End}
Compared to object-oriented programming languages, where the receiver of an
operation is always placed before the point (i.e., at the beginning), functions
in Haskell usually expect it as the last argument. If one sticks to this, it is
possible to give a new name to a partially applied (pre-configured)
function. Line~\ref{line:precon_fs_hs_1} of the following Haskell
listing~\ref{lst:precon_fs_hs} does precisely this and results in a reusable
pre-configured function (|doubleAll|) with a very descriptive name:
\begin{lstlisting}[
  style=Haskell,
  caption=Pre-configured functions in Haskell,
  label={lst:precon_fs_hs}
]
double    = \x -> 2 * x
doubleAll = map double --*'\label{line:precon_fs_hs_1}'* preconfigure map with the function double
as        = [1,2,3,4]
bs        = [5,6,7,8]

print $ doubleAll as ++ doubleAll bs
-- Prints '[2,4,6,8,10,12,14,16]'
\end{lstlisting}

This paradigm allows it also to link existing functions together using the |.|
(dot) operator:

\begin{lstlisting}[
  style=Haskell,
  caption=Function composition in Haskell,
  label={lst:fn_comp_hs}
]
-- "sum" sums up all elements in the list
doubleAndSum = sum . doubleAll

print $ doubleAndSum [1,2,3,4]
-- Prints '20'
\end{lstlisting}

Listing~\ref{lst:fn_comp_hs} links the functions |doubleAll| and |sum| to
create a new function applicable to any list with numerical values. This
concept makes the code very reusable.\\
Placing the receiver at the end is another powerful modularization
tool. Therefore, the sequence library supports this kind of modularization.
Refactoring the initial example from listing~\ref{lst:hof_in_js}, which 
shows how higher-order functions work in JavaScript results in:

\begin{lstlisting}[
  style=ES6,
  caption=Receiver before the point,
  label={lst:rec_bef_point}
]
// initial code
const double  = x => 2 * x;
const doubles = [1,2,3,4].map(double);

// refactored using plain JS
const doubleAll = receiver => receiver.map(double);
console.log(doubleAll([1, 2, 3, 4]));
// => Logs '2, 4, 6, 8'
\end{lstlisting}

Listing~\ref{lst:rec_bef_point} fixes this issue using a function that gets
the receiver as an argument. However, this is not very convenient. The Sequence
library provides a more elegant way to solve this:

\begin{lstlisting}[
  style=ES6,
  caption=Place the receiver at the end in JavaScript,
  label={lst:rcv_at_end_js}
]
const doubleAll = map(double);
\end{lstlisting}

The code of listing~\ref{lst:rcv_at_end_js} is not only less to type but also
more accurate when reading because the term |doubleAll = map(double)| says that
the pre-configured function |map(double)| gets a new name (|doubleAll|). No worrying about
parameter naming is needed here.
% subsubsection Placing the Receiver at the End (end)

\subsubsection{Conclusion} % (fold)
\label{subsub:modularizing_programs_conclusion}
Higher-order functions and lazy evaluation are powerful glues to compose
smaller parts into a more extensive program. Haskell provides a good reference for
this. In JavaScript, higher-order functions are familiar. However, it
offers less support for lazy evaluation - it eagerly evaluates arrays and their
operations. The Sequence library closes this gap because sequences work
similarly to Haskell lists. Thus, the Sequence library allows using both types
of glue in JavaScript.
% subsubsection Conclusion (end)
