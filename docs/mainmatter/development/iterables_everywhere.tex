\section{Iterables Everywhere}
\label{sec:Iterables Everywhere}
This chapter concerns the implications of the previous findings for other
data structures.
Some immutable collections are already implemented in the Kolibri Web Ui
Toolkit~\cite{kolibri}. The Pair example demonstrates the advantages of 
iterating existing implementations in the following section.

\subsection{Making immutable Data Structures iterable}
\label{sub:Making immutable Data Structures iterable}
In the following, we will examine the implementation the immutable data structure |Pair|. 
Among other immutable data structures, the Kolibri Web Ui Toolkit provides an
implementation for |Pair|. 
Listing~\ref{lst:pair_non_iterable} shows how to create and use a Pair.

\begin{lstlisting}[
  style=ES6, 
  caption=Immutable Pair,
  label={lst:pair_non_iterable}
  ]
/** *'\label{line:start_pair_type}'*
 * @typedef PairType
 * @type {  <_T_, _U_>
 *          (x: _T_)
 *       => (y: _U_)
 *       => (s: PairSelectorType<_T_, _U_>) => ( _T_ | _U_ ) 
 *      }
 */ *'\label{line:end_pair_type}'*
const Pair = x => y => selector => selector(x)(y);

const pair = Pair(1)(2);

const one  = pair(fst);*'\label{line:fst_pair}'*
const two  = pair(snd);*'\label{line:snd_pair}'*

console.log(one + " " + two);
// => Logs '1 2'
\end{lstlisting}

The only way to make a Pair immutable is to build it using functions. The type 
signature on line~\ref{line:start_pair_type}~-~\ref{line:end_pair_type} shows 
that the first two arguments are arbitrary values. Pair stores these two values
in its closure scope. 
Selector functions named |fst| on line~\ref{line:fst_pair} and |snd| on 
line~\ref{line:snd_pair} grant access to these values. However, |Pair| offers
no way to modify these values. Listing~\ref{lst:pair_non_iterable} shows that handling a Pair can be tedious. 
It would be great to use the built-in JavaScript language features to access 
the content of a Pair. 

\subsubsection{Iterable Pair}
\label{subsub:Iterable Pair}
Listing~\ref{lst:pair_iterable} demonstrates the implementation of an iterable 
Pair. Still, Pair operates only with functions. However, it additionally defines the
|[Symbol.iterator]| property.

\begin{lstlisting}[
  style=ES6, 
  caption=Iterable Pair,
  label={lst:pair_iterable}
  ]
const Pair = x => y => {
  /**
   * @template _T_, _U_
   * @type { PairSelectorType<_T_,_U_> }
   */
  const pair = selector => selector(x)(y);

  pair[Symbol.iterator] = () => [x,y][Symbol.iterator]();*'\label{line:pair_symbol_iterator}'*

  return pair;
};
\end{lstlisting}

Line~\ref{line:pair_symbol_iterator} shows that 
this property defines a function, which only returns the |[Symbol.iterator]| 
property of array. The array stores the values of the Pair. With that, |Pair|
is an |Iterable|. Therefore, all iterable functions are now available to Pair.
Listing~\ref{lst:handling_pair_iterable} shows the usage of the iterable Pair. 
Line~\ref{line:pair_destructuring} shows the deconstruction of a Pair in the same way as an array. 
This access option is more convenient than in Listing~\ref{lst:pair_non_iterable}. 
Line~\ref{line:show_pair} demonstrates the use of operations from the Sequence 
Library. |show| converts an iterable into a string, analogous to how |toString|
works.

\begin{lstlisting}[
  style=ES6, 
  caption=Working with iterable Pairs,
  label={lst:handling_pair_iterable}
  ]
const pair = Pair(1)(2);

const [one, two] = pair;*'\label{line:pair_destructuring}'*

console.log(show(pair));*'\label{line:show_pair}'*
// => Logs '[1,2]'
\end{lstlisting}

This has significant advantages because it is now possible to process different 
collections with the same abstractions. Therefore, the motivation is great to 
make all collections iterable.

