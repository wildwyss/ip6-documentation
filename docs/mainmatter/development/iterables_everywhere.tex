\section{Iterables Everywhere}
\label{sec:Iterables Everywhere}
The JS iteration protocols are suitable for a wide variety of data structures.
With the Sequence library defining many operations for iterables, it makes
sense to make more objects iterable. For example, the Kolibri Web UI toolkit
defines the immutable data structure "pair", for which there are noteworthy
applications if it is iterable. 
\subsection{Making immutable Data Structures iterable}
\label{sub:Making immutable Data Structures iterable}
Listing~\ref{lst:pair_non_iterable} defines the type pair and shows its usage:
\begin{lstlisting}[
  style=ES6, 
  caption=Immutable Pair,
  label={lst:pair_non_iterable}
  ]
/** *'\label{line:start_pair_type}'*
 * @typedef PairType
 * @type {  <_T_, _U_>
 *          (x: _T_)
 *       => (y: _U_)
 *       => (s: PairSelectorType<_T_, _U_>) => ( _T_ | _U_ ) 
 *      }
 */ *'\label{line:end_pair_type}'*
const Pair = x => y => selector => selector(x)(y);

const pair = Pair(1)(2);

const one  = pair(fst);*'\label{line:fst_pair}'*
const two  = pair(snd);*'\label{line:snd_pair}'*

console.log(one + " " + two);
// => Logs '1 2'
\end{lstlisting}

The only way to make a pair immutable is to build it using functions. The type 
signature on line~\ref{line:start_pair_type}~-~\ref{line:end_pair_type} shows 
that the first two arguments are arbitrary values. Pair stores these two values
in its closure scope, the only scope in JavaScript which can not be changed in
any way from the outside.
Selector functions named |fst| on line~\ref{line:fst_pair} and |snd| on 
line~\ref{line:snd_pair} grant access to these values. However, pair offers
no way to modify these values. Listing~\ref{lst:pair_non_iterable} shows that
handling a pair can be cumbersome. It would be great to use the built-in
JavaScript language features to access the elements of a pair.

\subsubsection{Iterable Pair}
\label{subsub:Iterable Pair}
Listing~\ref{lst:pair_iterable} demonstrates the implementation of an iterable
pair. Still, pair operates only with functions. However, it additionally
defines the |[Symbol.iterator]| property:

\begin{lstlisting}[
  style=ES6, 
  caption=Iterable Pair,
  label={lst:pair_iterable}
  ]
const Pair = x => y => {
  /**
   * @template _T_, _U_
   * @type { PairSelectorType<_T_,_U_> }
   */
  const pair = selector => selector(x)(y);

  pair[Symbol.iterator] = () => [x,y][Symbol.iterator]();*'\label{line:pair_symbol_iterator}'*

  return pair;
};
\end{lstlisting}

Line~\ref{line:pair_symbol_iterator} shows that 
this property defines a function, which only returns the |[Symbol.iterator]|
property of array. The array stores the values of the pair. With that, pairs
are now iterable. Listing~\ref{lst:handling_pair_iterable} shows the usage of
such an iterable pair. Line~\ref{line:pair_destructuring} shows the
deconstruction of a pair in the same way as an array. This access option is
more convenient than in listing~\ref{lst:pair_non_iterable}.
Line~\ref{line:show_pair} demonstrates the usage of operations from the Sequence
library alongside a pair. |show| converts an iterable into a string, analogous
to how |toString| works.

\begin{lstlisting}[
  style=ES6, 
  caption=Working with iterable Pairs,
  label={lst:handling_pair_iterable}
  ]
const pair = Pair(1)(2);

const [one, two] = pair;*'\label{line:pair_destructuring}'*

console.log(show(pair));*'\label{line:show_pair}'*
// => Logs '[1,2]'
\end{lstlisting}

This has significant advantages because it is now possible to process different 
collections with the same abstractions. Therefore, the motivation is great to 
make all collections iterable.

% chapter Iterators in JavaScript (end)
