\chapter{API} % (fold)
\label{chap:api}

\section*{Reader Guidance} % (fold)
\label{sec:api_reader_guidance}
By studying this Chapter, a library user will get an overview of its features.
The provided code examples illustrate simple cases. Therefore, the reader will
only partially understand the concepts and background of the work when reading
this Chapter. Chapter~\ref{chap:development} provides a much deeper conceptual
understanding. \\ 
The Chapter starts with an overview of this library and explains how to import
it in your project. After that, it shows the whole API of the Sequence library,
JINQ and explains the additions to the Kolibri standard library. The Sequence
library is extensible with custom features - the end of this Chapter shows how
to achieve this.


% section Reader Guidance (end)
% chapter API (end)
\section{Introduction} % (fold)
\label{sec:Introduction}

\subsection{Library Overview} % (fold)
\label{sub:Sequence Library Overview}
This library~\cite{wildwyss_kolibri} consists of several parts. Table~\ref{tab:library_overview} gives a brief
overview of each of those. The following Sections discuss these different
aspects.
% subsection Sequence Library Overview (end)

\begin{table}[H]
  \centering
  \begin{tabularx}{\textwidth}{| l | X |} \hline
    \textbf{Part} & \textbf{Description} \\ \hline
    Sequence library & The Sequence library provides operations for processing \lstinline{Iterable}s. Additionally, there are constructors to create Sequences.\\ \hline 
    JINQ & A range generates a sequence of numbers. \\ \hline 
    Stdlib & Extends the Kolibri standard library with additional features.\\ \hline 
    FocusRing & A FocusRing is an immutable data structure, whose elements are arranged in a ring. \\ \hline 
  \end{tabularx}
  \caption{The available operators in the Sequence library}
  \label{tab:library_overview}
\end{table}

\textit{Note:} The FocusRing results from the predecessor project and is only
listed here for completeness. The same applies to the Range, which is part of
the Sequence library. Please find more information in the documentation of
the previous work.~\cite{wild_ip5_2023}. 

\subsection{Including the Library in a Project} % (fold)
\label{sub:Including the Library in a Project}
This library is part of the Kolibri project~\cite{kolibri} and uses many of its
functionalities. The Kolibri ships this library, therefore. The Kolibri
is available on GitHub and can be included directly in existing projects via
the ES6 module system.
% subsection Including the Sequence Library in a Project (end)
% section Introduction (end)

\section{Sequence Library} % (fold)
A Sequence is a lazy data structure. This section overviews which operations
exist to create and modify sequences and how to import this library into your
own project. An in-depth view of how a Sequence works
internally gives Chapter~\ref{sec:Sequence and Iterable in General}.

\subsection{Getting Started} % (fold)
\label{sub:Getting Started}
\subsubsection{Importing the Sequence Library} % (fold)
\label{sec:Importing the Sequence Library}

% subsubsection Importing the Sequence Library (end)
The module |sequence/sequence.js| exports the whole sequence library. The
easiest way to use it is named imports. As
Listing~\ref{lst:named_imports_seq_lib_example} shows, this approach allows
using the dot notation, which enables the IDE to support the development with
auto-completion. Developers who have been using the library longer can import
individual functions directly.
\begin{lstlisting}[
  style=ES6,
  caption=Importing the Sequence library using named imports,
  label={lst:named_imports_seq_lib_example}
]
import * as _ from "./sequence/sequence.js"

const seq = _.Sequence(0, i => i < 5, i => i + 1);

console.log(_.show(seq));
// => Logs '[0, 1, 2, 3, 4]'
\end{lstlisting}
\textit{Note:} Of course, importing the Sequence library with any name is
possible. The advantage of |_| is that it is short and takes up very little
white space. Other special characters, such as the \lstinline{$}, negatively
affect the code's readability. 
% subsection Importing the Sequence Library (end)

\subsubsection{Evaluating endless Sequences} % (fold)
\label{sec:Evaluating endless Sequences}
Section~\ref{Lazy Evaluate Iterables} describes the benefits of lazy
evaluation, which allows working with sequences consisting of infinite
elements. When working with such sequences, only evaluate parts of the sequence
because evaluating endless values would take forever. Functions of the Sequence
library, which evaluate a whole sequence, are therefore marked with a
\lstinline{$} symbol in the name. These functions are only applicable to
sequences with a finite number of elements. \\
Listing~\ref{lst:api_endl_seq_eval} shows an example for that:

\begin{lstlisting}[
  style=ES6,
  caption=Evaluating a part of an endless Sequence,
  label={lst:api_endl_seq_eval}
]
import * as _ from "./sequence/sequence.js"

const seq      = _.Sequence(0, _ => true, x => x + 1); *'\label{line:api_endl_seq_eval1}'*
const part     = _.take(5)(seq); *'\label{line:api_endl_seq_eval2}'*

const reversed = _.reverse$(part);

console.log(...reversed);
// => Logs '4, 3, 2, 1, 0'
\end{lstlisting}

Line~\ref{line:api_endl_seq_eval1} creates a Sequence |seq| which contains
infinite elements. Evaluating this would go on forever. Therefore,
Line~\ref{line:api_endl_seq_eval2} uses the function |_.take(5)| which only
evaluates |5| elements.

% subsubsection Conventions (end)
\label{sec:api_sequences}
\subsection{Constructors} % (fold)
\label{sub:api_Constructors}
A constructor is anything that builds a |Sequence| without depending on another
one. So they serve as an entry point to the Sequence library. Some of these
constructors create a specific series of values, including, for example, the
|PrimeNumberSequence| which yields the infinite sequence of all prime
numbers.\\
Table~\ref{tab:api_ctors} gives an overview of all available constructors. Code
examples and more information about the constructors deliver the
appendices~\ref{sub:appendix_constructors} and
\ref{sub:appendix_special_constructors}.

\begin{table}[H]
  \centering
  \begin{tabularx}{\textwidth}{| l | X |} \hline
    \textbf{Name} & \textbf{Description} \\ \hline
    \texttt{Sequence} & Creates a new Sequence based on a start \texttt{value}, \texttt{incrFn} and \texttt{stopFn}. \\ \hline
    \texttt{PureSequence} & Creates a Sequence which contains just the given value. \\ \hline
    \texttt{repeat} & Returns a Sequene that will repeatedly yield the value of \texttt{arg} when iterated over. \texttt{repeat} will never be exhausted. \\ \hline
    \texttt{replicate} & \texttt{replicate(n)(x)} creates a Sequence of length \texttt{n} with \texttt{x} the value of every element. \\ \hline
    \texttt{StackSequence} & Creates a \texttt{SequenceType} on top of the given \texttt{stack}. \\ \hline
    \texttt{TupleSequence} & Constructs a new \texttt{SequenceType} based on the given tuple. \\ \hline
    \texttt{AngleSequence} & Creates a Sequence which generates evenly spaced angles between 0 and 360. \\ \hline
    \texttt{FibonacciSequence} & Generates the Fibonacci sequence. \\  \hline
    \texttt{PrimeNumberSequence} & Generates the sequence of prime numbers. \\ \hline
    \texttt{SquareNumberSequence} & Generates the sequence of square numbers. \\ \hline
    \texttt{Range} & Generates a sequence of numbers. \cite{wild_ip5_2023} \\ \hline
  \end{tabularx}
  \caption{The available constructors in the Sequence library}
  \label{tab:api_ctors}
\end{table}
% subsection Constructors (end)
\subsection{Operators} % (fold)
\label{sub:api_Operators}
Operators are all functions that operate on an existing |Sequence| and create a
new |Sequence| from it. \\
Table~\ref{tab:api_operators} gives an overview of all available operators.\\
Code examples and more information about the operators delivers the
appendix~\ref{sub:appendix_operators}.
\begin{table}[H]
  \centering
  \begin{tabularx}{\textwidth}{| l | X |} \hline
    \textbf{Name} & \textbf{Description} \\ \hline
    \texttt{bind} & Applies the given function to each element of the \texttt{Iterable} and flats it afterward. Note This operation adds a monadic API to the \texttt{SequenceType}. \\ \hline 
    \texttt{catMaybes} & The catMaybes function takes an \texttt{Iterable} of \texttt{MaybeType}s and returns a \texttt{SequenceType} of all the Just's values. \\ \hline 
    \texttt{concat} & Adds the second iterable to the first iterables end. \\ \hline 
    \texttt{cons} & Adds the given element to the front of an iterable. \\ \hline 
    \texttt{cycle} & Ties a finite \texttt{Iterable} into a circular one, or equivalently, the infinite repetition of the original \texttt{Iterable}. \\ \hline 
    \texttt{drop} & Jumps over so many elements. \\ \hline 
    \texttt{dropWhile} & Discards all elements until the first element does not satisfy the predicate anymore. \\ \hline 
    \texttt{map} & Transforms each element using the given function. \\ \hline 
    \texttt{mconcat} & Flatten an \texttt{Iterable} of \texttt{Iterable}s. \\ \hline 
    \texttt{pipe} & Transforms the given \texttt{Iterable} using the passed operators. \\ \hline 
    \texttt{rejectAll} & Only keeps elements which does not fulfill the given predicate. \\ \hline 
    \texttt{retainAll} & Only keeps elements which fulfill the given predicate. \\ \hline 
    \texttt{reverse\$} & Processes the \texttt{Iterable} backwards. \\ \hline 
    \texttt{snoc} & Adds the given element to the end of the \texttt{Iterable}. \\ \hline 
    \texttt{take} & Stop after so many elements. \\ \hline 
    \texttt{takeWhile} & Proceeds with the iteration until the predicate becomes true. \\ \hline 
    \texttt{zip} & Zip takes two \texttt{Iterable}s and returns an \texttt{Iterable} of corresponding \texttt{Pair}s. \\ \hline 
    \texttt{zipWith} & Generalises \texttt{zip} by zipping with the function given as the first argument. \\ \hline 
  \end{tabularx}
  \caption{The available operators in the Sequence library}
  \label{tab:api_operators}
\end{table}
% section Operators (end)

\subsection{Terminal Operations} % (fold)
\label{sub:api_Terminal Operations}
Terminal operations are all functions that operate on an existing |Sequence|
and do not necessarily create a new |Sequence| from it. In other words,
terminal operations evaluate a |Sequence|.\\ 
Table~\ref{tab:api_term_ops} gives an overview of all available terminal
operations.\\
Code examples and more information about the constructors delivers the
appendix~\ref{sub:appendix_terminal_operations}.

\begin{table}[H]
  \centering
  \begin{tabularx}{\textwidth}{| l | X |} \hline
    \textbf{Name} & \textbf{Description} \\ \hline
    \texttt{eq\$} & Checks the equality of two finite \texttt{Iterable}s. \\ \hline 
    \texttt{foldr} & Performs a reduction on the elements from right to left, using the provided start value and an accumulation function, and returns the reduced value. \\ \hline 
    \texttt{forEach\$} & Executes the callback for each element. \\ \hline 
    \texttt{head} & Return the next value without consuming it. \texttt{undefined} when there is no value. \\ \hline 
    \texttt{isEmpty} & Returns \texttt{true}, if the iterables head is undefined. \\ \hline 
    \texttt{max\$} & Returns the largest element of a \textbf{non-empty} \texttt{Iterable}. \\ \hline 
    \texttt{safeMax\$} & Returns the largest element of an \texttt{Iterable}. \\ \hline 
    \texttt{min\$} & Returns the smallest element of a \textbf{non-empty} \texttt{Iterable}. \\ \hline 
    \texttt{safeMin\$} & Returns the smallest element of an \texttt{Iterable}. \\ \hline 
    \texttt{foldl\$} & Performs a reduction on the elements, using the provided start value and an accumulation function, and returns the reduced value. \\ \hline 
    \texttt{show} & Transforms the passed \texttt{Iterable} to a \texttt{String}. \\ \hline 
    \texttt{uncons} & Removes the first element of this iterable. \\ \hline 
  \end{tabularx}
  \caption{The available constructors of the Sequence library}
  \label{tab:api_term_ops}
\end{table}
% section Terminal Operations (end)
\subsection{Sequence Prototype} % (fold)
\label{sub:Sequence Prototype}
The prototype of the Sequence provides some operations as well. This improves
code readability in some situations. The monadic operations are also available
on the prototype of the Sequence. Thus, it is compatible with all functions
that expect a monad as a parameter.

\subsubsection{Verify the Prototype} % (fold)
\label{subsub:Verify the Prototype}
Listing 3 shows how to query the prototype of any object using the function
|Object.getPrototypeOf|. Applying this to any sequence will return
|SequencePrototype|.


\begin{lstlisting}[
  style=ES6,
  caption=The Prototype of a \lstinline{Sequence},
  label={lst:proto_seq_query}
]
import { Sequence, SequencePrototype } from "sequence/sequence.js";

const seq          = Sequence(0, _ => true, i => i + 1);
const seqPrototype = Object.getPrototypeOf(seq);

console.log(seqPrototype === SequencePrototype);
// => Logs 'true'
\end{lstlisting}

% subsubsection Verify the Prototype (end)

\subsubsection{The operations on the Prototype} % (fold)
\label{subsub:The operations on the Prototype}
\begin{table}[H]
  \centering
  \begin{tabularx}{\textwidth}{| l | X |} \hline
    \textbf{Name} & \textbf{Description} \\ \hline
    \texttt{fmap} & Is the same as \lstinline{map} described in Section~\ref{sub:api_Operators}\\ \hline 
    \texttt{and} & Is the same as \lstinline{bind} described in Section~\ref{sub:api_Operators}\\ \hline 
    \texttt{pure} & Is the same as \lstinline{PureSequence} described in Section~\ref{sub:api_Constructors}\\ \hline 
    \texttt{empty} & Is the same as \lstinline{nil} described in Section~\ref{sub:api_Constructors}\\ \hline 
    \texttt{["=="]} & Is the same as \lstinline{eq$} described in Section~\ref{sub:api_Terminal Operations}\\ \hline 
    \texttt{toString} & Is the same as \lstinline{show} described in Section~\ref{sub:api_Terminal Operations}\\ \hline 
    \texttt{pipe} & Is the same as \lstinline{pipe} described in Section~\ref{sub:api_Operators}\\ \hline 
  \end{tabularx}
  \caption{The operations served on the prototype of the \lstinline{Sequence}}
  \label{tab:api_term_ops}
\end{table}

% subsubsection The operations on the Prototype (end)
% subsection Sequence Prototype (end)
% section Sequence (end)



\section{Extension of the Kolibri Standard Library}
\label{sec:Extension of the Kolibri Standard Library}
From this project's findings, some existing functionalities of the Kolibri
could also be improved or extended. This section describes those extensions.

\subsection{Extensions to Maybe} % (fold)
\label{sub:Extensions to Maybe}
The data structure |Maybe| introduced in section~\ref{sec:Wrapping values in a
context} was already part of the Kolibri before this project work. Now the
prototype of |Maybe| also supports monadic operations.\\
Listing~\ref{lst:monadic_ops_maybe_in_action} shows these monadic operations of
|Maybe| in action:

\begin{lstlisting}[
  style=ES6,
  caption=The monadic operations of Maybe in action,
  label={lst:monadic_ops_maybe_in_action}
]
import { Nothing, Just } from "./src/stdlib/maybe.js";

/** Prints the value of a Maybe if it exists */
const evalMaybe = maybe =>
  maybe
    (_ => console.log("There was no value!"))
    (x => console.log(x));

const just    = Just(5);
const nothing = Nothing;

// fmap 
const justMapped    = just   .fmap(x => 2*x); // results in 10
const nothingMapped = nothing.fmap(x => 2*x); // nothing happens!
evalMaybe(justMapped);                        // => Logs '10'
evalMaybe(nothingMapped);                     // => Logs 'There was no value!' 

// and
const justAnd    = just   .and(x => nothing); // Turns this value into Nothing
const nothingAnd = nothing.and(x => just);    // Can't change Nothing
evalMaybe(justAnd);                           // => Logs 'There was no value!'
evalMaybe(nothingAnd);                        // => Logs 'There was no value!' 

// pure
evalMaybe(just.pure(2));                      // => Logs '2', Same as Just(2)
evalMaybe(nothing.pure(2));                   // => Logs '2', Same as Just(2)

// empty
evalMaybe(just.empty());    // => Logs 'There was no value!', Same as Nothing
evalMaybe(nothing.empty()); // => Logs 'There was no value!', Same as Nothing
\end{lstlisting}
% subsection Extensions to Maybe (end)

\subsection{Extensions to Pair} % (fold)
\label{sub:Extensions to Pair}
Section~\ref{sec:Iterables Everywhere} describes the |Pair| as an immutable
data structure. To access its values simpler, it is now iterable.
Listing~\ref{lst:api_it_pair} shows these advantages:

\begin{lstlisting}[
  style=ES6,
  caption=Iterable Pair,
  label={lst:api_it_pair}
]
import { Pair } from "./src/stdlib/pair.js"

const pair = Pair(1)(2);
console.log(...pair);
// => Logs '1, 2'
\end{lstlisting}

Since the |Pair| is iterable, it is also compatible with all operations defined
for sequences:

\begin{lstlisting}[
  style=ES6,
  caption=Transforming a \lstinline{Pair} using \lstinline{map},
  label={lst:Mapping the values of a Pair}
]
import { Pair } from "./src/stdlib/pair.js"
import { map }  from "./src/sequence/sequence.js"

const pair = Pair(1)(2);
console.log(...map(x => 2*x)(pair));
// => Logs '2, 4'
\end{lstlisting}


% subsection Extensions to Pair (end)

\input{./mainmatter/api/prototype_functions}
\section{Sequence Builder}
\label{sec:Sequence Builder}
A mutable builder for a |SequenceType|.
\newline
|SequenceBuilder| allows the creation of a |SequenceType| by generating elements individually 
and adding them to the |SequenceBuilder| (without the call stack overhead when doing so
with |cons|).
\newline
\textit{Note:} It is strongly recommended, therefore, to use |SequenceBuilder|
for creating |Sequences| with more than 1000 elements when adding them
individually.

\subsection{Functions}
\label{sub:sb_Functions}

|SequenceBuilder| provides the following functions:

\begin{table}[H]
  \centering
  \begin{tabularx}{\textwidth}{| l | X |} \hline
    \textbf{Name}    & \textbf{Description} \\ \hline
    \texttt{prepend} & adds one or more elements to the beginning of the sequence\\ \hline 
    \texttt{append}  & adds one or more elements to the end of the sequence\\ \hline 
    \texttt{build}   & builds a sequence containing the previously passed elements \\ \hline 
   \end{tabularx}
  \caption{SequenceBuilder functions}
  \label{tab:sb_functions}
\end{table}

\subsection{Examples}
\label{sub:sb_Examples}
Creating a sequence from different values and iterables:

\begin{lstlisting}[
  style=ES6, 
  caption=SequenceBuilder example,
  label={lst:sb_example}
  ]
import * as _              from "./src/sequence/sequence.js"
import { SequenceBuilder } from "./src/sequence/SequenceBuilder.js";

const range = _.Range(1, 3);

const result = SequenceBuilder()
.append(range)
.append(4)
.prepend(0)
.append(5,6,7)
.build();

console.log(_.show(result));
// => Logs '[0,1,2,3,4,5,6,7]'
\end{lstlisting}

\section{JINQ} % (fold)
\label{sec:API_JINQ}
Write JINQ queries (JavaScript language integrated queries) with every monadic type.
\newline
For further details and explanations consider the documentation in
Chapter~\ref{sub:Introducing JINQ}.
\newline
To create a monadic type, visit the Chapter~\ref{sec:Monads in JavaScript}.


\subsection{Functions}
\label{sub:JINQ_Functions}
JINQ provides the following functions:

\begin{table}[H]
  \centering
  \begin{tabularx}{\textwidth}{| l | X |} \hline
    \textbf{Name}       & \textbf{Description} \\ \hline
    \texttt{from}       & Serves as starting point to enter JINQ and specifies a data source \\ \hline 
    \texttt{map}        & Maps the current value to a new value \\ \hline 
    \texttt{select}     & Alias for \texttt{map} (same functionality) \\ \hline 
    \texttt{where}      & Only keeps the items that fulfill the predicate \\ \hline 
    \texttt{inside}     & Maps the current value to a new \texttt{MonadType} \\ \hline 
    \texttt{pairWith}   & Combines the underlying data structure with the given data structure as \texttt{PairType} \\ \hline 
    \texttt{result}     & Returns the result of this query\\ \hline 

  \end{tabularx}
  \caption{JINQ functions}
  \label{tab:jinq_functions}
\end{table}

\subsection{Examples}
\label{sub:JINQ_Examples}

\subsubsection{Even Numbers}
\label{subsub:JINQ_Even Numbers}
Generating a Sequence of even numbers using JINQ and a Range as data source:
\newline
\textit{Note:} Since |result| returns a |MonadType|, casting is needed
for converting with |show|.

\begin{lstlisting}[
  style=ES6, 
  caption=Even numbers generated by JINQ,
  label={lst:jinq_even_numbers}
  ]
import * as _    from "./src/sequence/sequence.js"
import { from }  from "./src/jinq/jinq.js";
import { Range } from "./src/range/range.js";

const range = Range(10)
const result =
    from(range)
      .where(x => x % 2 === 0)
      .result();

console.log(_.show(/** @type SequenceType */ result));
// => Logs '[0,2,4,6,8,10]
\end{lstlisting}

\subsubsection{Pythagorean Triple}
\label{subsub:JINQ_Pythagorean Triple}
Searching for all Pythagorean Triple~\cite{pythagorean_triple} between 1 and 10.

\begin{lstlisting}[
  style=ES6, 
  caption=The Pythagorean Triple between 1 and 10,
  label={lst:triple}
  ]
import * as _    from "./src/sequence/sequence.js"
import { from }  from "./src/jinq/jinq.js";
import { Range } from "./src/range/range.js";

const range= Range(1, 10);

const result =
  from(range)
    .pairWith(range)
    .pairWith(range)
    .where ( ([ [a,b], c ]) => a * a + b * b === c * c)
    .select( ([ [a,b], c ]) => `[${a}, ${b}, ${c}]`)
    .result();

console.log(show(/** @type SequenceType */ result));
// => Logs '[[3, 4, 5],[4, 3, 5],[6, 8, 10],[8, 6, 10]]
\end{lstlisting}

\section{How to Extend the Library}
\label{sec:How to Extend the Library}
This Chapter gives a quick start to extending the Sequence Library. Because
applications are versatile, you may want to add further functionality to the
Library. Therefore, this Chapter explains the relevant steps to simplify the entry.
\newline
Whether you want to add a constructor, an operator, or a terminal operation is
similar. Below is an example of an operator. You can adapt the procedure for
the others.

\subsection{Adding a new Operator}
\label{sub:Adding a new Operator}
In this scenario, we include |concat| to the Sequence Library. |concat| is the
operation that appends one |Iterable| to another.

\subsubsection{Kind of the Operation}
The library distinguishes between operators and terminal operations. The
difference depends on the return type. If a function returns an |Iterable|, it
is an operator. It is a terminal operation if it produces a different type than an
|Iterable|, such as |isEmpty|.
Since |concat| returns an |Iterable|, we include the related files into the
operations folder.

\subsubsection{Directory Structure}
\label{subsub:Directory Structure}
|concat| has its folder in the operations directory. Therefore, we create a
|concat.js| and a |concatTest.js| file in the folder |concat|. The reasoning for 
splitting tests into a separate file is that it leads to a better overview and faster
navigation between the particular functionalities. Figure~\ref{fig:concat_dir}
shows the relevant directories.

\begin{figure}[H]
\dirtree{%
  .1 src .
  .2 sequence .
  .3 operators .
  .4 concat .
  .5 concat.js .
  .5 concatTest.js .
  .4 \ldots .
  .4 operators.js .
  .4 operatorsTest.js .
  .3 \ldots .
  .2 AllTests.html .
  .2 \ldots .
  }
  \caption{concat directory structure}
  \label{fig:concat_dir}
\end{figure}

\subsubsection{Exports and Imports}
All artifacts of the Sequence Library are available via the |sequence.js| file.
To make |concat| support this, we export it via the |operators.js| file.
Listings~\ref{lst:concat_export} and \ref{lst:concat_export_operators} shows the corresponding statements.

\begin{lstlisting}[
  style=ES6, 
  caption=export of concat,
  label={lst:concat_export}
  ]
// concat.js

export { concat }
\end{lstlisting}



\begin{lstlisting}[
  style=ES6, 
  caption=export of concat in operator.js,
  label={lst:concat_export_operators}
  ]
// operators.js
...
export * from "./concat/concat.js"
\end{lstlisting}

The same principle applies to the test files. Importing the |concatTest.js| in
|operatorsTest.js| enables to run of the test cases.

\begin{lstlisting}[
  style=ES6, 
  caption=export of concat in operator.js,
  label={lst:concat_import_operators}
  ]
// operatorsTest.js
...
import "./concat/concat.js"
\end{lstlisting}

\subsection{Implementing a new Operator}
\label{subsub:Implementing a new Operator}
Now we're getting down to the nitty-gritty.
Because great software engineers start with the tests, we do that first.

\subsubsection{Write the Tests}
\label{subsub:Write the Tests}
Chapter~\ref{sec:Testing} has already explained testing in detail. Therefore, we will only deal 
with the practical part here.
\newline
Let us start with the imports. For testing reasons, you need the following
functionalities:

\begin{itemize}
  \item{TestSuite}
  \item{addToTestingTable}
  \item{createTestConfig}
\end{itemize}

You may need additional imports to implement the tests.
\newline
Now you are ready to create a test config and, optionally, some special cases.
Creating a test config requires the following steps:

\begin{itemize}
  \item{Create a |TestSuite| with a meaningful name corresponding to the function you are implementing.}
  \item{|createTestConfig| expects an object of type |SequenceTestConfigType|. Have a look
    to this type or to Chapter~\ref{subsub:The Configuration} to get the information about the available properties. }
      \item{If neccessary, add some further test cases using the same TestSuite.}
  \item{Run the |TestSuite| by calling the |run()| function at the end of the file. }
\end{itemize}

Listing~\ref{lst:concat_test_import} shows a scaffold of a test file. 

\begin{lstlisting}[
  style=ES6, 
  caption=Imports of concatTest.js,
  label={lst:concat_test_import}
  ]
import { addToTestingTable }  from "../../util/testingTable.js";
import { TestSuite }          from "../../../test/test.js";
import { createTestConfig }   from "../../util/testUtil.js";
import { concat }             from "../../sequence.js";
...

const testSuite = TestSuite("Name of the TestSuite");

addToTestingTable(testSuite)(
  createTestConfig({
      ...
    })
);

testSuite.add("test special case", assert => {
  \\ Given
  ...
  \\ When
  ...
  \\ Then
  ...
});

testSuite.run();
\end{lstlisting}

Run the |AllTests.html| file, and you should see that your tests are failing.
If so, then everything is fine.

\textit{Note:} While running the tests, always observe the console.

\subsubsection{Write the Functionality}
Now you are in a comfortable position to implement the function against tests.
For this, we create a function |concat| in the file |concat.js|.
Probably, you can use existing functionalities to implement them. A list of all
provided functionality is in Chapter~\ref{sub:Constructors} - \ref{sub:Terminal Operations}.
\newline
Again, run the tests and be proud if everything is running!

