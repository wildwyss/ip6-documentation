\chapter{API} % (fold)
\label{chap:api}

\section*{Reader Guidance} % (fold)
\label{sec:api_reader_guidance}
By studying this Chapter, a library user will get an overview of its features.
The provided code examples illustrate simple cases. Therefore, the reader will
only partially understand the concepts and background of the work when reading
this Chapter. Chapter~\ref{chap:development} provides a much deeper conceptual
understanding. \\ 
The Chapter starts with an overview of this library and explains how to import
it in your project. After that, it shows the whole API of the Sequence library,
JINQ and explains the additions to the Kolibri standard library. The Sequence
library is extensible with custom features - the end of this Chapter shows how
to achieve this.


% section Reader Guidance (end)
% chapter API (end)
\section{Introduction}

\section{Sequence Library} % (fold)
\label{sec:Sequence Library}
This section overviews which operations exist to create and process sequences
and how to import this library into your own project.
\subsection{Getting Started} % (fold)
\label{sub:Getting Started}
\subsubsection{Importing the Sequence Library} % (fold)
\label{subsub:Importing the Sequence Library}

The module |sequence/sequence.js| exports the whole Sequence library. The
easiest way to use it is through named imports. As
listing~\ref{lst:named_imports_seq_lib_example} shows, this approach allows
using the dot notation, which enables the IDE to support the development with
auto-completion. Developers who have been using the library longer can import
individual functions directly.
\begin{lstlisting}[
  style=ES6,
  caption=Importing the Sequence library using named imports,
  label={lst:named_imports_seq_lib_example}
]
import * as _ from "./sequence/sequence.js"

const seq = _.Sequence(0, i => i < 5, i => i + 1);

console.log(_.show(seq));
// => Logs '[0, 1, 2, 3, 4]'
\end{lstlisting}
\textit{Note:} Of course, importing the Sequence library with any name is
possible. The advantage of |_| is that it is short and takes up very little
white space. Other special characters, such as the \lstinline{$}, negatively
affect the code's readability. 
% subsubsection Importing the Sequence Library (end)
% subsection Importing the Sequence Library (end)

\subsubsection{Evaluating endless Sequences} % (fold)
\label{subsub:Evaluating endless Sequences}
Section~\ref{subsub:Lazy Evaluate Iterables} describes the benefits of lazy
evaluation, which allows working with sequences consisting of infinite
elements. When working with such sequences, pay attention to only evaluating
parts of the sequence because evaluating endless values would take forever.
Functions of the sequence library, which will process the whole sequence, are
therefore marked with a \lstinline{$} symbol in the name. These functions are
only applicable to sequences with a finite number of elements. \\
Listing~\ref{lst:api_endl_seq_eval} shows an example of that:

\begin{lstlisting}[
  style=ES6,
  caption=Evaluating a part of an endless Sequence,
  label={lst:api_endl_seq_eval}
]
import * as _ from "./sequence/sequence.js"

const seq      = _.Sequence(0, _ => true, x => x + 1); *'\label{line:api_endl_seq_eval1}'*
const part     = _.take(5)(seq); *'\label{line:api_endl_seq_eval2}'*

const reversed = _.reverse$(part);

console.log(...reversed);
// => Logs '4, 3, 2, 1, 0'
\end{lstlisting}

Line~\ref{line:api_endl_seq_eval1} creates a sequence |seq| which contains
infinite elements. Evaluating this would go on forever. Therefore,
Line~\ref{line:api_endl_seq_eval2} uses the function |_.take(5)| which only
evaluates |5| elements.

% subsubsection Conventions (end)
\subsection{Constructors} % (fold)
\label{sub:api_Constructors}
A constructor is anything that creates a sequence without depending on another
one. So they serve as an entry point to the Sequence library. \\
Table~\ref{tab:api_ctors} gives an overview of all available constructors. Code
examples and more information about the constructors delivers the
appendix~\ref{sub:appendix_constructors}.

\begin{table}[H]
  \centering
  \begin{tabularx}{\textwidth}{| l | X |} \hline
    \textbf{Name} & \textbf{Description} \\ \hline
    \texttt{nil} & This constant represents a sequence containing no values. \\ \hline
    \texttt{PureSequence} & Creates a sequence which contains just the given value. \\ \hline
    \texttt{Range} & Generates a sequence of numbers. \cite{wild_ip5_2023} \\ \hline
    \texttt{repeat} & Returns a sequence that will repeatedly yield the value of \texttt{arg} when iterated over. \texttt{repeat} will never be exhausted. \\ \hline
    \texttt{replicate} & \texttt{replicate(n)(x)} creates a sequence of length \texttt{n} with \texttt{x} the value of every element. \\ \hline
    \texttt{Sequence} & Creates a new sequence based on a start \texttt{value}, \texttt{whileFunction} and \texttt{incrementFunction}. \\ \hline
    \texttt{StackSequence} & Creates a \texttt{SequenceType} on top of the given \texttt{stack}. \\ \hline
    \texttt{TupleSequence} & Constructs a new \texttt{SequenceType} based on the given tuple. \\ \hline
  \end{tabularx}
  \caption{The available constructors in the Sequence library}
  \label{tab:api_ctors}
\end{table}
% subsection Constructors (end)

\subsection{Operators} % (fold)
\label{sub:api_Operators}
Operators are functions that operate on any existing iterable and create a
new sequence from it. \\
Table~\ref{tab:api_operators} gives an overview of all available operators.\\
Code examples, types and more information about the operators delivers the
appendix~\ref{sub:appendix_operators}.
\begin{table}[H]
  \centering
  \begin{tabularx}{\textwidth}{| l | X |} \hline
    \textbf{Name} & \textbf{Description} \\ \hline
    \texttt{bind} & Applies the given function to each element of the \texttt{Iterable} and flattens it afterward. Note This operation adds a monadic API to the \texttt{SequenceType}. \\ \hline 
    \texttt{catMaybes} & The catMaybes function takes an \texttt{Iterable} of \texttt{MaybeType}s and returns a \texttt{SequenceType} of all the Just's values. \\ \hline 
    \texttt{concat} & Adds the second \texttt{Iterable} to the first iterables end. \\ \hline 
    \texttt{cons} & Adds the given element to the front of an \texttt{Iterable}. \\ \hline 
    \texttt{cycle} & Ties a finite \texttt{Iterable} into a circular one, or equivalently, the infinite repetition of the original \texttt{Iterable}. \\ \hline 
    \texttt{drop} & Jumps over so many elements. \\ \hline 
    \texttt{dropWhile} & Discards all elements until the first element does not satisfy the predicate any more. \\ \hline 
    \texttt{map} & Transforms each element using the given function. \\ \hline 
    \texttt{mconcat} & Flatten an \texttt{Iterable} of iterables. \\ \hline 
    \texttt{pipe} & Transforms the given \texttt{Iterable} using the passed operators. \\ \hline 
    \texttt{rejectAll} & Only keeps elements which do not fulfil the given predicate. \\ \hline 
    \texttt{retainAll} & Only keeps elements which fulfil the given predicate. \\ \hline 
    \texttt{reverse\$} & Processes the \texttt{Iterable} backwards. \\ \hline 
    \texttt{snoc} & Adds the given element to the end of the \texttt{Iterable}. It is the opposite of \texttt{cons}.\\ \hline 
    \texttt{take} & Stop after so many elements. \\ \hline 
    \texttt{takeWhile} & Proceeds with the iteration until the predicate becomes true. \\ \hline 
    \texttt{zip} & Zip takes two iterables and returns an \texttt{Iterable} of corresponding pairs. \\ \hline 
    \texttt{zipWith} & Generalises \texttt{zip} by zipping with the function given as the first argument. \\ \hline 
  \end{tabularx}
  \caption{The available operators in the Sequence library}
  \label{tab:api_operators}
\end{table}

\subsubsection{Using the pipe Operator} % (fold) 
\label{subsub:Using the pipe Operator}
The operator |pipe|, is unique because it provides no new functionality but is
pure syntactic sugar. It offers the possibility of combining several operators.
Lines~\ref{line:api_pipe1}~-~\ref{line:api_pipe2} of listing~\ref{lst:api_pipe}
show a nesting of multiple operators. Readability suffers very much from this.
In comparison, it is easier to read the code
lines~\ref{line:api_pipe3}~-~\ref{line:api_pipe4}, combining multiple operators
using |pipe|. 
\begin{lstlisting}[
  style=ES6,
  caption=pipe combines multiple operators,
  label={lst:api_pipe}
]
import * as _ from "./src/sequence/sequence.js"

const numbers = _.Sequence(0, _ => true, x => x + 1);

const mySequence = _.take(5)( *'\label{line:api_pipe1}'*
  _.retainAll(x => x > 50)(
    _.map(x => 2 * x)(
      _.rejectAll(x => x % 2 === 1)(numbers)
    )
  )
);*'\label{line:api_pipe2}'*

console.log(...mySequence);
// => Logs '52, 56, 60, 64, 68'

const mySequence2 = _.pipe(*'\label{line:api_pipe3}'*
  _.rejectAll(x => x % 2 === 1),
  _.map(x => 2 * x),
  _.retainAll(x => x > 50),
  _.take(5)
)(numbers); *'\label{line:api_pipe4}'*

console.log(...mySequence2);
// => Logs '52, 56, 60, 64, 68'
\end{lstlisting}

As listing~\ref{lst:api_pipe_proto} shows, using the version of |pipe| defined
on the prototype of the sequence, the readability improves even more:
\begin{lstlisting}[
  style=ES6,
  caption=Using pipe defined on the prototype,
  label={lst:api_pipe_proto}
]
const mySequence3 = seq.pipe(
  _.rejectAll(x => x % 2 === 1),
  _.map(x => 2 * x),
  _.retainAll(x => x > 50),
  _.take(5)
);
console.log(...mySequence3);
// => Logs '52, 56, 60, 64, 68'
\end{lstlisting}

% subsubsection Using the pipe Operator (end)
% section Operators (end)

\subsection{Terminal Operations} % (fold)
\label{sub:api_Terminal Operations}
Terminal operations are all functions that operate on an existing sequence
and do not necessarily create a new sequence. In other words, terminal
operations evaluate a sequence.\\ 
Table~\ref{tab:api_term_ops} gives an overview of all available terminal
operations.\\
Code examples and more information about the terminal operations delivers the
appendix~\ref{sub:appendix_terminal_operations}.

\begin{table}[H]
  \centering
  \begin{tabularx}{\textwidth}{| l | X |} \hline
    \textbf{Name} & \textbf{Description} \\ \hline
    \texttt{eq\$} & Checks the equality of two finite iterables. \\ \hline 
    \texttt{foldr} & Performs a reduction on the elements from right to left, using the provided start value and an accumulation function, and returns the reduced value. \\ \hline 
    \texttt{forEach\$} & Executes the callback for each element. \\ \hline 
    \texttt{head} & Return the next value without consuming it. \texttt{undefined} when there is no value. \\ \hline 
    \texttt{isEmpty} & Returns \texttt{true}, if the iterables head is undefined. \\ \hline 
    \texttt{max\$} & Returns the largest element of a \textbf{non-empty} \texttt{Iterable}. \\ \hline 
    \texttt{safeMax\$} & Returns the largest element of an \texttt{Iterable}. \\ \hline 
    \texttt{min\$} & Returns the smallest element of a \textbf{non-empty} \texttt{Iterable}. \\ \hline 
    \texttt{safeMin\$} & Returns the smallest element of an \texttt{Iterable}. \\ \hline 
    \texttt{foldl\$} & Performs a reduction on the elements, using the provided start value and an accumulation function, and returns the reduced value. \\ \hline 
    \texttt{show} & Transforms the passed \texttt{Iterable} to a \texttt{String}. \\ \hline 
    \texttt{uncons} & Removes the first element of this \texttt{Iterable}. \\ \hline 
  \end{tabularx}
  \caption{The available constructors of the Sequence library}
  \label{tab:api_term_ops}
\end{table}
% section Terminal Operations (end)

\subsection{Sequence Prototype} % (fold)
\label{sub:Sequence Prototype}
The prototype of the sequence provides some operations as well. This improves
code readability in some situations. The monadic operations are also available
on the prototype of the sequence. Thus, it is compatible with all functions
that expect a monad as a parameter.

\subsubsection{Verify the Prototype} % (fold)
\label{subsub:Verify the Prototype}
Listing~\ref{lst:proto_seq_query} shows how to query the prototype of any
object using the function |Object.getPrototypeOf|. Applying this to any
sequence will return |SequencePrototype|.

\begin{lstlisting}[
  style=ES6,
  caption=The Prototype of a \lstinline{Sequence},
  label={lst:proto_seq_query}
]
import { Sequence, SequencePrototype } from "sequence/sequence.js";

const seq          = Sequence(0, _ => true, i => i + 1);
const seqPrototype = Object.getPrototypeOf(seq);

console.log(seqPrototype === SequencePrototype);
// => Logs 'true'
\end{lstlisting}

% subsubsection Verify the Prototype (end)

\subsubsection{The operations on the Prototype} % (fold)
\label{subsub:The operations on the Prototype}
\begin{table}[H]
  \centering
  \begin{tabularx}{\textwidth}{| l | X |} \hline
    \textbf{Name} & \textbf{Description} \\ \hline
    \texttt{fmap} & the same as \lstinline{map} described in section~\ref{sub:api_Operators}\\ \hline 
    \texttt{and} & the same as \lstinline{bind} described in section~\ref{sub:api_Operators}\\ \hline 
    \texttt{pure} & the same as \lstinline{PureSequence} described in section~\ref{sub:api_Constructors}\\ \hline 
    \texttt{empty} & the same as \lstinline{nil} described in section~\ref{sub:api_Constructors}\\ \hline 
    \texttt{["=="]} & the same as \lstinline{eq$} described in section~\ref{sub:api_Terminal Operations}\\ \hline 
    \texttt{toString} & the same as \lstinline{show} described in section~\ref{sub:api_Terminal Operations}\\ \hline 
    \texttt{pipe} & the same as \lstinline{pipe} described in section~\ref{sub:api_Operators}\\ \hline 
  \end{tabularx}
  \caption{The operations served on the prototype of the \lstinline{Sequence}}
  \label{tab:prototype_operations}
\end{table}

% subsubsection The operations on the Prototype (end)
\subsubsection{When to use the operations on the Prototype?} % (fold)
\label{subsub:When to use the operations on the prototype?}
Each operation on the prototype is also available as an operator or terminal
operation. The prototype offers these functionalities additionally because
sometimes it has additional advantages - the function |toString| for example,
converts by convention an object into its string representation. Other
operations, such as |["=="]|, often make the code more readable. So when to use
the prototype operations? It depends on the context and code readability!
% subsubsection When to use the operations on the prototype? (end)
% subsection Sequence Prototype (end)
% section Sequence (end)



\section{Extension of the Kolibri Standard Library}
From this project's findings, some existing functionalities of the Kolibri
could also be improved or extended. This section describes those extensions.

\subsection{Extensions to Maybe} % (fold)
\label{sub:Extensions to Maybe}
The existing |Maybetype| has been extended to support monadic operations.
Analogous to the Sequence, |Maybe| has a prototype, providing the monadic
functions. \\ 
Listing~\ref{lst:monadic_ops_maybe_in_action} shows these monadic operations on
Maybe in action:

\begin{lstlisting}[
  style=ES6,
  caption=The monadic operations of Maybe in action,
  label={lst:monadic_ops_maybe_in_action}
]
import { Nothing, Just } from "stdlib/maybe.js";

/** Prints the value of a Maybe if it exists */
const evalMaybe = maybe =>
  maybe
    (_ => console.log("There was no value!"))
    (x => console.log(x));

const just    = Just(5);
const nothing = Nothing;

// fmap 
const justMapped    = just   .fmap(x => 2*x); // results in 10
const nothingMapped = nothing.fmap(x => 2*x); // Mapping Nothing - nothing happens!
evalMaybe(justMapped);      // => Logs '10'
evalMaybe(nothingMapped);   // => Logs 'There was no value!' 

// and
const justAnd    = just   .and(x => nothing); // Turns this value into Nothing
const nothingAnd = nothing.and(x => just);    // Can't change Nothing
evalMaybe(justAnd);         // => Logs 'There was no value!'
evalMaybe(nothingAnd);      // => Logs 'There was no value!' 

// pure
evalMaybe(just.pure(2));    // => Logs '2', Same as Just(2)
evalMaybe(nothing.pure(2)); // => Logs '2', Same as Just(2)

// empty
evalMaybe(just.empty());    // => Logs 'There was no value!', Same as Nothing
evalMaybe(nothing.empty()); // => Logs 'There was no value!', Same as Nothing
\end{lstlisting}
% subsection Extensions to Maybe (end)

\subsection{Extensions to Pair} % (fold)
\label{sub:Extensions to Pair}
Section~\ref{sec:Iterables Everywhere} describes the |Pair| as an immutable
data structure. To access its values simpler, it is now iterable.
Listing~\ref{lst:api_it_pair} shows these advantages:

\begin{lstlisting}[
  style=ES6,
  caption=Iterable Pair,
  label={lst:api_it_pair}
]
const pair = Pair(1)(2);
console.log(...pair);
// => Logs '1, 2'
\end{lstlisting}
% subsection Extensions to Pair (end)

\section{Sequence Builder}
\label{sec:Sequence Builder}
A mutable builder for a |SequenceType|.
\newline
|SequenceBuilder| allows the creation of a |SequenceType| by generating elements individually 
and adding them to the |SequenceBuilder| (without the call stack overhead when doing so
with |cons|).

\subsection{Functions}
\label{sub:sb_Functions}

|SequenceBuilder| provides the following functions:

\begin{table}[H]
  \centering
  \begin{tabularx}{\textwidth}{| l | X |} \hline
    \textbf{Name}    & \textbf{Description} \\ \hline
    \texttt{prepend} & Adds one or more elements to the beginning of the texttt{Sequence}\\ \hline 
    \texttt{append}  & Adds one or more elements to the end of the texttt{Sequence} \\ \hline 
    \texttt{build}   & The built phase and returns a texttt{SequenceType} which iterates over the added elements. \\ \hline 
   \end{tabularx}
  \caption{SequenceBuilder functions}
  \label{tab:sb_functions}
\end{table}

\subsection{Examples}
\label{sub:sb_Examples}
Creating a Sequence from different values and |Iterable|s:

\begin{lstlisting}[
  style=ES6, 
  caption=SequenceBuilder example,
  label={lst:sb_example}
  ]
import * as _              from "./src/sequence/sequence.js"
import { SequenceBuilder } from "./src/sequence/SequenceBuilder.js";
import { Range }           from "./src/range/range.js";

const range = Range(1, 3);

const result = SequenceBuilder()
.append(range)
.append(4)
.prepend(0)
.append(5,6,7)
.build();

console.log(_.show(result));
// => Logs '[0,1,2,3,4,5,6,7]'
\end{lstlisting}

\section{Query different Data Structures}
\label{sec:Query different Data Structures}
Based on the knowledge from section~\ref{chap:Monads in JavaScript}, other
possibilities analogous to |keepEven| become realizable - i.e., functions that
handle arbitrary monadic structures. \\ 
This section introduces such a concept called Language Integrated Queries
(LINQ), which queries any monadic data structure. It goes into detail
about the implementation of LINQ in JavaScript and shows how you can easily
browse lists of JavaScript objects using LINQ.
\subsection{Introduction to LINQ} % (fold)
\label{sub:Introduction to LINQ}
Some programming languages offer uniform ways to query different data
structures in an SQL-like syntax.

C\# calls this concept LINQ (Language Integrated Query). LINQ allows to
decarlatively query compatible data sources. Listing~\ref{lst:linq_in_csharp}
shows how to use LINQ to query a simple array of numeric values.

\begin{lstlisting}[
  style=sharpc,
  caption=LINQ in C\# \cite{billwagner_language-integrated_2023},
  label={lst:linq_in_csharp}
]
// Specify the data source.
int[] scores = { 97, 92, 81, 60 }; *'\label{line:linq_in_csharp1}'*

// Define the query expression.
IEnumerable<int> scoreQuery =
    from score in scores
    where score > 80
    select score;

// Execute the query.
foreach (int i in scoreQuery)
{
    Console.Write(i + " ");
}

// Output: 97 92 81
\end{lstlisting}

You can replace the data structure defined on line~\ref{line:linq_in_csharp1}
with any other one compatible with this API. This abstraction makes it very
easy to define reusable queries.


% subsection Introduction to LINQ (end)
\subsection{Why does this not exist in JavaScript?} % (fold)
\label{sub:Why does this not exist in JavaScript?}
However, JavaScript does not define a uniform API for data structures except
for the JS iteration protocols and thus cannot offer language-integrated queries
without further effort. \\
Section~\ref{sub:Making the Sequence Monadic} introduced the monadic sequence,
which provides additional operations to work with a sequence. All of these
operations are available through the sequence prototype. Since these
operations work solely on the committed properties of the JS iteration
protocols, the function |toMonadicIterable|, explained in
Section~\ref{sec:Iterables Everywhere} can quickly turn them into a monadic
sequence.\\
With that, a more extensive API is now available to every data structure
conforming to the JS iteration protocols. This monadic API makes it possible to
implement abstractions similar to LINQ in JavaScript.
% subsection Why does this not exist in JavaScript? (end)

\subsection{Introducing JINQ} % (fold)
\label{sub:Introducing JINQ}
JINQ (JavaScript integrated query) is the implementation of LINQ for
JavaScript. It can handle any data that conforms to the |MonadType| explained
in Section~\ref{subsub:The MonadType}. So JINQ can handle monadic iterables and
every monadic type, such as the type |Maybe| introduced in
Section~\ref{sub:Doing the same in JavaScript}.\\
\textit{Note:} All operations supported by JINQ are explained in detail in 
Section~\ref{sec:API_JINQ}. This sections descibres how JINQ works
internally.

Listing~\ref{lst:keepeven_recap} shows again the function |keepEven| already
introduced in Section~\ref{subsub:The MonadType}, which works on a |Maybe| as
well as on a |Sequence|:
\begin{lstlisting}[
  style=ES6,
  caption=Recapitulate keepEven,
  label={lst:keepeven_recap}
]
const keepEven = monad => monad
  .and(x => {
    if (x % 2 === 0) {
      return monad.pure(x);
    } else {
      return monad.empty();
    }
}); 
\end{lstlisting}

JINQ makes it possible to simplify the implementation of this function.
Listing~\ref{lst:keepeven_jinq} therefore introduces a new function
|keepEvenJINQ|, which does precisely the same as |keepEven|:

\begin{lstlisting}[
  style=ES6,
  caption=keepEvenJINQ does the same as keepEven,
  label={lst:keepeven_jinq}
]
const keepEvenJINQ = monad =>
  from(monad)
    .where(x => x % 2 === 0)
    .result();
\end{lstlisting}

|keepEvenJINQ| is not only shorter but also more readable! It becomes clear
within moments what this function does, as it reads almost like prose!\\
This is the power of these abstractions - types must only follow a
minimal API to be compatible with JINQ. Additionally, they are very readable.

\textit{Note:} As explained before, JINQ works only on the monadic API. So you
can just as well use the monadic functions to achieve the same. However, the
example |keepEven| shows that it is easier to work with JINQ and save lines of
code.

\subsubsection{How does JINQ work?} % (fold)
\label{subsub:How does JINQ work?}
JINQ uses a pattern analogue to the Builder pattern
\cite[Chapter~3.2]{gang_of_four_depa} to create a new structure which can be
used to transform the initially passed monad. \\
Have a look at the function on line~\ref{line:jinq_impl1} of
Listing~\ref{lst:jinq_impl}: If |from| is called, a new builder is created.
|from| expects a monad, which serves as the starting point of the builder. The
next operation executed on the builder operates on this monad. |result| then
returns the monad created based on the builder.

\textit{Note:} None of the functions change the parameter |monad| - therefore,
JINQ is immutable! This allows reusing an intermediate state of the JINQ
builder!

\begin{lstlisting}[
  style=ES6,
  caption=How is where implemented?,
  label={lst:jinq_impl}
]
// jinq.js
export { from }
const jinq = monad => ({ *'\label{line:jinq_impl1}'*
  pairWith: pairWith(monad),
  where:    where   (monad),
  select:   select  (monad),
  map:      select  (monad),
  inside:   inside  (monad),
  result:   () =>    monad
});

const from = jinq;*'\label{line:jinq_impl2}'*

// ...
\end{lstlisting}

Listing~\ref{lst:jinq_where_impl} uses the already familiar function |where| to
showcase how the builder operations work:

\begin{lstlisting}[
  style=ES6,
  caption=How is where implemented?,
  label={lst:jinq_where_impl}
]
// jinq.js
// ...

const where = monad => predicate => {
  const processed = 
    monad.and(a => predicate(a) ? monad.pure(a) : monad.empty()); *'\label{line:jinq_where_impl1}'*
  return jinq(processed);*'\label{line:jinq_where_impl2}'*
};

// ...
\end{lstlisting}

You may have already guessed it - the implementation of |where| is almost
precisely the implementation of |keepEven|! It is more general because the
predicate |x % 2 === 0| is outsourced to a function called |predicate|. See
line~\ref{line:jinq_where_impl1} in Listing~\ref{lst:jinq_where_impl}: |and|
keeps a value matching the predicate using |monad.pure| or discards it
otherwise using |monad.empty|.\\
Line~\ref{line:jinq_where_impl2} then wraps the resulting monad in a new
builder instance and returns it.

Another notable function of JINQ is |pairWith| - its implementation is shown in
Listing~\ref{lst:jinq_pairwith_impl}. Use it to combine a data source with
another one (or even with itself):

\begin{lstlisting}[
  style=ES6,
  caption=How is pairWith implemented?,
  label={lst:jinq_pairwith_impl}
]
// jinq.js
// ...

cconst pairWith = monad1 => monad2 => {
  const processed = monad1.and(x =>
    monad2.fmap(y => Pair(x)(y))
  );

  return jinq(processed)
};
// ...
\end{lstlisting}

|pairWith| takes a second monad. It now forms a new monad with a |Pair| in it!
But what happens here? Difficult to say because we can not know it! It just
calls |and| on |monad1| and combines it with |monad2|. \\
So when combining two sequences, every element of the first sequence gets
paired up with every element of the second sequence. One could argue that this
will take too much memory and could be more efficient. But let us step back and
think about Section 2.3, which discusses the laziness of sequences. |monad1|
(a sequence in the current example) is evaluated lazily! So never all
combinations will be materialized in memory at once!
% subsubsection How does JINQ work? (end)

\subsubsection{Creating Sequences using JINQ} % (fold)
\label{subsub:Creating Sequences using JINQ}
A list comprehension in Haskell is an expression form that allows generating
lists in a declarative way. See \cite[Chapter 5]{hutton_pih_2016} for an
in depth introduction to list comprehensions. \\
Listing~\ref{lst:list_comp_hs} creates a list using a list comprehension in
Haskell:

\begin{lstlisting}[
  style=Haskell,
  caption=List comprehension in Haskell,
  label={lst:list_comp_hs}
]
pairs = [(i,j) |        -- create a list of pairs where
          i <- [1..10], -- i can have the values 1 to 10
          j <- [1..4],  -- j can have the values 1 to 4
          i - j == 1]   -- only keep pairs with i - j == 1

print pairs
-- Prints '[(2,1),(3,2),(4,3),(5,4)]'
\end{lstlisting}

Using JINQ, it is possible to create sequences similarly:
\begin{lstlisting}[
  style=ES6,
  caption=Creating Sequences using JINQ,
  label={lst:list_comp_js}
]
const pairs =
  from(Range(1,10))                 // create a seq with values 1 to 10
    .pairWith(Range(1, 4))          // union with a seq containing 1 to 4
    .where(([i, j]) => i - j === 1) // only keep pairs with i - j == 1
    .result();

console.log(pairs.fmap(show).toString());
// => Logs '[[2,1],[3,2],[4,3],[5,4]]'
\end{lstlisting}

Of course, list comprehensions in Haskell provide more syntacitcal sugar than
JINQ. Nevertheless, JINQ provides an easy way to create sequences based on
rules, coming quite close to a list comprehension.
% subsubsection Creating Sequences using JINQ (end)

\subsubsection{Using the JSONMonad to process lists of JavaScript objects} % (fold)
\label{subsub:Using the JSONMonad to process lists of JavaScript objects}
The |JSONMonad| provides arrays of JavaScript objects with a monadic API. The
|JSONMonad| makes them, therefore, compatible with JINQ. This is especially
useful when JavaScript Objects are created from JSON because they often have
missing or nullable fields. The monadic operations of |JSONMonad| deal with
that and ensure that querying nullable attributes do not cause problems.

The Listing~\ref{lst:jsonmonad_data} dives right into an example defining two
records that are related to each other. The first one contains data about an
ancient battle, while the second one contains data about the heroes of the
battle. The |heroId| connects these two records. The battle had a |winner| and
a |loser|. Imagine you want to find all names of the heroes of the winner team.

\begin{lstlisting}[
  style=ES6,
  caption=Two data sources,
  label={lst:jsonmonad_data}
]
const battleData = JSON.parse(`
  {
    "battleName": "The battle of Curly",
    "numberOfDeaths": 420000,
    "winner": { "teamName": "JSON", "outStandingHeroes": [1] },
    "loser": { "teamName": "XML", "outStandingHeroes": [] }
  }
`);

const heroes = JSON.parse(`
  [
    { "heroId": 1, "kills": 47076, "name": "Atonadias" },
    { "heroId": 2, "kills": 5691,  "name": "Tanobiri" },
    { "heroId": 3, "kills": 3707,  "name": "Tonadri" }
  ]
`);
\end{lstlisting}

Listing~\ref{lst:jsonmonad_example} combines these two data sources using JINQ.
Line~\ref{line:jsonmonad_example1} wraps the |battleData| with the |JSONMonad|
to become queriable. |select| then accesses the property |winner| and from
there the property |outstandingHeroes|. Line~\ref{line:jsonmonad_example2}
pairs the second data source before Line~\ref{line:jsonmonad_example3}, then
destructures the |Pair| and only keeps the heroes (from the second data source)
that belong to the winning team (from the first data source).

\begin{lstlisting}[
  style=ES6,
  caption=Combining data sources using JINQ and JSONMonad,
  label={lst:jsonmonad_example}
]
const outstandingHeroNames =
  from(JsonMonad(battleData)) *'\label{line:jsonmonad_example1}'*
    .select   (x => x["winner"])
    .select   (x => x["outStandingHeroes"])
    .pairWith (JsonMonad(heroes)) *'\label{line:jsonmonad_example2}'*
    .where    (([heroId, hero]) => heroId === hero["heroId"]) *'\label{line:jsonmonad_example3}'*
    .select   (([_, hero]) => hero["name"])
    .result   ();

console.log(...outstandingHeroNames);
// => Los 'Atonadias'
\end{lstlisting}

% subsubsection Using the JSONMonad to process lists of JavaScript objects (end)
\subsubsection{Conclusion} % (fold)
\label{subsub:JINQ_Conclusion}
Monadic APIs allow the building of very general abstractions that can handle a
wide variety of data types, even if they seem to have nothing in common at
first glance - such as a sequence or a Maybe! JINQ showcases an excellent
instance of such a general abstraction - its operations only compose generic
monadic functions. Nevertheless, JINQs operations have very expressive names
describing their purpose, making it easy to use JINQ with different data types!
That is truly beautiful and quite rare: general concepts with specific
names!\\ 
JINQ shows its versatility through the possibility of creating new lists based
on declarative rules and using the |JSONMonad|.
% subsubsection Conclusion (end)
% subsection Introducing JINQ (end)
% section Why does this not exist in JavaScript? (end)

\section{How to Extend the Library}
\label{sec:How to Extend the Library}
This Chapter gives a quick start to extending the Sequence library. Because
applications are versatile, you may want to add further functionality to the
Library. Therefore, this Chapter explains the relevant steps to simplify the entry.
\newline
Whether you want to add a constructor, an operator, or a terminal operation is
similar. Below is an example of an operator. You can adapt the procedure for
the others.

\subsection{Adding a new Operator}
\label{sub:Adding a new Operator}
In this scenario, we include |concat| to the Sequence library. |concat| is the
operation that appends one |Iterable| to another.

\subsubsection{Kind of the Operation}
The library distinguishes between operators and terminal operations. The
difference depends on the return type. If a function returns an |Iterable|, it
is an operator. It is a terminal operation if it produces a different type than an
|Iterable|, such as |isEmpty|.
Since |concat| returns an |Iterable|, we include the related files into the
operations folder.

\subsubsection{Directory Structure}
\label{subsub:Directory Structure}
|concat| has its folder in the operations directory. Therefore, we create a
|concat.js| and a |concatTest.js| file in the folder |concat|. The reasoning for 
splitting tests into a separate file is that it leads to a better overview and faster
navigation between the particular functionalities. Figure~\ref{fig:concat_dir}
shows the relevant directories.

\begin{figure}[H]
\dirtree{%
  .1 src .
  .2 sequence .
  .3 operators .
  .4 concat .
  .5 concat.js .
  .5 concatTest.js .
  .4 \ldots .
  .4 operators.js .
  .4 operatorsTest.js .
  .3 \ldots .
  .2 AllTests.html .
  .2 \ldots .
  }
  \caption{concat directory structure}
  \label{fig:concat_dir}
\end{figure}

\subsubsection{Exports and Imports}
All artifacts of the Sequence library are available via the |sequence.js| file.
To make |concat| support this, we export it via the |operators.js| file.
Listings~\ref{lst:concat_export} and \ref{lst:concat_export_operators} shows the corresponding statements.

\begin{lstlisting}[
  style=ES6, 
  caption=Export of concat,
  label={lst:concat_export}
  ]
// concat.js

export { concat }
\end{lstlisting}



\begin{lstlisting}[
  style=ES6, 
  caption=Export of concat in operators.js,
  label={lst:concat_export_operators}
  ]
// operators.js
...
export * from "./concat/concat.js"
\end{lstlisting}

The same principle applies to the test files. Importing the |concatTest.js| in
|operatorsTest.js| enables to run of the test cases.

\begin{lstlisting}[
  style=ES6, 
  caption=Export of concat in operatorTest.js,
  label={lst:concat_import_operators}
  ]
// operatorsTest.js
...
import "./concat/concat.js"
\end{lstlisting}

\subsection{Implementing a new Operator}
\label{subsub:Implementing a new Operator}
Now we're getting down to the nitty-gritty.
Because great software engineers start with the tests, we do that first.

\subsubsection{Write the Tests}
\label{subsub:Write the Tests}
Chapter~\ref{sec:Testing} has already explained testing in detail. Therefore, we will only deal 
with the practical part here.
\newline
Let us start with the imports. For testing reasons, we need the following
functionalities:

\begin{itemize}
  \item{TestSuite}
  \item{addToTestingTable}
  \item{createTestConfig}
\end{itemize}

\textit{Note:} You may need additional imports to implement the tests itself.
\newline
Now you are ready to create a test config and, optionally, some special cases.
Creating a test config requires the following steps:

\begin{itemize}
  \item{Create a |TestSuite| with a meaningful name corresponding to the function you are implementing.}
  \item{|createTestConfig| expects an object of type |SequenceTestConfigType|. Have a look
    to this type or to Chapter~\ref{subsub:The Configuration} to get the information about the available properties. }
      \item{If neccessary, add some further test cases using the same TestSuite.}
  \item{Run the |TestSuite| by calling the |run()| function at the end of the file. }
\end{itemize}

Listing~\ref{lst:concat_test_import} shows a scaffold of a test file. 

\begin{lstlisting}[
  style=ES6, 
  caption=Imports of concatTest.js,
  label={lst:concat_test_import}
  ]
import { addToTestingTable }  from "./src/sequence/util/testingTable.js";
import { TestSuite }          from "./src/sequence/test/test.js";
import { createTestConfig }   from "./src/sequence/util/testUtil.js";
import { concat }             from "./src/sequence/sequence.js";
...

const testSuite = TestSuite("Name of the TestSuite");

addToTestingTable(testSuite)(
  createTestConfig({
      ...
    })
);

testSuite.add("test special case", assert => {
  // Given
  ...
  // When
  ...
  // Then
  ...
});

testSuite.run();
\end{lstlisting}

Run the |AllTests.html| file, and you should see that your tests are failing.
If so, then everything is fine.

\textit{Note:} While running the tests, always observe the console.

\subsubsection{Write the Functionality}
Now you are in a comfortable position to implement the function against tests.
For this, we create a function |concat| in the file |concat.js|.
Probably, you can use existing functionalities of the Sequence library to implement them. A list of all
provided functionality is in Chapter~\ref{sub:api_Constructors} - \ref{sub:api_Terminal Operations}.
\newline
Again, run the tests and be proud if everything is running!

