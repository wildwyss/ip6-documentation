\section{How to Extend the Library}
\label{sec:How to Extend the Library}
This Chapter gives a quick start to extending the Sequence library. Because
applications are versatile, you may want to add further functionality to the
Library. Therefore, this Chapter explains the relevant steps to simplify the entry.
\newline
Whether you want to add a constructor, an operator, or a terminal operation is
similar. Below is an example of an operator. You can adapt the procedure for
the others.

\subsection{Adding a new Operator}
\label{sub:Adding a new Operator}
In this scenario, we include |concat| to the Sequence library. |concat| is the
operation that appends one |Iterable| to another.

\subsubsection{Kind of the Operation}
\label{subsub:Kind of the Operation}
The library distinguishes between operators and terminal operations. The
difference depends on the return type. If a function returns an |Iterable|, it
is an operator. It is a terminal operation if it produces a different type than an
|Iterable|, such as |isEmpty|.
Since |concat| returns an |Iterable|, we include the related files into the
operations folder.

\subsubsection{Directory Structure}
\label{subsub:Directory Structure}
|concat| has its folder in the operations directory. Therefore, we create a
|concat.js| and a |concatTest.js| file in the folder |concat|. The reasoning for 
splitting tests into a separate file is that it leads to a better overview and faster
navigation between the particular functionalities. Figure~\ref{fig:concat_dir}
shows the relevant directories.

\begin{figure}[H]
\dirtree{%
  .1 src .
  .2 sequence .
  .3 operators .
  .4 concat .
  .5 concat.js .
  .5 concatTest.js .
  .4 \ldots .
  .4 operators.js .
  .4 operatorsTest.js .
  .3 \ldots .
  .2 AllTests.html .
  .2 \ldots .
  }
  \caption{concat directory structure}
  \label{fig:concat_dir}
\end{figure}

\subsubsection{Exports and Imports}
\label{subsub:Exports and Imports}
All artifacts of the Sequence library are available via the |sequence.js| file.
To make |concat| support this, we export it via the |operators.js| file.
Listings~\ref{lst:concat_export} and \ref{lst:concat_export_operators} shows the corresponding statements.

\begin{lstlisting}[
  style=ES6, 
  caption=Export of concat,
  label={lst:concat_export}
  ]
// concat.js

export { concat }
\end{lstlisting}



\begin{lstlisting}[
  style=ES6, 
  caption=Export of concat in operators.js,
  label={lst:concat_export_operators}
  ]
// operators.js
...
export * from "./concat/concat.js"
\end{lstlisting}

The same principle applies to the test files. Importing the |concatTest.js| in
|operatorsTest.js| enables to run of the test cases.

\begin{lstlisting}[
  style=ES6, 
  caption=Export of concat in operatorTest.js,
  label={lst:concat_import_operators}
  ]
// operatorsTest.js
...
import "./concat/concat.js"
\end{lstlisting}

\subsection{Implementing a new Operator}
\label{subsub:Implementing a new Operator}
Now we're getting down to the nitty-gritty.
Because great software engineers start with the tests, we do that first.

\subsubsection{Write the Tests}
\label{subsub:Write the Tests}
Chapter~\ref{chap:Effective_Testing} has already explained testing in detail. Therefore, we will only deal 
with the practical part here.
\newline
Let us start with the imports. For testing reasons, we need the following
functionalities:

\begin{itemize}
  \item{TestSuite}
  \item{addToTestingTable}
  \item{createTestConfig}
\end{itemize}

\textit{Note:} You may need additional imports to implement the tests itself.
\newline
Now you are ready to create a test config and, optionally, some special cases.
Creating a test config requires the following steps:

\begin{itemize}
  \item{Create a |TestSuite| with a meaningful name corresponding to the function you are implementing.}
  \item{|createTestConfig| expects an object of type |SequenceTestConfigType|. Have a look
    to this type or to section~\ref{sub:Configuring the Testing Table} to get the information about the available properties. }
      \item{If neccessary, add some further test cases using the same TestSuite.}
  \item{Run the |TestSuite| by calling the |run()| function at the end of the file. }
\end{itemize}

Listing~\ref{lst:concat_test_import} shows a scaffold of a test file. 

\begin{lstlisting}[
  style=ES6, 
  caption=Imports of concatTest.js,
  label={lst:concat_test_import}
  ]
import { addToTestingTable }  from "./src/sequence/util/testingTable.js";
import { TestSuite }          from "./src/sequence/test/test.js";
import { createTestConfig }   from "./src/sequence/util/testUtil.js";
import { concat }             from "./src/sequence/sequence.js";
...

const testSuite = TestSuite("Name of the TestSuite");

addToTestingTable(testSuite)(
  createTestConfig({
      ...
    })
);

testSuite.add("test special case", assert => {
  // Given
  ...
  // When
  ...
  // Then
  ...
});

testSuite.run();
\end{lstlisting}

Run the |AllTests.html| file, and you should see that your tests are failing.
If so, then everything is fine.

\textit{Note:} While running the tests, always observe the console.

\subsubsection{Write the Functionality}
\label{subsub:Write the Functionality}
Now you are in a comfortable position to implement the function against tests.
For this, we create a function |concat| in the file |concat.js|.
Probably, you can use existing functionalities of the Sequence library to implement them. A list of all
provided functionality is in Chapter~\ref{sub:api_Constructors} - \ref{sub:api_Terminal Operations}.
\newline
Again, run the tests and be proud if everything is running!
