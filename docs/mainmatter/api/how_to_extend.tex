\section{How to Extend the Library}
\label{sec:How to Extend the Library}
Since applications are versatile, you can add further functionality
to the library. This section shows how to extend the Sequence library.
\subsection{Adding a new Operator}
\label{sub:Adding a new Operator}
In this scenario, we include |concat| to the Sequence library. |concat| is the
operation that appends one iterable to another.

\subsubsection{Kind of the Operation}
\label{subsub:Kind of the Operation}
The Sequence library distinguishes between operators and terminal operations. The
difference depends on the return type. If a function takes a sequence and
returns a sequence, it is an operator. \\ 
It is a terminal operation if it takes a sequence and produces a different type
than a sequence, such as |isEmpty|.
\newline
\textit{Note:} It doesn't matter if you want to add a constructor, an operator, or a terminal
operation to the sequence library - the procedure is always the same. The
following example can, therefore, also be applied to constructors and terminal
operations.

\subsubsection{Directory Structure}
\label{subsub:Directory Structure}
Each operation of the Sequence library consists of two files: a test file and
an implementation file. Therefore, we create a |concat.js| and a
|concatTest.js| file in a new folder |concat| in the existing folder
|operators|. This helps keep overview and finding desired files faster.
Figure~\ref{fig:concat_dir} shows the relevant directories.

\begin{figure}[H]
\dirtree{%
  .1 src .
  .2 sequence .
  .3 operators .
  .4 concat .
  .5 concat.js .
  .5 concatTest.js .
  .4 \ldots .
  .4 operators.js .
  .4 operatorsTest.js .
  .3 \ldots .
  .2 AllTests.html .
  .2 \ldots .
  }
  \caption{concat directory structure}
  \label{fig:concat_dir}
\end{figure}

\subsubsection{Exports and Imports}
\label{subsub:Exports and Imports}
All artefacts of the Sequence library are available via the |sequence.js| file.
To make |concat| support this, we export it via the |operators.js| file.
Listings~\ref{lst:concat_export} and \ref{lst:concat_export_operators} show the
corresponding statements:

\begin{lstlisting}[
  style=ES6, 
  caption=Export of concat,
  label={lst:concat_export}
  ]
// concat.js

export { concat }
\end{lstlisting}



\begin{lstlisting}[
  style=ES6, 
  caption=Export of concat in operators.js,
  label={lst:concat_export_operators}
  ]
// operators.js
...
export * from "./concat/concat.js"
\end{lstlisting}

The same principle applies to the test files. Importing the |concatTest.js| in
|operatorsTest.js| enables to run the test cases.

\begin{lstlisting}[
  style=ES6, 
  caption=Export of concat in operatorTest.js,
  label={lst:concat_import_operators}
  ]
// operatorsTest.js
...
import "./concat/concat.js"
\end{lstlisting}

\subsection{Implementing a new Operator}
\label{subsub:Implementing a new Operator}

\subsubsection{Writing the Tests}
\label{subsub:Write the Tests}
The first user of a new operator of the Sequence library is usually a test!
Therefore, we start by writing the tests.\\
\textit{Note:} Chapter~\ref{chap:Effective_Testing} explains testing of the
Sequence library in detail. This sections therefore does not cover how this
exactly works.
\newline
Let us start with the imports. A test needs the following dependencies:

\begin{itemize}
  \item{TestSuite}
  \item{addToTestingTable}
  \item{createTestConfig}
\end{itemize}
Now you are ready to create a test config and, optionally, some special cases.
Creating a test config requires the following steps:

\begin{itemize}
  \item{Create a |TestSuite| with a meaningful name corresponding to the function you are implementing.}
  \item{|createTestConfig| expects an object of type |SequenceTestConfigType|. Have a look
    to this type or to section~\ref{sub:Configuring the Testing Table} to get the information about the available properties. }
      \item{If necessary, add some further test cases using the same TestSuite.}
  \item{Run the |TestSuite| by calling the |run()| function at the end of the file. }
\end{itemize}

Listing~\ref{lst:concat_test_import} shows a scaffold of a test file. 

\begin{lstlisting}[
  style=ES6, 
  caption=Imports of concatTest.js,
  label={lst:concat_test_import}
  ]
import { addToTestingTable }  from "./src/sequence/util/testingTable.js";
import { TestSuite }          from "./src/sequence/test/test.js";
import { createTestConfig }   from "./src/sequence/util/testUtil.js";
import { concat }             from "./src/sequence/sequence.js";
...

const testSuite = TestSuite("Name of the TestSuite");

addToTestingTable(testSuite)(
  createTestConfig({
      ...
    })
);

testSuite.add("test special case", assert => {
  // Given
  ...
  // When
  ...
  // Then
  ...
});

testSuite.run();
\end{lstlisting}

Run the |AllTests.html| file, and you should see that your tests are failing.
If so, then everything is fine.

\textit{Note:} While running the tests, always observe the console.

\subsubsection{Writing the Functionality}
\label{subsub:Write the Functionality}
Now you are in a comfortable position to implement your code against tests.
For this, create a function |concat| in the file |concat.js|!
Probably, you can use existing functionalities of the Sequence library for your
implementation. Sections~\ref{sub:api_Constructors} - \ref{sub:api_Terminal
Operations} provide a list of all provided functionality of the Sequence
library.
\newline
Again, run the tests and be proud if everything is running!
