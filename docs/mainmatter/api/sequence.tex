\section{Sequence Library} % (fold)
A Sequence is a lazy data structure. This section overviews which operations
exist to create and modify sequences and how to import this library into your
own project. An in-depth view of how a Sequence works
internally gives Chapter~\ref{sec:Sequence and Iterable in General}.

\subsection{Getting Started} % (fold)
\label{sub:Getting Started}
\subsubsection{Importing the Sequence Library} % (fold)
\label{sec:Importing the Sequence Library}

% subsubsection Importing the Sequence Library (end)
The module |sequence/sequence.js| exports the whole sequence library. The
easiest way to use it is named imports. As
Listing~\ref{lst:named_imports_seq_lib_example} shows, this approach allows
using the dot notation, which enables the IDE to support the development with
auto-completion. Developers who have been using the library longer can import
individual functions directly.
\begin{lstlisting}[
  style=ES6,
  caption=Importing the Sequence library using named imports,
  label={lst:named_imports_seq_lib_example}
]
import * as _ from "./sequence/sequence.js"

const seq = _.Sequence(0, i => i < 5, i => i + 1);

console.log(_.show(seq));
// => Logs '[0, 1, 2, 3, 4]'
\end{lstlisting}
\textit{Note:} Of course, importing the Sequence library with any name is
possible. The advantage of |_| is that it is short and takes up very little
white space. Other special characters, such as the \lstinline{$}, negatively
affect the code's readability. 
% subsection Importing the Sequence Library (end)

\subsubsection{Evaluating endless Sequences} % (fold)
\label{sec:Evaluating endless Sequences}
Section~\ref{Lazy Evaluate Iterables} describes the benefits of lazy
evaluation, which allows working with sequences consisting of infinite
elements. When working with such sequences, only evaluate parts of the sequence
because evaluating endless values would take forever. Functions of the Sequence
library, which evaluate a whole sequence, are therefore marked with a
\lstinline{$} symbol in the name. These functions are only applicable to
sequences with a finite number of elements. \\
Listing~\ref{lst:api_endl_seq_eval} shows an example for that:

\begin{lstlisting}[
  style=ES6,
  caption=Evaluating a part of an endless Sequence,
  label={lst:api_endl_seq_eval}
]
import * as _ from "./sequence/sequence.js"

const seq      = _.Sequence(0, _ => true, x => x + 1); *'\label{line:api_endl_seq_eval1}'*
const part     = _.take(5)(seq); *'\label{line:api_endl_seq_eval2}'*

const reversed = _.reverse$(part);

console.log(...reversed);
// => Logs '4, 3, 2, 1, 0'
\end{lstlisting}

Line~\ref{line:api_endl_seq_eval1} creates a Sequence |seq| which contains
infinite elements. Evaluating this would go on forever. Therefore,
Line~\ref{line:api_endl_seq_eval2} uses the function |_.take(5)| which only
evaluates |5| elements.

% subsubsection Conventions (end)
\label{sec:api_sequences}
\subsection{Constructors} % (fold)
\label{sub:api_Constructors}
A constructor is anything that builds a |Sequence| without depending on another
one. So they serve as an entry point to the Sequence library. Some of these
constructors create a specific series of values, including, for example, the
|PrimeNumberSequence| which yields the infinite sequence of all prime
numbers.\\
Table~\ref{tab:api_ctors} gives an overview of all available constructors. Code
examples and more information about the constructors deliver the
appendices~\ref{sub:appendix_constructors} and
\ref{sub:appendix_special_constructors}.

\begin{table}[H]
  \centering
  \begin{tabularx}{\textwidth}{| l | X |} \hline
    \textbf{Name} & \textbf{Description} \\ \hline
    \texttt{Sequence} & Creates a new Sequence based on a start \texttt{value}, \texttt{incrFn} and \texttt{stopFn}. \\ \hline
    \texttt{PureSequence} & Creates a Sequence which contains just the given value. \\ \hline
    \texttt{repeat} & Returns a Sequene that will repeatedly yield the value of \texttt{arg} when iterated over. \texttt{repeat} will never be exhausted. \\ \hline
    \texttt{replicate} & \texttt{replicate(n)(x)} creates a Sequence of length \texttt{n} with \texttt{x} the value of every element. \\ \hline
    \texttt{StackSequence} & Creates a \texttt{SequenceType} on top of the given \texttt{stack}. \\ \hline
    \texttt{TupleSequence} & Constructs a new \texttt{SequenceType} based on the given tuple. \\ \hline
    \texttt{AngleSequence} & Creates a Sequence which generates evenly spaced angles between 0 and 360. \\ \hline
    \texttt{FibonacciSequence} & Generates the Fibonacci sequence. \\  \hline
    \texttt{PrimeNumberSequence} & Generates the sequence of prime numbers. \\ \hline
    \texttt{SquareNumberSequence} & Generates the sequence of square numbers. \\ \hline
    \texttt{Range} & Generates a sequence of numbers. \cite{wild_ip5_2023} \\ \hline
  \end{tabularx}
  \caption{The available constructors in the Sequence library}
  \label{tab:api_ctors}
\end{table}
% subsection Constructors (end)
\subsection{Operators} % (fold)
\label{sub:api_Operators}
Operators are all functions that operate on an existing |Sequence| and create a
new |Sequence| from it. \\
Table~\ref{tab:api_operators} gives an overview of all available operators.\\
Code examples and more information about the operators delivers the
appendix~\ref{sub:appendix_operators}.
\begin{table}[H]
  \centering
  \begin{tabularx}{\textwidth}{| l | X |} \hline
    \textbf{Name} & \textbf{Description} \\ \hline
    \texttt{bind} & Applies the given function to each element of the \texttt{Iterable} and flats it afterward. Note This operation adds a monadic API to the \texttt{SequenceType}. \\ \hline 
    \texttt{catMaybes} & The catMaybes function takes an \texttt{Iterable} of \texttt{MaybeType}s and returns a \texttt{SequenceType} of all the Just's values. \\ \hline 
    \texttt{concat} & Adds the second iterable to the first iterables end. \\ \hline 
    \texttt{cons} & Adds the given element to the front of an iterable. \\ \hline 
    \texttt{cycle} & Ties a finite \texttt{Iterable} into a circular one, or equivalently, the infinite repetition of the original \texttt{Iterable}. \\ \hline 
    \texttt{drop} & Jumps over so many elements. \\ \hline 
    \texttt{dropWhile} & Discards all elements until the first element does not satisfy the predicate anymore. \\ \hline 
    \texttt{map} & Transforms each element using the given function. \\ \hline 
    \texttt{mconcat} & Flatten an \texttt{Iterable} of \texttt{Iterable}s. \\ \hline 
    \texttt{pipe} & Transforms the given \texttt{Iterable} using the passed operators. \\ \hline 
    \texttt{rejectAll} & Only keeps elements which does not fulfill the given predicate. \\ \hline 
    \texttt{retainAll} & Only keeps elements which fulfill the given predicate. \\ \hline 
    \texttt{reverse\$} & Processes the \texttt{Iterable} backwards. \\ \hline 
    \texttt{snoc} & Adds the given element to the end of the \texttt{Iterable}. \\ \hline 
    \texttt{take} & Stop after so many elements. \\ \hline 
    \texttt{takeWhile} & Proceeds with the iteration until the predicate becomes true. \\ \hline 
    \texttt{zip} & Zip takes two \texttt{Iterable}s and returns an \texttt{Iterable} of corresponding \texttt{Pair}s. \\ \hline 
    \texttt{zipWith} & Generalises \texttt{zip} by zipping with the function given as the first argument. \\ \hline 
  \end{tabularx}
  \caption{The available operators in the Sequence library}
  \label{tab:api_operators}
\end{table}
% section Operators (end)

\subsection{Terminal Operations} % (fold)
\label{sub:api_Terminal Operations}
Terminal operations are all functions that operate on an existing |Sequence|
and do not necessarily create a new |Sequence| from it. In other words,
terminal operations evaluate a |Sequence|.\\ 
Table~\ref{tab:api_term_ops} gives an overview of all available terminal
operations.\\
Code examples and more information about the constructors delivers the
appendix~\ref{sub:appendix_terminal_operations}.

\begin{table}[H]
  \centering
  \begin{tabularx}{\textwidth}{| l | X |} \hline
    \textbf{Name} & \textbf{Description} \\ \hline
    \texttt{eq\$} & Checks the equality of two finite \texttt{Iterable}s. \\ \hline 
    \texttt{foldr} & Performs a reduction on the elements from right to left, using the provided start value and an accumulation function, and returns the reduced value. \\ \hline 
    \texttt{forEach\$} & Executes the callback for each element. \\ \hline 
    \texttt{head} & Return the next value without consuming it. \texttt{undefined} when there is no value. \\ \hline 
    \texttt{isEmpty} & Returns \texttt{true}, if the iterables head is undefined. \\ \hline 
    \texttt{max\$} & Returns the largest element of a \textbf{non-empty} \texttt{Iterable}. \\ \hline 
    \texttt{safeMax\$} & Returns the largest element of an \texttt{Iterable}. \\ \hline 
    \texttt{min\$} & Returns the smallest element of a \textbf{non-empty} \texttt{Iterable}. \\ \hline 
    \texttt{safeMin\$} & Returns the smallest element of an \texttt{Iterable}. \\ \hline 
    \texttt{foldl\$} & Performs a reduction on the elements, using the provided start value and an accumulation function, and returns the reduced value. \\ \hline 
    \texttt{show} & Transforms the passed \texttt{Iterable} to a \texttt{String}. \\ \hline 
    \texttt{uncons} & Removes the first element of this iterable. \\ \hline 
  \end{tabularx}
  \caption{The available constructors of the Sequence library}
  \label{tab:api_term_ops}
\end{table}
% section Terminal Operations (end)
\subsection{Sequence Prototype} % (fold)
\label{sub:Sequence Prototype}
The prototype of the Sequence provides some operations as well. This improves
code readability in some situations. The monadic operations are also available
on the prototype of the Sequence. Thus, it is compatible with all functions
that expect a monad as a parameter.

\subsubsection{Verify the Prototype} % (fold)
\label{subsub:Verify the Prototype}
Listing 3 shows how to query the prototype of any object using the function
|Object.getPrototypeOf|. Applying this to any sequence will return
|SequencePrototype|.


\begin{lstlisting}[
  style=ES6,
  caption=The Prototype of a \lstinline{Sequence},
  label={lst:proto_seq_query}
]
import { Sequence, SequencePrototype } from "sequence/sequence.js";

const seq          = Sequence(0, _ => true, i => i + 1);
const seqPrototype = Object.getPrototypeOf(seq);

console.log(seqPrototype === SequencePrototype);
// => Logs 'true'
\end{lstlisting}

% subsubsection Verify the Prototype (end)

\subsubsection{The operations on the Prototype} % (fold)
\label{subsub:The operations on the Prototype}
\begin{table}[H]
  \centering
  \begin{tabularx}{\textwidth}{| l | X |} \hline
    \textbf{Name} & \textbf{Description} \\ \hline
    \texttt{fmap} & Is the same as \lstinline{map} described in Section~\ref{sub:api_Operators}\\ \hline 
    \texttt{and} & Is the same as \lstinline{bind} described in Section~\ref{sub:api_Operators}\\ \hline 
    \texttt{pure} & Is the same as \lstinline{PureSequence} described in Section~\ref{sub:api_Constructors}\\ \hline 
    \texttt{empty} & Is the same as \lstinline{nil} described in Section~\ref{sub:api_Constructors}\\ \hline 
    \texttt{["=="]} & Is the same as \lstinline{eq$} described in Section~\ref{sub:api_Terminal Operations}\\ \hline 
    \texttt{toString} & Is the same as \lstinline{show} described in Section~\ref{sub:api_Terminal Operations}\\ \hline 
    \texttt{pipe} & Is the same as \lstinline{pipe} described in Section~\ref{sub:api_Operators}\\ \hline 
  \end{tabularx}
  \caption{The operations served on the prototype of the \lstinline{Sequence}}
  \label{tab:api_term_ops}
\end{table}

% subsubsection The operations on the Prototype (end)
% subsection Sequence Prototype (end)
% section Sequence (end)


