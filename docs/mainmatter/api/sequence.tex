\section{Sequence Library} % (fold)
\label{sec:Sequence Library}
This section overviews which operations exist to create and process sequences
and how to import this library into your own project.
\subsection{Getting Started} % (fold)
\label{sub:Getting Started}
\subsubsection{Importing the Sequence Library} % (fold)
\label{subsub:Importing the Sequence Library}

The module |sequence/sequence.js| exports the whole Sequence library. The
easiest way to use it is through named imports. As
listing~\ref{lst:named_imports_seq_lib_example} shows, this approach allows
using the dot notation, which enables the IDE to support the development with
auto-completion. Developers who have been using the library longer can import
individual functions directly.
\begin{lstlisting}[
  style=ES6,
  caption=Importing the Sequence library using named imports,
  label={lst:named_imports_seq_lib_example}
]
import * as _ from "./sequence/sequence.js"

const seq = _.Sequence(0, i => i < 5, i => i + 1);

console.log(_.show(seq));
// => Logs '[0, 1, 2, 3, 4]'
\end{lstlisting}
\textit{Note:} Of course, importing the Sequence library with any name is
possible. The advantage of |_| is that it is short and takes up very little
white space. Other special characters, such as the \lstinline{$}, negatively
affect the code's readability. 
% subsubsection Importing the Sequence Library (end)
% subsection Importing the Sequence Library (end)

\subsubsection{Evaluating endless Sequences} % (fold)
\label{subsub:Evaluating endless Sequences}
Section~\ref{subsub:Lazy Evaluate Iterables} describes the benefits of lazy
evaluation, which allows working with sequences consisting of infinite
elements. When working with such sequences, pay attention to only evaluating
parts of the sequence because evaluating endless values would take forever.
Functions of the sequence library, which will process the whole sequence, are
therefore marked with a \lstinline{$} symbol in the name. These functions are
only applicable to sequences with a finite number of elements. \\
Listing~\ref{lst:api_endl_seq_eval} shows an example of that:

\begin{lstlisting}[
  style=ES6,
  caption=Evaluating a part of an endless Sequence,
  label={lst:api_endl_seq_eval}
]
import * as _ from "./sequence/sequence.js"

const seq      = _.Sequence(0, _ => true, x => x + 1); *'\label{line:api_endl_seq_eval1}'*
const part     = _.take(5)(seq); *'\label{line:api_endl_seq_eval2}'*

const reversed = _.reverse$(part);

console.log(...reversed);
// => Logs '4, 3, 2, 1, 0'
\end{lstlisting}

Line~\ref{line:api_endl_seq_eval1} creates a sequence |seq| which contains
infinite elements. Evaluating this would go on forever. Therefore,
Line~\ref{line:api_endl_seq_eval2} uses the function |_.take(5)| which only
evaluates |5| elements.

% subsubsection Conventions (end)
\subsection{Constructors} % (fold)
\label{sub:api_Constructors}
A constructor is anything that creates a sequence without depending on another
one. So they serve as an entry point to the Sequence library. \\
Table~\ref{tab:api_ctors} gives an overview of all available constructors. Code
examples and more information about the constructors delivers the
appendix~\ref{sec:appendix_ctor}.

\begin{table}[H]
  \centering
  \begin{tabularx}{\textwidth}{| l | X |} \hline
    \textbf{Name} & \textbf{Description} \\ \hline
    \texttt{nil} & This constant represents a sequence containing no values. \\ \hline
    \texttt{PureSequence} & Creates a sequence which contains just the given value. \\ \hline
    \texttt{Range} & Generates a sequence of numbers. \cite{wild_ip5_2023} \\ \hline
    \texttt{repeat} & Returns a sequence that will repeatedly yield the value of \texttt{arg} when iterated over. \texttt{repeat} will never be exhausted. \\ \hline
    \texttt{replicate} & \texttt{replicate(n)(x)} creates a sequence of length \texttt{n} with \texttt{x} the value of every element. \\ \hline
    \texttt{Sequence} & Creates a new sequence based on a start \texttt{value}, \texttt{whileFunction} and \texttt{incrementFunction}. \\ \hline
    \texttt{StackSequence} & Creates a \texttt{SequenceType} on top of the given \texttt{stack}. \\ \hline
    \texttt{TupleSequence} & Constructs a new \texttt{SequenceType} based on the given tuple. \\ \hline
  \end{tabularx}
  \caption{The available constructors in the Sequence library}
  \label{tab:api_ctors}
\end{table}
% subsection Constructors (end)

\subsection{Operators} % (fold)
\label{sub:api_Operators}
Operators are functions that operate on any existing iterable and create a
new sequence from it. \\
Table~\ref{tab:api_operators} gives an overview of all available operators.\\
Code examples, types and more information about the operators deliver the
appendix~\ref{sec:appendix_operators}.
\begin{table}[H]
  \centering
  \begin{tabularx}{\textwidth}{| l | X |} \hline
    \textbf{Name} & \textbf{Description} \\ \hline
    \texttt{bind} & Applies the given function to each element of the \texttt{Iterable} and flattens it afterward. Note This operation adds a monadic API to the \texttt{SequenceType}. \\ \hline 
    \texttt{catMaybes} & The catMaybes function takes an \texttt{Iterable} of \texttt{MaybeType}s and returns a \texttt{SequenceType} of all the Just's values. \\ \hline 
    \texttt{concat} & Adds the second \texttt{Iterable} to the first iterables end. \\ \hline 
    \texttt{cons} & Adds the given element to the front of an \texttt{Iterable}. \\ \hline 
    \texttt{cycle} & Ties a finite \texttt{Iterable} into a circular one, or equivalently, the infinite repetition of the original \texttt{Iterable}. \\ \hline 
    \texttt{drop} & Jumps over so many elements. \\ \hline 
    \texttt{dropWhile} & Discards all elements until the first element does not satisfy the predicate any more. \\ \hline 
    \texttt{map} & Transforms each element using the given function. \\ \hline 
    \texttt{mconcat} & Flatten an \texttt{Iterable} of iterables. \\ \hline 
    \texttt{pipe} & Transforms the given \texttt{Iterable} using the passed operators. \\ \hline 
    \texttt{rejectAll} & Only keeps elements which do not fulfil the given predicate. \\ \hline 
    \texttt{retainAll} & Only keeps elements which fulfil the given predicate. \\ \hline 
    \texttt{reverse\$} & Processes the \texttt{Iterable} backwards. \\ \hline 
    \texttt{snoc} & Adds the given element to the end of the \texttt{Iterable}. It is the opposite of \texttt{cons}.\\ \hline 
    \texttt{take} & Stop after so many elements. \\ \hline 
    \texttt{takeWhile} & Proceeds with the iteration until the predicate becomes true. \\ \hline 
    \texttt{zip} & Zip takes two iterables and returns an \texttt{Iterable} of corresponding pairs. \\ \hline 
    \texttt{zipWith} & Generalises \texttt{zip} by zipping with the function given as the first argument. \\ \hline 
  \end{tabularx}
  \caption{The available operators in the Sequence library}
  \label{tab:api_operators}
\end{table}

\subsubsection{Using the pipe Operator} % (fold) 
\label{subsub:Using the pipe Operator}
The operator |pipe|, is unique because it provides no new functionality but is
pure syntactic sugar. It offers the possibility of combining several operators.
Lines~\ref{line:api_pipe1}~-~\ref{line:api_pipe2} of listing~\ref{lst:api_pipe}
shows nesting of multiple operators. Readability suffers very much from this.
In comparison, it is easier to read the code
lines~\ref{line:api_pipe3}~-~\ref{line:api_pipe4}, combining multiple operators
using |pipe|. 
\begin{lstlisting}[
  style=ES6,
  caption=pipe combines multiple operators,
  label={lst:api_pipe}
]
import * as _ from "./src/sequence/sequence.js"

const numbers = _.Sequence(0, _ => true, x => x + 1);

const mySequence = _.take(5)( *'\label{line:api_pipe1}'*
  _.retainAll(x => x > 50)(
    _.map(x => 2 * x)(
      _.rejectAll(x => x % 2 === 1)(numbers)
    )
  )
);*'\label{line:api_pipe2}'*

console.log(...mySequence);
// => Logs '52, 56, 60, 64, 68'

const mySequence2 = _.pipe(*'\label{line:api_pipe3}'*
  _.rejectAll(x => x % 2 === 1),
  _.map(x => 2 * x),
  _.retainAll(x => x > 50),
  _.take(5)
)(numbers); *'\label{line:api_pipe4}'*

console.log(...mySequence2);
// => Logs '52, 56, 60, 64, 68'
\end{lstlisting}

As listing~\ref{lst:api_pipe_proto} shows, using the version of |pipe| defined
on the prototype of the sequence, the readability improves even more:
\begin{lstlisting}[
  style=ES6,
  caption=Using pipe defined on the prototype,
  label={lst:api_pipe_proto}
]
const mySequence3 = seq.pipe(
  _.rejectAll(x => x % 2 === 1),
  _.map(x => 2 * x),
  _.retainAll(x => x > 50),
  _.take(5)
);
console.log(...mySequence3);
// => Logs '52, 56, 60, 64, 68'
\end{lstlisting}

% subsubsection Using the pipe Operator (end)
% section Operators (end)

\subsection{Terminal Operations} % (fold)
\label{sub:api_Terminal Operations}
Terminal operations are all functions that operate on an existing sequence
and do not necessarily create a new sequence. In other words, terminal
operations evaluate a sequence.\\ 
Table~\ref{tab:api_term_ops} gives an overview of all available terminal
operations.\\
Code examples and more information about the terminal operations deliver the
appendix~\ref{sec:appendix_terminal_operations}.

\begin{table}[H]
  \centering
  \begin{tabularx}{\textwidth}{| l | X |} \hline
    \textbf{Name} & \textbf{Description} \\ \hline
    \texttt{eq\$} & Checks the equality of two finite iterables. \\ \hline 
    \texttt{foldr} & Performs a reduction on the elements from right to left, using the provided start value and an accumulation function, and returns the reduced value. \\ \hline 
    \texttt{forEach\$} & Executes the callback for each element. \\ \hline 
    \texttt{head} & Return the next value without consuming it. \texttt{undefined} when there is no value. \\ \hline 
    \texttt{isEmpty} & Returns \texttt{true}, if the iterables head is undefined. \\ \hline 
    \texttt{max\$} & Returns the largest element of a \textbf{non-empty} \texttt{Iterable}. \\ \hline 
    \texttt{safeMax\$} & Returns the largest element of an \texttt{Iterable}. \\ \hline 
    \texttt{min\$} & Returns the smallest element of a \textbf{non-empty} \texttt{Iterable}. \\ \hline 
    \texttt{safeMin\$} & Returns the smallest element of an \texttt{Iterable}. \\ \hline 
    \texttt{foldl\$} & Performs a reduction on the elements, using the provided start value and an accumulation function, and returns the reduced value. \\ \hline 
    \texttt{show} & Transforms the passed \texttt{Iterable} to a \texttt{String}. \\ \hline 
    \texttt{uncons} & Removes the first element of this \texttt{Iterable}. \\ \hline 
  \end{tabularx}
  \caption{The available constructors of the Sequence library}
  \label{tab:api_term_ops}
\end{table}
% section Terminal Operations (end)

\subsection{Sequence Prototype} % (fold)
\label{sub:Sequence Prototype}
The prototype of the sequence provides some operations as well. This improves
code readability in some situations. The monadic operations are also available
on the prototype of the sequence. Thus, it is compatible with all functions
that expect a monad as a parameter.

\subsubsection{Verify the Prototype} % (fold)
\label{subsub:Verify the Prototype}
Listing~\ref{lst:proto_seq_query} shows how to query the prototype of any
object using the function |Object.getPrototypeOf|. Applying this to any
sequence will return |SequencePrototype|.

\begin{lstlisting}[
  style=ES6,
  caption=The prototype of a \lstinline{Sequence},
  label={lst:proto_seq_query}
]
import { Sequence, SequencePrototype } from "sequence/sequence.js";

const seq          = Sequence(0, _ => true, i => i + 1);
const seqPrototype = Object.getPrototypeOf(seq);

console.log(seqPrototype === SequencePrototype);
// => Logs 'true'
\end{lstlisting}

% subsubsection Verify the Prototype (end)

\subsubsection{The Operations on the Prototype} % (fold)
\label{subsub:The operations on the Prototype}
\begin{table}[H]
  \centering
  \begin{tabularx}{\textwidth}{| l | X |} \hline
    \textbf{Name} & \textbf{Description} \\ \hline
    \texttt{fmap} & the same as \lstinline{map} described in section~\ref{sub:api_Operators}\\ \hline 
    \texttt{and} & the same as \lstinline{bind} described in section~\ref{sub:api_Operators}\\ \hline 
    \texttt{pure} & the same as \lstinline{PureSequence} described in section~\ref{sub:api_Constructors}\\ \hline 
    \texttt{empty} & the same as \lstinline{nil} described in section~\ref{sub:api_Constructors}\\ \hline 
    \texttt{["=="]} & the same as \lstinline{eq$} described in section~\ref{sub:api_Terminal Operations}\\ \hline 
    \texttt{toString} & the same as \lstinline{show} described in section~\ref{sub:api_Terminal Operations}\\ \hline 
    \texttt{pipe} & the same as \lstinline{pipe} described in section~\ref{sub:api_Operators}\\ \hline 
  \end{tabularx}
  \caption{The operations served on the prototype of the \lstinline{Sequence}}
  \label{tab:prototype_operations}
\end{table}

% subsubsection The operations on the Prototype (end)
\subsubsection{When to use the Operations on the Prototype?} % (fold)
\label{subsub:When to use the operations on the prototype?}
Each operation on the prototype is also available as an operator or terminal
operation. The prototype offers these functionalities additionally because
sometimes it has additional advantages - the function |toString|, for example,
converts by convention an object into its string representation. Other
operations, such as |["=="]|, often make the code more readable. So when to use
the prototype operations? It depends on the context and code readability!
% subsubsection When to use the operations on the prototype? (end)
% subsection Sequence Prototype (end)
% section Sequence (end)


