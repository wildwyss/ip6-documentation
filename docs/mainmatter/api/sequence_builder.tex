\section{Sequence Builder}
\label{sec:Sequence Builder}
A mutable builder to create sequences.
\newline
|SequenceBuilder| allows the creation of a sequence by generating elements individually 
and adding them to the |SequenceBuilder| (without the call stack overhead when doing so
with |cons|).\\
\\
\textit{Note:} It is strongly recommended to use |SequenceBuilder| to create
Sequences with more than 1000 elements when adding them individually.
\subsection{Functions}
\label{sub:sb_Functions}

|SequenceBuilder| provides the following functions:

\begin{table}[H]
  \centering
  \begin{tabularx}{\textwidth}{| l | X |} \hline
    \textbf{Name}    & \textbf{Description} \\ \hline
    \texttt{prepend} & adds one or more elements or a sequence to the beginning of the sequence\\ \hline 
    \texttt{append}  & adds one or more elements or a sequence to the end of the sequence\\ \hline 
    \texttt{build}   & builds a sequence containing the previously passed elements \\ \hline 
   \end{tabularx}
  \caption{SequenceBuilder functions}
  \label{tab:sb_functions}
\end{table}

\subsection{Examples}
\label{sub:sb_Examples}
Listing~\ref{lst:sb_example} creates a sequence from different values and iterables:

\begin{lstlisting}[
  style=ES6, 
  caption=SequenceBuilder example,
  label={lst:sb_example}
  ]
import * as _              from "./src/sequence/sequence.js"
import { SequenceBuilder } from "./src/sequence/SequenceBuilder.js";

const range = _.Range(1, 3);

const result = SequenceBuilder()
.append(range)
.append(4)
.prepend(0)
.append(5,6,7)
.build();

console.log(_.show(result));
// => Logs '[0,1,2,3,4,5,6,7]'
\end{lstlisting}
