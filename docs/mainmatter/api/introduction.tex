\section{Introduction} % (fold)
\label{sec:Introduction}

\subsection{Library Overview} % (fold)
\label{sub:Sequence Library Overview}
This library consists of several parts. Table~\ref{tab:library_overview} gives a brief
overview of each of those. The following sections discuss these different
aspects.
% subsection Sequence Library Overview (end)

\begin{table}[H]
  \centering
  \begin{tabularx}{\textwidth}{| l | X |} \hline
    \textbf{Part} & \textbf{Description} \\ \hline
    Sequence library & The Sequence library provides operations for processing \lstinline{Iterable}s. Additionally, there are constructors to create Sequences.\\ \hline 
    JINQ & A range generates a sequence of numbers. \\ \hline 
    FocusRing & A FocusRing is an immutable data structure, whose elements are arranged in a ring. \\ \hline 
    Range & A Range generates a sequence of numbers. \\ \hline 
  \end{tabularx}
  \caption{The available operators in the Sequence library}
  \label{tab:library_overview}
\end{table}

\textit{Note:} The Range and the FocusRing are the results of the predecessor
project IP5 and are only listed here for completeness. You can find more
information in the IP5 documentation~\cite{wild_ip5_2023}. 

\subsection{Including the Library in a Project} % (fold)
This library is part of the Kolibri project and uses many of its
functionalities. The Kolibri ships this library, therefore. The Kolibri
is available on GitHub and can be included directly in existing projects via
the ES6 module system.
\label{sub:Including the Library in a Project}
% subsection Including the Sequence Library in a Project (end)
% section Introduction (end)
