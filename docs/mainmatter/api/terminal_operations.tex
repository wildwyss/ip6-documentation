\section{Terminal Operations} % (fold)
\label{sec:Terminal Operations}
Terminal operations are all functions that operate on an existing |Sequence|
and do not necessarily create a new |Sequence| from it. In other words,
terminal operations evaluate a |Sequence|.\\ 
Table~\ref{tab:api_term_ops} gives an overview of all available terminal
operations.\\
Code examples and more information about the constructors delivers the
appendix~\ref{sub:appendix_terminal_operations}

\begin{table}[H]
  \centering
  \begin{tabularx}{\textwidth}{| l | X |} \hline
    \textbf{Name} & \textbf{Description} \\ \hline
    \texttt{eq\$} & Checks the equality of two finite \texttt{Iterable}s. \\ \hline 
    \texttt{foldr} & Performs a reduction on the elements from right to left, using the provided start value and an accumulation function, and returns the reduced value. \\ \hline 
    \texttt{forEach\$} & Executes the callback for each element. \\ \hline 
    \texttt{head} & Return the next value without consuming it. \texttt{undefined} when there is no value. \\ \hline 
    \texttt{isEmpty} & Returns \texttt{true}, if the iterables head is undefined. \\ \hline 
    \texttt{max\$} & Returns the largest element of a \textbf{non-empty} \texttt{Iterable}. \\ \hline 
    \texttt{safeMax\$} & Returns the largest element of an \texttt{Iterable}. \\ \hline 
    \texttt{min\$} & Returns the smallest element of a \textbf{non-empty} \{@link Iterable\}. \\ \hline 
    \texttt{safeMin\$} & Returns the smallest element of an \texttt{Iterable}. \\ \hline 
    \texttt{foldl\$} & Performs a reduction on the elements, using the provided start value and an accumulation function, and returns the reduced value. \\ \hline 
    \texttt{show} & Transforms the passed \texttt{Iterable} to a \texttt{String}. \\ \hline 
    \texttt{uncons} & Removes the first element of this iterable. \\ \hline 

  \end{tabularx}
  \caption{The available constructors of the Sequence library}
  \label{tab:api_term_ops}
\end{table}
% section Terminal Operations (end)
