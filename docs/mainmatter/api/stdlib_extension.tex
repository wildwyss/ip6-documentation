\section{Extension of the Kolibri Standard Library}
\label{sec:Extension of the Kolibri Standard Library}
From this project's findings, some existing functionalities of the Kolibri
could also be improved or extended. This section describes those extensions.

\subsection{Extensions to Maybe} % (fold)
\label{sub:Extensions to Maybe}
The data structure |Maybe| introduced in Section~\ref{sub:Wrapping values in a
context} was already part of the Kolibri before this project work. Now it also
supports monadic operations.to the Sequence, |Maybe| has a prototype, providing
the monadic functions. \\ 
Listing~\ref{lst:monadic_ops_maybe_in_action} shows these monadic operations on
Maybe in action:

\begin{lstlisting}[
  style=ES6,
  caption=The monadic operations of Maybe in action,
  label={lst:monadic_ops_maybe_in_action}
]
import { Nothing, Just } from "./src/stdlib/maybe.js";

/** Prints the value of a Maybe if it exists */
const evalMaybe = maybe =>
  maybe
    (_ => console.log("There was no value!"))
    (x => console.log(x));

const just    = Just(5);
const nothing = Nothing;

// fmap 
const justMapped    = just   .fmap(x => 2*x); // results in 10
const nothingMapped = nothing.fmap(x => 2*x); // Mapping Nothing - nothing happens!
evalMaybe(justMapped);      // => Logs '10'
evalMaybe(nothingMapped);   // => Logs 'There was no value!' 

// and
const justAnd    = just   .and(x => nothing); // Turns this value into Nothing
const nothingAnd = nothing.and(x => just);    // Can't change Nothing
evalMaybe(justAnd);         // => Logs 'There was no value!'
evalMaybe(nothingAnd);      // => Logs 'There was no value!' 

// pure
evalMaybe(just.pure(2));    // => Logs '2', Same as Just(2)
evalMaybe(nothing.pure(2)); // => Logs '2', Same as Just(2)

// empty
evalMaybe(just.empty());    // => Logs 'There was no value!', Same as Nothing
evalMaybe(nothing.empty()); // => Logs 'There was no value!', Same as Nothing
\end{lstlisting}
% subsection Extensions to Maybe (end)

\subsection{Extensions to Pair} % (fold)
\label{sub:Extensions to Pair}
Section~\ref{sec:Iterables Everywhere} describes the |Pair| as an immutable
data structure. To access its values simpler, it is now iterable.
Listing~\ref{lst:api_it_pair} shows these advantages:

\begin{lstlisting}[
  style=ES6,
  caption=Iterable Pair,
  label={lst:api_it_pair}
]
import { Pair } from "./src/stdlib/pair.js"

const pair = Pair(1)(2);
console.log(...pair);
// => Logs '1, 2'
\end{lstlisting}

Since the |Pair| is iterable, it is also compatible with all operations defined
for sequences:

\begin{lstlisting}[
  style=ES6,
  caption=Transforming a \lstinline{Pair} using \lstinline{map},
  label={lst:Mapping the values of a Pair}
]
import { Pair } from "./src/stdlib/pair.js"
import { map }  from "./src/sequence/sequence.js"

const pair = Pair(1)(2);
console.log(...map(x => 2*x)(pair));
// => Logs '2, 4'
\end{lstlisting}


% subsection Extensions to Pair (end)
