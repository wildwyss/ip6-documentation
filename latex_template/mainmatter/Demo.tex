\chapter{Demo}\label{chp:demo}
\section{aaa}\label{sec:aaa}
\subsection{bbb}
\subsubsection{ccc}


Demonstration der Quellenangaben, Akronym- und Symbol- Referenzierung

Die Farben der Verlinkungen können im File header.sty geändert werden.

Eine einfache Referenz ins Literaturverzeichnis\cite{einstein}\\

\textcquote{einstein}{Dies ist ein Zitat}\\
\blockcquote{einstein}{Dieses Zitat ist im Displaymode}

nachher kommt die Fussnote\footnote{footnotes working fine}\\

Hier verwende ich das Symbol \gls{pi},\gls{rho} und \gls{p} das geht auch im Mathematik Modus \\

$$\gls{rho}$$

Und hier eine Abkürzung \gls{eth}\\
Bei der zweiten Verwendung steht nur noch die Abkürzung ohne Erklärung \gls{eth}\\[5mm]

Einheiten werdern so verwendet: \SI{2}{m}


Eine Aufzählung:
\begin{itemize}
    \item Bla
    \item BlaBla
\end{itemize}


\begin{figure}[H]
    \centering
    \includegraphics[width=0.25\textwidth]{./mainmatter/pictures/Demo/Mona_Lisa}
    \caption{a nice picture}
    \label{img:MonaLisa}
\end{figure}


\begin{table}[H]
\centering
\begin{tabular}{|c|c|c|}
	\hline & Messbereich & Auflösung\\ 
	\hline Fx & 80N   & 1/50N\\ 
	\hline Fy & 80N   & 1/50N\\ 
	\hline Fz & 240N  & 1/25N\\ 
	\hline Mx & 4Nm   & 1/2000Nm\\ 
	\hline My & 4Nm   & 1/2000Nm\\ 
	\hline Mz & 4Nm   & 1/2000Nm\\
	\hline 
\end{tabular} 
\caption{das ist eine Tabelle}
\label{tbl:Tabelle}
\end{table}


Ich kann auch auf das Bild verweisen (\autoref{img:MonaLisa})\\
oder auf den Anhang: \autoref{app:codeofconduct} \\
oder auf das Kapitel: \autoref{chp:demo}\\
oder auf die section: \autoref{sec:aaa}\\
Wenn ich nur die Nummer ohen Keyword will, dann nehme ich \ref{chp:demo}
